\chapter{Conclusões} 
% ----------------------------------------------------------

O principal objetivo deste trabalho foi o desenvolvimento e a implementação uma metodologia numérica robusta para a análise tridimensional de problemas de interação fluido-estrutura (IFE) no contexto da análise isogeométrica e do método dos elementos finitos, onde o domínio fluido é representado por uma discretização global fixa à qual é superposta uma discretização local móvel e conforme aos contornos da estrutura, de modo a captar tanto a movimentação da estrutura como os efeitos localizados em torno da interface fluido-estrutura com eficiência computacional, permitindo ainda que sejam combinadas as discretizações isogeométrica e por elementos finitos. Embora a análise numérica de interação fluido-estrutura já tenha certa maturidade, ainda há grandes desafios, especialmente ao se tratar de problemas onde efeitos localizados, como os de camada limite, ocorrem junto a estruturas tridimensionais submetidas a regimes de grandes deslocamentos, especialmente nos casos de grandes rotações de corpo rígido que impedem que a deformação da malha do fluido seja suficiente para acomodar a movimentação da estrutura, o que motiva este trabalho.

O presente trabalho teve início com o desenvolvimento de um código computacional para a análise tridimensional de escoamentos isotérmicos, incompressíveis e com contornos móveis, empregando uma formulação estabilizada. A discretização espacial foi implementada de modo a permitir o uso tanto do Método dos Elementos Finitos quanto da Análise Isogeométrica, com integração temporal implícita. A verificação inicial do código foi realizada a partir de exemplos clássicos da literatura.
Na sequência, implementou-se no código de dinâmica dos fluidos uma técnica de partição de domínio denominada método de combinação de espaços de funções, aplicada inicialmente a domínios fixos. Embora eficiente em outros contextos, essa técnica apresentou limitações em testes realizados para a formulação estabilizada em escoamentos incompressíveis com altos números de Reynolds. Assim, desenvolveu-se e avaliou-se uma segunda estratégia, o Método Arlequin estabilizado, primeiramente testado em domínios fixos e posteriormente estendido a problemas com contornos móveis, de maneira a permitir o acoplamento entre malhas globais fixas e malhas locais móveis.
Por fim, prosseguiu-se com o estudo de uma metodologia para a dinâmica dos sólidos computacional, baseada em uma versão posicional do Método dos Elementos Finitos aplicada a estruturas de casca. Os códigos de fluido e de estrutura foram então acoplados por meio de um acoplamento particionado do tipo forte. A ferramenta computacional para análise de interação fluido-estrutura foi verificação por meio de exemplos numéricos \textit{benchmarks}. Além disso, propôs-se um exemplo qualitativo para demonstrar a potencialidade do método.

O programa computacional para escoamentos incompressíveis foi desenvolvido com base em uma formulação estabilizada, fundamentada nas técnicas SUPG (Streamline Upwind Petrov–Galerkin), PSPG (Pressure-Stabilizing Petrov–Galerkin) e LSIC (Least-Squares Incompressibility Constraint), metodologias amplamente reconhecidas e consolidadas na literatura científica. As equações de Navier–Stokes foram expressas por meio da descrição Arbitrária Euleriana–Lagrangiana (ALE), de forma a permitir a análise de problemas com contornos móveis posteriormente.

A discretização espacial inicialmente consistiu no emprego de elementos finitos quadráticos. Para problemas bidimensionais foram empregados elementos triangulares de seis nós e para problemas tridimensionais elementos tetraédricos de dez nós. Para discretização temporal, uma técnica de marcha no tempo implícita foi aplicada, que consiste no método $\alpha$-generalizado. Essa estratégia de integração temporal é escolhida por permitir um maior controle da dissipação numérica adicionada ao processo de solução, e ao mesmo tempo, preservar a convergência de segunda ordem. 

A implementação da análise isogeométrica no código de escoamentos incompressíveis tridimensionais teve como objetivo explorar as vantagens inerentes à isogeometria, especialmente a maior continuidade das funções de base e a representação geométrica exata dos domínios. Neste trabalho, adotaram-se funções do tipo NURBS, de forma a garantir essas propriedades. Na abordagem isogeométrica, utilizaram-se funções de base quadráticas, que resultam em células quadrilaterais com nove pontos de controle nos casos bidimensionais e hexaedros com vinte e sete pontos de controle nos casos tridimensionais.

A adaptação do código originalmente baseado no Método dos Elementos Finitos para possibilitar o uso da Análise Isogeométrica demandou modificações essenciais principalmente nas etapas de representação geométrica e pré-processamento, avaliação das funções de base, integração numérica, aplicação das condições de contorno e pós-processamento. Além disso, optou-se pela utilização de parâmetros de estabilização específicos para discretizações isogeométricas, com base em recomendações da literatura, que apontam para um melhor desempenho numérico nessas formulações. 

A verificação do código desenvolvido para escoamentos incompressíveis foi conduzida por meio da análise de diversos exemplos referências, de forma a assegurar precisão dos resultados e permitir a realização dos avanços sequentes.

Com o intuito de incorporar uma técnica de partição de domínios ao código de escoamento, de modo a capturar efeitos localizados, estudou-se e implementou-se a técnica de combinação dos espaços de funções. Essa abordagem já havia sido aplicada com sucesso no contexto da fratura elástico-linear com grandes deslocamentos, em trabalhos desenvolvidos concomitantemente a esta tese.
Entretanto, para escoamentos incompressíveis formulados com técnicas de estabilização, verificou-se que a metodologia apresentava bom desempenho para baixos números de Reynolds, mas que instabilidades nos campos de velocidade e pressão surgiam na região de superposição à medida que o número de Reynolds aumentava. Após diversas análises, constatou-se que ajustes na magnitude dos parâmetros de estabilização podiam melhorar a resposta numérica nessa região. Ainda assim, não foi possível determinar uma forma otimizada e automatizada de definição desses parâmetros que garantisse resultados satisfatórios para todos os casos analisados.

Nesse contexto, com o objetivo de viabilizar o desenvolvimento do código com uma técnica de partição de domínios, optou-se pela utilização da formulação estabilizada do método Arlequin.
A formulação estabilizada proposta por \citeonline{FernandesEtAll:2020}, denominada RBSAM, baseia-se na introdução de um termo estabilizante construído a partir do gradiente do resíduo da equação governante, sendo formulada sobre o operador de acoplamento $L^{2}$. O autor realizou verificações numéricas para problemas unidimensionais e bidimensionais, demonstrando a robustez do método, além de verificar que, dependendo do parâmetro de estabilização empregado, é possível obter um sistema algébrico com número de condicionamento comparável ao do operador $H^{1}$. 

Baseado nesses resultados, adotou-se a formulação RBSAM neste trabalho, com o propósito de estendê-la para o contexto tridimensional. Dada a relevância do parâmetro de estabilização ($\tauArlequin$), propôs-se — com base nos trabalhos de \citeonline{TezduyarO:2000} e \citeonline{TezduyarS:2003} — a definição de termos de estabilização cuja magnitude fosse compatível com os termos da equação de acoplamento, por meio do emprego de normas vetoriais. A implementação foi inicialmente verificada em simulações com domínios fixos.

Motivado pelo trabalho realizado durante o doutorado sanduíche \cite{Tononetal:2021}, optou-se pela implementação da técnica de movimentação de malha MJBS  (\textit{Mesh-Jacobian Based Stiffening}), apresentada por \citeonline{TezduyarBSJ:1992c} e \citeonline{TezduyarABJ:1993}, que resolve a movimentação da malha a partir da solução de um problema de elasticidade de Dirichlet fictício. A técnica atribui a cada elemento uma rigidez diferente, que visa preservar os aspectos dos elementos menores, impedindo inversão de elementos ou que elementos assumam volume muito pequeno. A partir dessa implementação, estendeu-se e verificou-se a formulação do método de Arlequin estabilizado para problemas com contornos móveis. Nesse cenário, a malha local é formulada com uma descrição em ALE utilizando-se a técnica MJBS, enquanto que a malha global apresenta descrição Euleriana e permanece fixa ao longo de toda análise.

A estrutura é modelada por elementos de casca com cinemática de Reissner–Mindlin, empregando uma formulação posicional do método dos elementos finitos em descrição Lagrangiana total. Essa abordagem é adequada à análise dinâmica com grandes deslocamentos e apresenta a vantagem de não utilizar rotações como parâmetros nodais \cite{CodaP:2007,SanchesC:2013}. A formulação, concebida no contexto isoparamétrico, considera naturalmente os efeitos de não linearidade geométrica. Na discretização espacial adotaram-se elementos triangulares quadráticos de seis nós, enquanto a integração temporal é implícita, utilizando o integrador de Newmark. Essa formulação, amplamente verificada na literatura, foi adotada por atender de forma eficaz às necessidades requeridas ao desenvolvimento deste estudo.

No que diz respeito à interação fluido–estrutura, adotou-se uma metodologia de particionamento forte do tipo bloco-iterativo, visando o completo desacoplamento entre os \textit{solvers} de fluido e estrutura. O acoplamento entre fluido e estrutura, quando aplicado a monomodelos (domínio do fluido definido por apenas uma discretização), foi realizado por meio da descrição ALE. Nos casos em que ocorre o particionamento do domínio do fluido — ou seja, a superposição de uma malha local à uma malha global —, aplicou-se uma técnica híbrida Euleriana–ALE, de modo que apenas a malha local, em contato com a estrutura, se movimente para acomodar as deformações e mudanças de configuração da estrutura, enquanto a malha global permanece fixa.

Devido ao esquema particionado, as condições cinemáticas e dinâmicas associadas ao acoplamento entre os meios são transferidas por meio de uma formulação do tipo Dirichlet–Neumann na interface fluido–sólido. O código desenvolvido permite o acoplamento entre interfaces com discretizações não coincidentes por meio de uma técnica de projeção, na qual os nós do contorno de uma malha são projetados sobre o contorno da outra (fluido e estrutura), estabelecendo correspondências entre pontos não coincidentes. A partir dessas projeções, são determinadas as coordenadas paramétricas equivalentes em cada domínio, o que possibilita impor corretamente as condições de contorno na interface fluido–estrutura, mesmo quando as discretizações são distintas.

A formulação foi verificada em problemas de interação fluido–estrutura (IFE) por meio da comparação entre os resultados obtidos com a formulação particionada e aqueles obtidos com os monomodelos. Além disso, os resultados foram confrontados com dados de referência disponíveis na literatura.

O primeiro problema testado consistiu em uma cavidade com velocidade oscilatória prescrita na parede superior, cujo fundo é composto por um painel flexível. Esse exemplo foi escolhido por possuir versões bidimensional e tridimensional, o que permitiu uma verificação inicial abrangente da formulação. Ressalta-se que, nesse caso, observou-se inicialmente a divergência dos resultados ao aplicar a técnica bloco-iterativa; tal problema foi contornado com a adoção da metodologia \textit{Augmented A22}, multiplicando-se a parcela correspondente da matriz tangente, referente à matriz de massa da estrutura, por um fator igual a 2.

Em seguida, foi analisado um problema bidimensional, discretizado por elementos 3D, constituído por um painel flexível engastado a um bloco prismático. Esse exemplo foi selecionado devido as características de instabilidade dinâmica autoexcitátoria apresentadas, isto é, envolve fenômenos complexos de interação fluido–estrutura (IFE). Os resultados obtidos foram bastante satisfatórios e evidenciaram as vantagens do uso de uma malha móvel refinada apenas na região da estrutura, permitindo capturar os efeitos localizados da camada limite e, simultaneamente, facilitar a movimentação adequado da malha ao longo de todo a análise, devido aos contornos externos deformáveis da malha local.

Por fim, foi proposto um exemplo tridimensional de caráter mais prático, envolvendo uma turbina hidráulica - composta por pás diretoras rígidas e fixas e um rotor - submetida a uma pressão superior à atmosférica na entrada. Esse exemplo foi inicialmente simulado sem as pás diretoras rígidas, utilizando um monomodelo baseado no método dos elementos finitos em descrição ALE. No entanto, para a aplicação completa, a simulação não pôde ser realizada diretamente com métodos de malha móvel. Nesse cenário, adotou-se a partição de domínios com o método RBSAM, com as pás diretoras representadas por meio de uma discretização global isogeométrica, e, inserindo-se o rotor através de uma malha local em elementos finitos. Embora avaliado de forma qualitativa, esse exemplo confirmou o potencial da metodologia proposta para o tratamento de estruturas sujeitas a grandes rotações de corpo rígido. 

De forma geral, os resultados obtidos neste trabalho demonstram a consistência, robustez e potencial de aplicação da metodologia proposta para a análise tridimensional de problemas complexos de interação fluido–estrutura. As principais contribuições originais desta tese concentram-se na integração entre a Análise Isogeométrica e o Método dos Elementos Finitos em um arcabouço particionado com malhas superpostas, associada à extensão do método Arlequin estabilizado (RBSAM) para escoamentos incompressíveis com contornos móveis tridimensionais. Além disso, a metodologia proposta combina as vantagens inerentes aos métodos baseados em malhas móveis e aos métodos imersos. Enquanto a formulação ALE assegura uma descrição precisa da interface e da cinemática da estrutura, a estratégia particionada com malhas superpostas confere flexibilidade e robustez frente a grandes deslocamentos e rotações, reduzindo significativamente a necessidade de remalhamentos ou tratamentos complexos de distorção de malha — que, quando necessários, ficam restritos apenas a regiões localizadas do domínio. Assim, esta pesquisa representa um avanço relevante tanto no campo teórico quanto no computacional, contribuindo para o desenvolvimento de técnicas numéricas mais eficientes e integradas para a modelagem e simulação de fenômenos complexos de interação fluido–estrutura.

\section{Sugestão para trabalhos futuros}

Como proposta para trabalhos futuros voltados à \textbf{dinâmica dos fluidos computacional}, recomenda-se a extensão da metodologia para a análise de escoamentos turbulentos, por meio da incorporação de modelos de turbulência do tipo LES (Large Eddy Simulation) ou RANS (Reynolds-Averaged Navier–Stokes). Essa ampliação é especialmente relevante considerando que muitos problemas de interação fluido–estrutura de interesse prático envolvem regimes turbulentos, nos quais a modelagem direta (DNS) é inviável do ponto de vista computacional.

No contexto da \textbf{Análise Isogeométrica}, uma possível linha de avanço consiste na integração do código desenvolvido com plataformas de modelagem geométrica consolidadas, como o Rhinoceros 3D, por meio da leitura direta de modelos NURBS exportados em formatos padronizados, como IGES ou STEP. Essa integração permitiria explorar geometrias mais complexas e realistas, eliminando a necessidade de geração manual das superfícies e volumes utilizados na discretização isogeométrica.

\textcolor{red}{Sugestão do chat - Eu particularmente não entendo direito sobre os métodos VMS}
No que diz respeito a \textbf{técnica de partição de domínios baseado na combinação dos espaços de funções} sugere-se como uma possível alternativa, para contornar as dificuldades observadas em formulações estabilizadas para escoamentos incompressíveis com altos números de Reynolds, a adoção de formulações baseadas em escalas variacionais (VMS – Variational Multiscale Methods). Nessa abordagem, o campo de velocidades e pressões é decomposto em componentes resolvidas e submalha, e o efeito das escalas não resolvidas é modelado de forma consistente a partir do resíduo das equações governantes. Dessa forma, a estabilização passa a emergir naturalmente da modelagem multiescala, reduzindo a dependência de parâmetros empíricos típicos das técnicas SUPG e PSPG. Além disso, as formulações VMS podem adaptar a dissipação numérica conforme a resolução local e as características do escoamento, o que as torna particularmente adequadas para regiões de sobreposição entre malhas, onde a calibração dos parâmetros de estabilização é mais sensível. 

Com relação ao \textbf{método Arlequin estabilizado} tem-se como sugestão, aquela apresentada por \citeonline{Fernandes:2020} em sua tese de doutorado. O autor cita, que embora o sistema algébrico resultante da inserção do termo de estabilização ao operador de acoplamento tenha seu condicionamento potencialmente melhorado, outros aspectos do método Arlequin também são fonte de mal condicionamento, como o valor da constante kA, por exemplo. Desta forma, o autor sugere o uso de uma técnica eficiente desenvolvida em trabalhos prévios, como os de \citeonline{DhiaER:2008} e \citeonline{SchlittlerC:2017} no contexto da computação de alto desempenho (HPC - High Performance Computing). Em ambos os trabalhos, os autores empregaram o algoritmo FETI (Finite Element Tearing and Interconnect), que tem como princípio o particionamento do sistema algébrico resultante em subproblemas. Desse modo, técnicas eficientes para resolução de problemas de escoamentos incompressíveis, como métodos iterativos como o GMRES (Generalized Minimum Residual Method ) aliado à pré-condicionadores apropriados, podem ser aproveitadas. 

Ainda, no que se refere ao \textbf{método Arlequin estabilizado} propõe-se explorar o aspecto das possibilidades de combinações das discretizações global e local, como por exemplo, ao utilizar-se ambas discretizações isogeométricas, ou ambas em elementos finitos, ou ainda, a malha local discetizada em isogeometria, enquanto que a malha global é definida por elementos finitos.

No contexto da \textbf{dinâmica dos sólidos computacionais}, aponta-se como alternativa à técnica de integração temporal atualmente empregada a implementação do método $\alpha$-generalizado no código computacional desenvolvido, de modo a dissipar seletivamente as componentes de alta frequência, preservando a precisão de segunda ordem no tempo. Além disso, para os problemas de FSI, o uso dos mesmos integradores temporais pode resultar em uma melhora de convergência e estabilidade do código. Ademais, propõe-se a exploração da discretização isogeométrica no contexto da estrutura.

Por fim, no contexto da \textbf{interação fluido–estrutura}, propõe-se a aplicação da técnica de relaxação de Aitken ao método de particionamento forte do tipo bloco iterativo. Essa técnica numérica pode acelerar a convergência (ou evitar a divergência) no processo iterativo de acoplamento entre o fluido e a estrutura. Em vez de atualizar diretamente as condições na interface com o valor obtido em cada nova iteração, realiza-se uma combinação ponderada entre os valores atual e anterior, utilizando um fator de relaxação adaptativo de Aitken, calculado automaticamente a partir do histórico do erro iterativo. Além disso, a adoção dessa estratégia elimina a necessidade do uso da metodologia \textit{Augmented A22}, uma vez que esta exige a escolha manual de um parâmetro multiplicador associado à parcela correspondente à matriz de massa da estrutura na matriz tangente global.
