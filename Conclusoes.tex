\chapter{Conclusão} 
% ----------------------------------------------------------

O principal objetivo deste trabalho foi o desenvolvimento e a implementação de uma metodologia numérica robusta para a análise tridimensional de problemas de interação fluido–estrutura (IFE) no contexto da análise isogeométrica e do método dos elementos finitos, em que o domínio fluido é representado por uma discretização global fixa, sobre a qual é superposta uma discretização local móvel e conforme aos contornos da estrutura. Essa abordagem permite captar simultaneamente a movimentação da estrutura e os efeitos localizados em torno da interface fluido–estrutura, com eficiência computacional, possibilitando ainda a combinação entre as discretizações isogeométrica e por elementos finitos.

Embora a análise numérica de interação fluido–estrutura já tenha alcançado certa maturidade, persistem desafios significativos, especialmente em problemas onde efeitos localizados, como os de camada limite, ocorrem junto a estruturas tridimensionais submetidas a grandes deslocamentos e rotações de corpo rígido, e especialmente nos casos de grandes rotações de corpo rígido que tornam impossível acomodar a movimentação da estrutura apenas defermando-se a malha do fluido. Esses desafios motivaram o desenvolvimento desta pesquisa.

O presente trabalho teve início com o desenvolvimento de um código computacional para a análise tridimensional de escoamentos isotérmicos, incompressíveis e com contornos móveis, empregando uma formulação estabilizada. A discretização espacial foi implementada de modo a permitir o uso tanto do método dos elementos finitos quanto da análise isogeométrica, com integração temporal implícita. 

O programa computacional para escoamentos incompressíveis foi desenvolvido com base em uma formulação estabilizada fundamentada nas técnicas SUPG (Streamline Upwind Petrov–Galerkin), PSPG (Pressure-Stabilizing Petrov–Galerkin) e LSIC (Least-Squares Incompressibility Constraint) — metodologias amplamente reconhecidas e consolidadas na literatura científica. As equações de Navier–Stokes foram expressas por meio da descrição Arbitrária Euleriana–Lagrangiana (ALE), de modo a permitir a análise de problemas com contornos móveis.

No contexto do método dos elementos finitos, a discretização espacial adotou aproximações quadráticas, utilizando elementos triangulares de seis nós em 2D e elementos tetraédricos de dez nós em 3D. A discretização temporal foi realizada por meio de uma técnica implícita de marcha no tempo, o método $\alpha$-generalizado, escolhido por permitir maior controle da dissipação numérica e, simultaneamente, preservar a precisão de segunda ordem. A verificação inicial do código foi realizada a partir de exemplos clássicos da literatura.

A implementação da análise isogeométrica teve como objetivo explorar as vantagens inerentes à isogeometria, especialmente a maior continuidade das funções de base e a representação geométrica exata dos domínios. Neste trabalho, foram utilizadas funções NURBS quadráticas, resultando em células quadrilaterais com nove pontos de controle em 2D e hexaédricas com vinte e sete em 3D.

A adaptação do código originalmente baseado no método dos elementos finitos exigiu modificações substanciais nas etapas de representação geométrica, pré-processamento, avaliação das funções de base, integração numérica, imposição de condições de contorno e pós-processamento. Além disso, foram adotados parâmetros de estabilização específicos para discretizações isogeométricas, conforme recomendações da literatura, visando melhorar o desempenho numérico. A verificação das implementações para escoamentos incompressíveis foi conduzida a partir de exemplos de referência antes de avançar para os desenvolvimentos subsequentes.

Na sequência, tratou-se da introdução de técnica de partição de domínio no programa para escoamentos incompressíveis de modo a capturar efeitos localizados, iniciando-se com a técnica denominada método da combinação de espaços de funções, que já vinha sendo aplicada com sucesso no contexto da fratura elástico-linear com grandes deslocamentos
em trabalhos desenvolvidos concomitantemente a esta tese.
Entretanto, para escoamentos incompressíveis formulados com técnicas de estabilização, verificou-se que a metodologia apresentava bom desempenho para baixos números de Reynolds, mas que instabilidades nos campos de velocidade e pressão surgiam na região de superposição à medida que o número de Reynolds aumentava. Após diversas análises, constatou-se que ajustes na magnitude dos parâmetros de estabilização podiam melhorar a resposta numérica nessa região, mas não garantiam estabilidade numérica de forma geral, descartando-se o emprego dessa técnica no presente trabalho.

Diante disso, adotou-se o método Arlequin estabilizado, na formulação RBSAM proposta por \citeonline{FernandesEtAll:2020}, iniciando-se com a extensão desse método para o contexto tridimensional. A técnica consiste em acoplar os modelos logal e global em uma região da zona de superposição por meio de multiplicadores de Lagrance, seguido da introdução de um termo estabilizante construído a partir do gradiente do resíduo da equação governante ponderado por um parâmetro de estabilização. O parâmetro de estabilização $\tauArlequin$ foi definido com base em \citeonline{TezduyarO:2000, TezduyarS:2003}, de modo que a magnitude dos termos de estabilização fosse compatível com os da equação de acoplamento. A implementação foi inicialmente verificada em domínios fixos e então partiu-se para o desenvolvimento nos casos de contornos locais móveis.

O modelo para movimentação da malha foi desenvolvido durante o estágio de Doutorado Sanduíche \cite{Tononetal:2021}, optando-se pela implementação da técnica de movimentação de malha MJBS  (\textit{Mesh-Jacobian Based Stiffening}), apresentada por \citeonline{TezduyarBSJ:1992c} e \citeonline{TezduyarABJ:1993}, que resolve a movimentação da malha a partir da solução de um problema de elasticidade de Dirichlet fictício. A técnica atribui a cada elemento uma rigidez diferente, que visa preservar os aspectos dos elementos menores, impedindo inversão de elementos ou que elementos assumam volume muito pequeno. A partir dessa implementação, estendeu-se e verificou-se a formulação do método de Arlequin estabilizado para problemas com contornos móveis. Nesse cenário, a malha local é formulada com uma descrição em ALE utilizando-se a técnica MJBS, enquanto que a malha global apresenta descrição Euleriana e permanece fixa ao longo de toda análise.

Após todas as implementações necessárias para a dinâmica dos fluidos, foi então conduzido um estudo de dinâmica dos sólidos computacional, escolhendo-se modelar a estrutura com elementos de casca de Reissner–Mindlin em descrição Lagrangiana total, empregando uma formulação baseada em posições, que não utiliza rotações como parâmetros nodais \cite{CodaP:2007, SanchesC:2013}. Os elementos de casca possuem flexibilidade para representar as estruturas envolvidas em uma ampla gama de problemas de IFE, e a formulação adotada é reconhecidamente robusta para aplicações como as pretendidas neste trabalho. Para discretização espacial utilizou-se elementos triangulares quadráticos de seis nós e a integração temporal implícita foi realizada com o integrador de Newmark, amplamente verificado na literatura.

Por fim, tratou-se do acoplamento fluido–estrutura, adotando-se uma metodologia particionada forte do tipo bloco-iterativo, que garante a modularidade completa entre os códigos computacionais para fluido e para estrutura. O acoplamento é do tipo Dirichlet–Neumann, implementado com projeção entre interfaces não coincidentes, garantindo a transferência consistente de quantidades cinemáticas e dinâmicas entre fluido e estrutura.

O acoplamento fluido-estrutura considerando o particionamento do domínio do fluido — ou seja, a superposição de uma malha local à uma malha global —, consiste numa técnica híbrida Euleriana–ALE, de modo que apenas a malha local, em contato com a estrutura, se movimenta para acomodar as mudanças de configuração da estrutura, enquanto a malha global permanece fixa.

O modelo computacional final para IFE foi testado por meio de exemplos com soluções numéricas de referência confiáveis, incluindo os problemas de cavidade com escoamento oscilatório e base flexível e painel flexível submetido a desprendimento de vórtices, sendo possível demonstrar a precisão e consistência da ferramenta desenvolvida. 

Por fim, foi proposta a análise de uma turbina hidráulica tridimensional, composta por pás diretoras rígidas fixas e um rotor vinculado a um eixo livre para girar, imersos em um tubo de seção circular, sendo submetida a uma diferença de pressão. Essa simulação não é possível de ser conduzida empregando os métodos tradicionais de malhas móveis, a menos que esses sejam associados a técnicas robustas de remalhamento dinâmico. No entanto, a partição de domínios com o método RBSAM, torna possível a simulação, sendo que uma malha global isogeométrica foi adotada para dizcretizar todo o domínio fluido, incluindo as interfaces com as pás diretoras, enquanto o rotor foi inserido por meio de uma malha local em elementos finitos, móvel e adaptada à estrutura. Embora avaliado de forma qualitativa, esse exemplo ilustra o potencial da metodologia proposta para o tratamento de estruturas sujeitas a grandes rotações de corpo rígido. 

De forma geral, os resultados obtidos demonstram a consistência, robustez e potencial de aplicação da metodologia proposta para a análise tridimensional de problemas complexos de interação fluido–estrutura. As principais contribuições originais concentram-se na integração entre a Análise Isogeométrica e o Método dos Elementos Finitos em um arcabouço particionado com malhas superpostas, associada à extensão do método Arlequin estabilizado (RBSAM) para escoamentos incompressíveis com contornos móveis tridimensionais.
A metodologia proposta combina as vantagens dos métodos de malha móvel e dos métodos imersos, garantindo descrição precisa da interface e flexibilidade frente a grandes deslocamentos e rotações, minimizando a necessidade de remalhamentos.
Assim, esta pesquisa representa um avanço teórico e computacional relevante, contribuindo para o desenvolvimento de técnicas numéricas mais eficientes e integradas para a modelagem e simulação de fenômenos complexos de interação fluido–estrutura.

\section{Sugestão para trabalhos futuros}

Como proposta para trabalhos futuros voltados à dinâmica dos fluidos computacional, recomenda-se a incorporação de modelos de turbulência do tipo LES (Large Eddy Simulation) ou RANS (Reynolds-Averaged Navier–Stokes). Essa ampliação é especialmente relevante, considerando que muitos problemas de interação fluido–estrutura de interesse prático envolvem regimes turbulentos, para os quais a modelagem direta (DNS), mesmo com a partição de modelos, pode se tornar inviável do ponto de vista computacional.

No contexto da análise isogeométrica, uma linha de avanço promissora consiste na integração do código desenvolvido com plataformas de modelagem geométrica consolidadas, como o Rhinoceros 3D, por meio da leitura direta de modelos NURBS exportados em formatos padronizados (IGES ou STEP). Essa integração viabilizaria o uso de geometrias complexas e realistas, eliminando a necessidade de geração manual das superfícies e volumes utilizados na discretização isogeométrica.

Em relação ao método Arlequin estabilizado, recomenda-se explorar a sugestão apresentada por \citeonline{Fernandes:2020} em sua tese de doutorado. O autor observa que, embora a inserção do termo de estabilização no operador de acoplamento melhore o condicionamento do sistema algébrico, outros fatores, como o valor da constante $k_A$, também contribuem para o mal condicionamento.  
Nesse sentido, o autor sugere o uso de técnicas desenvolvidas em trabalhos prévios, como os de \citeonline{DhiaER:2008} e \citeonline{SchlittlerC:2017}, no contexto da computação de alto desempenho (HPC -- \textit{High Performance Computing}).  
Em ambos os trabalhos, foi empregado o algoritmo FETI (\textit{Finite Element Tearing and Interconnect}), que se baseia no particionamento do sistema algébrico em subproblemas independentes.  
Assim, métodos iterativos eficientes, como o GMRES (\textit{Generalized Minimum Residual Method}), combinados com pré-condicionadores apropriados, podem ser empregados para a resolução eficiente de problemas de escoamentos incompressíveis.

Ainda no âmbito do método Arlequin estabilizado, propõe-se investigar as diferentes combinações possíveis de discretizações global e local -- por exemplo, utilizando ambas em análise isogeométrica, ambas em elementos finitos, ou ainda uma malha local isogeométrica combinada a uma malha global em elementos finitos. Essa flexibilidade pode revelar vantagens em precisão, estabilidade e custo computacional, dependendo do tipo de problema.

No contexto da dinâmica dos sólidos computacional, sugere-se, como alternativa à técnica de integração temporal atualmente empregada, a implementação do método $\alpha$-generalizado no código computacional desenvolvido, de modo a dissipar seletivamente as componentes de alta frequência, preservando a precisão de segunda ordem no tempo.  
Além disso, a utilização de integradores temporais consistentes entre os domínios de fluido e estrutura pode melhorar a convergência e estabilidade do acoplamento. Também se propõe explorar a discretização isogeométrica para o domínio estrutural, ampliando a coerência entre as formulações.

Por fim, no contexto da interação fluido--estrutura, propõe-se a aplicação da técnica de relaxação de Aitken ao método de particionamento forte do tipo bloco iterativo. Essa técnica numérica pode acelerar a convergência (ou evitar divergência) no processo iterativo de acoplamento entre o fluido e a estrutura.  
Em vez de atualizar diretamente as condições de interface com o valor obtido em cada iteração, realiza-se uma combinação ponderada entre as iterações atual e anterior, com um fator de relaxação adaptativo calculado automaticamente a partir do histórico do erro iterativo.  
A adoção dessa estratégia elimina a necessidade da metodologia \textit{Augmented A22}, que requer a escolha manual de um parâmetro multiplicador associado à parcela da matriz de massa estrutural na matriz tangente global.
