\chapter{Conclusão} 
% ----------------------------------------------------------

O principal objetivo desse estudo foi alcançado, ou seja, desenvolveu-se e implementou-se uma formulação
para análises de problemas tridimensionais de interação fluido-estrutura, que contempla uma técnica de 
partição de domínios, para a consideração de efeitos localizados, a qual permite o uso combinado de aproximações por elementos finitos clássicos
e análise isogeométrica na discretização do problema.

Para isso, conforme pode ser visto no \autoref{capitulo:Cap2} e no \autoref{capitulo:Cap3}, optou-se por uma formulação Arbitrária Euleriana-Lagrangiana (ALE) para a descrição de escoamentos incompressíveis isotérmicos e com contornos móveis tridimensionais. Nesse modelo adotado para o fluido, o domínio e as variáveis de interesse, podem ser discretizados tanto por elementos finitos clássicos, quanto através da análise isogeométrica fazendo-se o uso funções base NURBS. Para tratar questões numéricas recorrentes nesse sistema de equações, como as oscilações espúrias em casos de convecção dominante, típicas da aplicação do método dos resíduos ponderados baseado na formulação clássica de Galerkin, empregou-se a metodologia SUPG. Adicionalmente, a estabilização PSPG é aplicada com o objetivo de contornar a condição imposta pelo critério de \textit{Ladyzhenskaya-Babuška-Brezzi} (LBB). A integração no tempo na formulação é conduzida por meio do método $\alpha$-generalizado. Do ponto de vista computacional, partiu-se de um código baseado em elementos finitos clássicos bidimensionais e expandiu-se-o para que o mesmo contemplasse uma análise tridimensional e permitisse o uso da IGA.  

No que diz respeito a metodologia adotada para a análise das estruturas, conforme foi observado no \autoref{capitulo:Cap4}, adotou-se uma formulação não-linear geométrica dinâmica baseada em uma descrição Lagrangiana Total. A formulação é baseada no método dos elementos finitos com abordagem posicional, onde as variáveis principais são as posições nodais. Além disso, optou-se por trabalhar com elementos de cascas e a integração temporal utilizada é realizada através do método de Newmark. Ressalta-se que no aspecto computacional a formulação já estava totalmente implementada. Entretanto, fez-se necessário um profundo conhecimento das técnicas aplicadas e do código, para que se pudesse posteriormente realizar a integração com o programa da DFC, buscando atingir o objetivo de analisar problemas de interação fluido-estrutura.

Com relação a técnica de partição de domínios para as análises da DFC, realizou-se inicialmente o estudo e implementação da formulação apresentada no \autoref{capitulo:Cap5}, a qual permite utilizar uma malha local mais refinada sobreposta a uma malha global com discretização mais grosseira. A junção entre as discretizações ocorre um uma área de sobreposição, na qual as funções base de cada uma das discretizações são ponderadas e somadas de forma a garantirem a partição da unidade e formarem uma nova base linearmente independente. Embora a técnica apresente características muito promissoras, no âmbito da DFC, com emprego das metologias SUPG e PSPG, um estudo mais aprofundado deve ser ainda realizado para que os parâmetros de estabilização sejam adequadamente calculados na zona de sobreposição. Nesse estudo, observaram-se problemas de convergência para alguns dos problemas analisados.

Nesse contexto, para garantir o desenvolvimento do código com uma técnica de partição de domínios, optou-se pela utilização da formulação estabilizada do método Arlequin (\autoref{capitulo:Cap6}), o qual também leva em conta efeitos localizados através do uso de um modelo local mais refinado superposto a um modelo global com discretização mais grosseira. No método Arlequin, no entanto, o processo de união entre as discretizações, é realizado através do cruzamento e colagem entre os modelos em uma zona de colagem através da utilização de campos de multiplicadores de Lagrange. Para garantir a estabilidade do campo de multiplicadores de Lagrange, e, ao mesmo tempo, fornecer maior flexibilidade a formulação, adiciona-se um termo consistente de estabilização, baseado no resíduo das equações governantes. Do ponto de vista computacional, a utilização do Método Arlequin estabilizado, acarretou na implementação de rotinas adicionais para o reconhecimento dos elementos em zona de colagem e para a obtenção do valor da função ponderadora para os nós (ou pontos de controle) e pontos de integração que compõem as malhas; além de rotinas de cálculo de matrizes e vetores respectivas aos operadores de Lagrange.

Por fim, com base nesses desenvolvimentos, optou-se por um esquema de acoplamento particionado forte entre fluido e estrutura (\autoref{capitulo:Cap7}). Essa abordagem foi escolhida por proporcionar um total desacoplamento entre os \textit{solvers} de fluido e de estrutura, o que facilita a solução dos problemas propostos. Para o acoplamento, utilizou-se a técnica de malhas adaptadas para a malha local do fluido em contato com a estrutura, aplicando-se uma descrição ALE. Vale ressaltar que, embora a malha local possa se mover, a malha global permanece fixa com descrição Euleriana, fazendo com que o método de acoplamento possa ser classificado como uma técnica híbrida. Para a implementação do acoplamento entre os dois meios no aspecto computacional, foram desenvolvidas rotinas de cálculos que propiciassem a troca adequada de informações na interface, conforme detalhou-se no \autoref{capitulo:Cap7}. 

Conforme pode ser observado nos problemas simulados ao longo do texto, e particularmente as simulações de FSI, no \autoref{capitulo:Cap7}, o código computacional proporcionou resultados muito satisfatórios, e apresenta-se como uma ferramenta promissora para análise de problemas IFE com efeitos localizados. A principal vantagem do método está associada à metodologia híbrida de acoplamento entre os meios, que combina as vantagens das abordagens de rastreamento de interface (malhas móveis) e de captura de interface (contornos imersos). Essa característica permite que o fluido nas proximidades da estrutura seja discretizado de forma adequada, assegurando a captura de efeitos localizados. Além disso, por se tratar de uma malha local menor, ela é capaz de suportar maiores deformações e, em caso de necessidade de remalhamento, apenas essa região precisa ser reconstruída.
