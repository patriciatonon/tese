\documentclass[
	12pt,		
	openright,
	twoside,	
	a4paper,			
	english,	
	french,		
	spanish,		
	brazil				
	]{abntex2}

%%%%%%%%%%%%%%%%%%%%%%%%%%%%%%%%%%%%%%%%%%%%%%%%%%%%%%%%%%
% GRUMEC
%%%%%%%%%%%%%%%%%%%%%%%%%%%%%%%%%%%%%%%%%%%%%%%%%%%%%%%%%%
%Lagrangian position (inital configuration) vector
\newcommand{\lPosition}{\mathbf{x}}
\newcommand{\lDomain}{_{\Omega}_x}
\newcommand{\lBoundary}{\Gamma_x}
%Eulerian position (current configuration) vector
\newcommand{\ePosition}{\mathbf{y}}
\newcommand{\eDomain}{\Omega_y}
\newcommand{\eBoundary}{\Gamma_y}
%rigth Cauchy-Green stretch tensor
\newcommand{\cauchyStretch}{\mathbf{C}}
\newcommand{\constitutiveTensor}{\mathbb{C}}
%Green-Lagrange strain tensor
\newcommand{\greenStrain}{\mathbf{E}}
\newcommand{\piolaStress}{\mathbf{S}}
\newcommand{\cauchyStress}{\bm{\sigma}}
%Configuration change function
\newcommand{\fmap}{\mathbf{f}}
\newcommand{\deformation}{\mathbfcal{F}}
\newcommand{\lDeformation}{\mathbfcal{F}_x^h}
\newcommand{\eDeformation}{\mathbfcal{F}_y^h}
\newcommand{\lMedDeformation}{\mathbfcal{F}_x^{mh}}
\newcommand{\eMedDeformation}{\mathbfcal{F}_y^{mh}}
%%Gradient of Configuration change function
\newcommand{\gradDeformation}{\mathbf{A}}
\newcommand{\lGradDeformation}{\mathbf{A}_x^h}
\newcommand{\eGradDeformation}{\mathbf{A}_y^h}
\newcommand{\lMedGradDeformation}{\mathbf{A}_x^{mh}}
\newcommand{\eMedGradDeformation}{\mathbf{A}_y^{mh}}
%%Maping to initial
\newcommand{\initDeformation}{\mathbfcal{F}_0}
%%Gradient of maping to initial
\newcommand{\gradInitDeformation}{\mathbf{A}_0}
\newcommand{\jacobianDeformation}{J}
%strain tensor
\newcommand{\straintensor}{\bm{\varepsilon}}

\newcommand{\totalEnergy}{\Pi}
\newcommand{\extEnergy}{\mathbb{P}}
\newcommand{\kinEnergy}{\mathbb{K}}
\newcommand{\intEnergy}{\mathbb{U}_e}
\newcommand{\concLoad}{\mathbf{F}}
\newcommand{\ebodyLoad}{\mathbf{b}}
\newcommand{\bodyLoad}{\mathbf{b}_0}
\newcommand{\tractionLoad}{\mathbf{p}}
\newcommand{\ltractionLoad}{\mathbf{p}_{0}}
\newcommand{\solidVel}{\dot{\mathbf{y}}}
\newcommand{\solidVelh}{\dot{\mathbf{y}^h}}
\newcommand{\solidAccel}{\ddot{\mathbf{y}}}
\newcommand{\specificEnergy}{u_e}
\newcommand{\bulkModulus}{\kappa}
\newcommand{\shearModulus}{G}
\newcommand{\elasticModulus}{\mathbb{E}}
\newcommand{\lameParameter}{\lambda}
\newcommand{\SVKenergy}{u_e^{SVK}}

\newcommand{\intLoads}{\mathbf{F}^{int}}
\newcommand{\extLoads}{\mathbf{F}^{ext}}
\newcommand{\solidMass}{\mathbf{M}}
\newcommand{\solidDamping}{\mathbf{C}}
\newcommand{\solidHessian}{\mathbf{H}}

\newcommand{\SolidInitPos}{\mathbf{X}}
\newcommand{\SolidPos}{\mathbf{Y}}
\newcommand{\SolidVel}{\dot{\mathbf{Y}}}
\newcommand{\SolidAccel}{\ddot{\mathbf{Y}}}

\newcommand{\lGenVector}{\mathbf{g}_x}
\newcommand{\eGenVector}{\mathbf{g}_y}
\newcommand{\lUnitVector}{\mathbf{e}_x}
\newcommand{\eUnitVector}{\mathbf{e}_y}
\newcommand{\lThickness}{h_x}
\newcommand{\eThickness}{h_y}
\newcommand{\stRate}{a}
\newcommand{\lTheta}{\theta_x}
\newcommand{\eTheta}{\theta_y}
\newcommand{\NNSS}{\mathbf{R}_S}
\newcommand{\lGrad}{\boldsymbol{\nabla}_{\mathbf{x}}}

\newcommand{\poisonsRatio}{\nu}


%%%%%%%%%%%%%%%%%%%%%%%%%%%%%%%%%%%%%%%%%%%%%%%%%%%%%%%%%%
% GRUMEC - TIME INTEGRATION
%%%%%%%%%%%%%%%%%%%%%%%%%%%%%%%%%%%%%%%%%%%%%%%%%%%%%%%%%%
%Alpha_f
\newcommand{\alphaf}{\alpha_f}
%alpha_m
\newcommand{\alpham}{\alpha_m}
%spectral radius
\newcommand{\specRadius}{\rho_\infty}
%time
\newcommand{\tempo}{t}
%time step
\newcommand{\timeStep}{\Delta t}
%initial velocity
\newcommand{\initialVelocity}{\mathbf{u}_s^h}
%Total time
\newcommand{\totalTime}{T}

%%%%%%%%%%%%%%%%%%%%%%%%%%%%%%%%%%%%%%%%%%%%%%%%%%%%%%%%%%
% GRUMEC - ARLEQUIN
%%%%%%%%%%%%%%%%%%%%%%%%%%%%%%%%%%%%%%%%%%%%%%%%%%%%%%%%%%
%Global model
\newcommand{\globalModel}{\Omega_0}
\newcommand{\globalBoundary}{\Gamma_0}
\newcommand{\globalBoundaryN}{\Gamma_{N0}}
%Arlequin model
\newcommand{\arlqModel}{\Omega_i}
\newcommand{\arlqBoundary}{\Gamma_i}

%Local model
\newcommand{\localModel}{\Omega_1}
\newcommand{\localBoundary}{\Gamma_1}
\newcommand{\localBoundaryN}{\Gamma_{N1}}
%Gluing zone
\newcommand{\gluingZone}{\Omega_c}
%Free zone
\newcommand{\freeZone}{\Omega_{fr}}
%Overlapping Zone
\newcommand{\overlappingZone}{\Omega_s}
%Weighting function
\newcommand{\arlequinWF}{\alpha}
\newcommand{\arlequinWFi}{\alpha_i}
%weigthing function local model
\newcommand{\arlequinWFLocal}{\alpha_1}
%weighting function global model
\newcommand{\arlequinWFGlobal}{\alpha_0}

%local velocity
\newcommand{\ulocal}{\mathbf{u}_1}
\newcommand{\ulocalh}{\mathbf{u}_1^h}
%global velocity
\newcommand{\uglobal}{\mathbf{u}_0}
\newcommand{\uglobalh}{\mathbf{u}_0^h}
%velocity Arlequin from model i
\newcommand{\uArlqi}{\mathbf{u}_i^{h}}
%global velocity weight function
\newcommand{\wglobal}{\mathbf{w}_0}
\newcommand{\wglobalh}{\mathbf{w}_0^h}
%local velocity weight function
\newcommand{\wlocal}{\mathbf{w}_1}
%global velocity weight function
\newcommand{\wlocalh}{\mathbf{w}_1^h}
%velocity Arlequin weight function from model i
\newcommand{\wArlqi}{\mathbf{w}_i^h}

%global pressure
\newcommand{\pglobal}{p_0}
\newcommand{\pglobalh}{p_0^h}
%local pressure
\newcommand{\plocal}{p_1}
\newcommand{\plocalh}{p_1^h}
%pressure Alequin from model i
\newcommand{\pArlqi}{p_i^{h}}
%global pressure weight function
\newcommand{\qglobal}{q_0}
\newcommand{\qglobalh}{q_0^h}
%local pressure weight function
\newcommand{\qlocal}{q_1}
\newcommand{\qlocalh}{q_1^h}
%pressure Arlequin weight function from model i
\newcommand{\qArlqi}{q_i^h}

%Surface and body forces in each model
\newcommand{\globalSurfaceLoadh}{\mathbf{h}_0^h}
\newcommand{\localSurfaceLoadh}{\mathbf{h}_1^h}
\newcommand{\arlqSurfaceLoadh}{\mathbf{h}_i^h}
\newcommand{\globalsbodyforceh}{\mathbf{f}_0^h}
\newcommand{\localsbodyforceh}{\mathbf{f}_1^h}
\newcommand{\arlqsbodyforceh}{\mathbf{f}_i^h}

%Lagrange multiplier
\newcommand{\lagrangeMultiplier}{\boldsymbol{\lambda}}
\newcommand{\LagrangeMultiplier}{\boldsymbol{\Lambda}}
\newcommand{\lagrangeMultiplierh}{\boldsymbol{\lambda}^h}
%Lagrange multiplier weight function
\newcommand{\lagrangeMultiplierWF}{\boldsymbol{\zeta}}
\newcommand{\lagrangeMultiplierWFh}{\boldsymbol{\zeta}^h}

%Trial and test spaces
%Solution space global model - velocity
\newcommand{\uGlobalSolution}{\mathcal{S}_{u0}^h}
%Solution space local model - velocity
\newcommand{\uLocalSolution}{\mathcal{S}_{u1}^h}
%Solution space Arlequin model - velocity
\newcommand{\uArlequinSolution}{\mathcal{S}_{ui}^h}
%Solution space global model - velocity
\newcommand{\uGlobalTest}{\mathcal{V}_{u0}^h}
%Solution space global model - velocity
\newcommand{\uLocalTest}{\mathcal{V}_{u1}^h}
%Solution space Arlequin model - velocity
\newcommand{\uArlequinTest}{\mathcal{V}_{ui}^h}

%Solution space global model - pressure
\newcommand{\pGlobalSolution}{\mathcal{S}_{p0}^h}
%Solution space local model - pressure
\newcommand{\pLocalSolution}{\mathcal{S}_{p1}^h}
%Solution space Arlequin model - pressure
\newcommand{\pArlequinSolution}{\mathcal{S}_{pi}^h}
%Solution space global model - pressure
\newcommand{\pGlobalTest}{\mathcal{V}_{p0}^h}
%Solution space global model - pressure´
\newcommand{\pLocalTest}{\mathcal{V}_{p1}^h}
%Solution space Arlequin model - pressure
\newcommand{\pArlequinTest}{\mathcal{V}_{pi}^h}

%Solution space global model - lagrange multiplier
\newcommand{\lagSolution}{\mathcal{M}^h}
%Solution space global model - lagrange multiplier
\newcommand{\lagTest}{\mathcal{Q}^h}

%Arlequin Stabilization Parameter
\newcommand{\tauArlequin}{\tau_{ARLQ}}
\newcommand{\tauArlequini}{\tau_{ARLQi}}
\newcommand{\tauArlequinGlobal}{\tau_{ARLQ0}}
\newcommand{\tauArlequinLocal}{\tau_{ARLQ1}}
%Number of subdomains
\newcommand{\nSubDomains}{n_{dom}}

\newcommand{\resMomGlobal}{\mathbf{r}_{\mathrm{M0}}^{h}}
\newcommand{\resMomI}{\mathbf{r}_{\mathrm{Mi}}^{h}}
\newcommand{\resMomLocal}{\mathbf{r}_{\mathrm{M1}}^{h}}
\newcommand{\resPreGlobal}{r_{\mathrm{C0}}^{h}}
\newcommand{\resPreI}{r_{\mathrm{Ci}}^{h}}
\newcommand{\resPreLocal}{r_{\mathrm{C1}}^{h}}

\newcommand{\NNSL}{\mathrm{R}_\mathrm{L}}
\newcommand{\NNSMLocal}{\mathrm{R}_\mathrm{M1}}
\newcommand{\NNSMArlq}{\mathrm{R}_\mathrm{M,i}}
\newcommand{\NNSCLocal}{\mathrm{R}_\mathrm{C1}}
\newcommand{\NNSMGlobal}{\mathrm{R}_\mathrm{M0}}
\newcommand{\NNSCGlobal}{\mathrm{R}_\mathrm{C0}}
\newcommand{\NNSCArlq}{\mathrm{R}_\mathrm{C,i}}


%%%%%%%%%%%%%%%%%%%%%%%%%%%%%%%%%%%%%%%%%%%%%%%%%%%%%%%%%%
% MESH PROBLEM
%%%%%%%%%%%%%%%%%%%%%%%%%%%%%%%%%%%%%%%%%%%%%%%%%%%%%%%%%%
%Mesh Moving
\newcommand{\domaintt}{\Omega_{\tilde{t}}}
\newcommand{\stressTensorM}{\stressTensor_{M}}
\newcommand{\testfunction}{\mathbf{w}_{M}^h}
\newcommand{\dispM}{\mathbf{z}}
\newcommand{\dispMh}{\bar{\mathbf{z}}^h}
\newcommand{\dispMtth}{\bar{\mathbf{z}}_{\tilde{t}}^h}
\newcommand{\postth}{\posh_{\tilde{t}}}
\newcommand{\tildet}{\tilde{t}}
\newcommand{\poissonm}{\nu_{M}}
\newcommand{\elasticityM}{E_{M}}
\newcommand{\jacM}{J_{M}}
\newcommand{\adimensionalcoordinates}{\boldsymbol{\xi}}
\newcommand{\error}{e_{L2}}





%%%%%%%%%%%%%%%%%%%%%%%%%%%%%%%%%%%%%%%%%%%%%%%%%%%%%%%%%%
% FLUID PROBLEM
%%%%%%%%%%%%%%%%%%%%%%%%%%%%%%%%%%%%%%%%%%%%%%%%%%%%%%%%%%

%mathematical commands
\newcommand{\trace}{\mathrm{tr}}
\newcommand{\unittensor}{\mathbf{I}}
\newcommand{\increment}{\varDelta}
\newcommand{\imaginary}{\imath}
\newcommand{\complexspace}{{\mathbb C}}
\newcommand{\laplacian}{\Delta}
\newcommand{\strainrate}{\boldsymbol{{\dot{\varepsilon}}}}
\newcommand{\strainratetensor}{\dot{\bm{\varepsilon}}}
\newcommand{\vzero}{\mathbf{0}}
\newcommand{\deriv}{\mathrm{d}}
\newcommand{\divergence}{\boldsymbol{\nabla}_y}
\newcommand{\gradient}{\boldsymbol{\nabla}_y}
\newcommand{\realspace}{\mathbb{R}}
\newcommand{\nrealspace}{\mathbb{R}^{\nsd}}


%fluid domain
%domain
\newcommand{\domain}{\Omega}
\newcommand{\domainh}{\Omega^h}
\newcommand{\domainT}{\Omega_t}
%element domain
\newcommand{\domainE}{\Omega^e}
%boundary
\newcommand{\boundary}{\Gamma}
\newcommand{\boundaryh}{\Gamma^h}
%Dirichlet boundary
\newcommand{\boundaryD}{\Gamma_{D}}
%Neumann boundary
\newcommand{\boundaryN}{\Gamma_{N}}
%boundary normal vectors^
\newcommand{\snormal}{\mathbf{n}}
\newcommand{\normal}{\mathbf{n}}
%ALE domains
\newcommand{\domainMat}{\Omega_{0}}
\newcommand{\domainRef}{\Omega_{\bar{x}}}
\newcommand{\domainALEN}{\domain_{t_{n+\alpha_{f}}}}

%fluid and flow properties
%density
\newcommand{\density}{\rho}
%dinamic viscosity
\newcommand{\viscosity}{\mu}
%cinematic viscosity
\newcommand{\kviscosity}{\nu}
%difusivity
\newcommand{\diffusivity}{\nu}
\newcommand{\mdiffusivity}{\kappa}
%Reynolds, Strouhal and Peclet
\newcommand{\Reynolds}{\mathrm{Re}}
\newcommand{\Strouhal}{\mathrm{St}}
\newcommand{\Pe}{\mathrm{Pe}}

%boundary conditions
%Dirichlet velocity
\newcommand{\velocityD}{\mathbf{u}_D}
%Neumann force
\newcommand{\surfaceLoad}{\mathbf{h}}
\newcommand{\surfaceLoadh}{\mathbf{h^h}}

%stress tensor and forces
%Cauchy stress tensor
\newcommand{\stressTensor}{\boldsymbol{\sigma}}
\newcommand{\stress}{\sigma}
%unity mass body forces  
\newcommand{\sbodyforce}{\mathbf{f}}
\newcommand{\sbodyforceh}{\mathbf{f^h}}



%ALE description
%mapping functions
\newcommand{\fmapAI}{\mathbf{f}}
\newcommand{\fmapAR}{\mathbf{\bar{f}}}
\newcommand{\fmapRI}{\mathfrak{f}}
%Jacobian mapping matrix
\newcommand{\FmapAI}{\mathbf{F}}
\newcommand{\FmapAR}{\mathbf{\bar{F}}}
\newcommand{\FmapRI}{\mathfrak{F}}
%physical domain position
\newcommand{\pos}{\mathbf{y}}
\newcommand{\posh}{\mathbf{y}^h}
%reference domain position
\newcommand{\posALE}{\bar{\mathbf{x}}}
\newcommand{\posALEh}{\bar{\mathbf{x}}^h}
%material position
\newcommand{\posMat}{\mathbf{x}}



%trial and test functions/trial and test spaces
%velocity vector
\newcommand{\velocity}{\mathbf{u}}
\newcommand{\velocityh}{\mathbf{u}^h}
%ALE velocity
\newcommand{\velocityALE}{\bar{\velocity}}
\newcommand{\velocityALEh}{\bar{\velocity}^h}
%press
\newcommand{\press}{p}
\newcommand{\pressh}{p^h}
%weighting velocity
\newcommand{\utest}{\mathbf{w}}
\newcommand{\utesth}{\mathbf{w}^h}
%weighting press
\newcommand{\ptest}{q}
\newcommand{\ptesth}{q^h}
%velocity solution space
\newcommand{\usolution}{\mathcal{S}_u}
\newcommand{\usolutionh}{\mathcal{S}_u^h}
%press solution space
\newcommand{\psolution}{\mathcal{S}_p}
\newcommand{\psolutionh}{\mathcal{S}_p^h}
%velocity weighting solution space
\newcommand{\uweighting}{\mathcal{V}_u}
\newcommand{\uweightingh}{\mathcal{V}_u^h}
%pressure weighting solution space
\newcommand{\pweighting}{\mathcal{V}_p}
\newcommand{\pweightingh}{\mathcal{V}_p^h}


%stabilization terms
\newcommand{\SUPG}{\tau_{\mathrm{SUPG}}}
\newcommand{\PSPG}{\tau_{\mathrm{PSPG}}}
\newcommand{\SUGNi}{\tau_{\mathrm{SUGN1}}}
\newcommand{\SUGNii}{\tau_{\mathrm{SUGN2}}}
\newcommand{\SUGNiii}{\tau_{\mathrm{SUGN3}}}
\newcommand{\SUGNxii}{\tau_{\mathrm{SUGN12}}}
\newcommand{\LSIC}{\nu_{\mathrm{LSIC}}}
\newcommand{\RGN}{h_{\mathrm{RGN}}}
\newcommand{\RQD}{h_{\mathrm{RQD}}}
\newcommand{\rRGN}{\mathbf{r}}
%patricia new commands for stabilization parameters
\newcommand{\matrixQ}{\mathbf{Q}}
\newcommand{\matrixQhat}{\hat{\mathbf{Q}}}
\newcommand{\matrixD}{\mathbf{D}}
\newcommand{\matrixCinv}{\mathbf{C}^{-1}}
\newcommand{\matrixC}{\mathbf{C}}
\newcommand{\matrixG}{\mathbf{G}}

%residual vectors and degrees of freedom vectors
%residual vectores weak form
\newcommand{\resMom}{\mathbf{r}_{\mathrm{M}}}
\newcommand{\resPre}{{r}_{\mathrm{C}}}
%semi discrete residual vectores
\newcommand{\NNSM}{\mathbf{R}_\mathrm{M}}
\newcommand{\NNSC}{\mathbf{R}_\mathrm{C}}
%velocity degrees of freedom
\newcommand{\Velocity}{\mathbf{U}}
%press degrees of freedom
\newcommand{\Press}{\mathbf{p}}
%acceleration degrees of freedom
\newcommand{\Acceleration}{\mathbf{\dot{U}}}

%boundary conditions names
\newcommand{\patom}{\press_{\mathrm{atm}}}
\newcommand{\pinfty}{\press_{\infty}}
\newcommand{\stressinfty}{\stresstensor_{\infty}}
\newcommand{\velocinfty}{\velocity_{\infty}}


%indexes
\newcommand{\neb}{n_{\mathrm{eb}}}
\newcommand{\nel}{n_{\mathrm{el}}}
\newcommand{\nnos}{n_{\mathrm{nos}}}
\newcommand{\ncp}{n_{\mathrm{np}}}
\newcommand{\nsd}{{n_{\mathrm{sd}}}}
\newcommand{\npd}{{n_{\mathrm{sd}}}}
\newcommand{\itercounter}{i}
\newcommand{\nen}{n_{\mathrm{en}}}

%element finit context 
%shape function index
\newcommand{\shapef}{N}
%element size
\newcommand{\elementsize}{h}
\newcommand{\integrationPoint}{\tilde{\boldsymbol{\xi}}}
\newcommand{\coordAdimen}{\bm{\xi}}


%FSI domain
\newcommand{\domainFSI}{\Omega_{IFE}}
\newcommand{\domainF}{\domain_{F}}
\newcommand{\domainS}{\domain_{E}}
\newcommand{\boundaryFSI}{\Gamma_{IFE}}
\newcommand{\stressTensorS}{\stressTensor_{E}}
\newcommand{\stressTensorF}{\stressTensor_{F}}
\newcommand{\snormalS}{\snormal_{E}}
\newcommand{\snormalF}{\snormal_{F}}
\newcommand{\Nf}{\mathbf{N}_{1}}
\newcommand{\Ns}{\mathbf{N}_{2}}
\newcommand{\Nm}{\mathbf{N}_{3}}
\newcommand{\Ni}{\mathbf{N}_{i}}
\newcommand{\df}{\mathbf{d}_{1}}
\newcommand{\ds}{\mathbf{d}_{2}}
\newcommand{\dm}{\mathbf{d}_{3}}
\newcommand{\djj}{\mathbf{d}_{j}}
\newcommand{\xf}{\mathbf{x}_{1}}
\newcommand{\xs}{\mathbf{x}_{2}}
\newcommand{\xm}{\mathbf{x}_{3}}
\newcommand{\cff}{\mathbf{c}_{1}}
\newcommand{\css}{\mathbf{c}_{2}}
\newcommand{\cmm}{\mathbf{c}_{3}}
\newcommand{\mA}{\mathbf{A}}

%%%%%%%%%%%%%%%%%%%%%%%%%%%%%%%%%%%%%%%%%%%%%%%%%%%%%%%%%%%%%%%%%%%%%%%%%%%%%
% Isogeometric variables
%%%%%%%%%%%%%%%%%%%%%%%%%%%%%%%%%%%%%%%%%%%%%%%%%%%%%%%%%%%%%%%%%%%%%%%%%%%%
\newcommand{\xsi}{\xi}
\newcommand{\Xsi}{\Xi}
\newcommand{\CP}{\mathbf{B}}
\newcommand{\Nb}{N^b}
\newcommand{\Nbl}{(N^{b}_{i,p})'}
\newcommand{\Nblc}{(N^{b}_{\hat{i},p})'}
\newcommand{\Mb}{M^b}
\newcommand{\Lb}{L^b}
\newcommand{\fNURBS}{R}

% ---
% Pacotes
% ---
\usepackage[T1]{fontenc}
\usepackage[utf8]{inputenc}
\usepackage{lastpage}			
\usepackage{indentfirst}	
\usepackage[table]{xcolor}	
\usepackage{graphicx,multicol}		
\usepackage{microtype} 			
\usepackage{lipsum}		
\usepackage[]{xcolor,soulutf8}
\usepackage{subeqnarray}
\usepackage{array}
\usepackage{amsmath}
\usepackage{amsfonts,eufrak}
\usepackage{amssymb}
\usepackage{amsthm}
\usepackage{xfrac}		
\usepackage{tabularx}
\usepackage[alf,abnt-etal-list=0,abnt-etal-cite=3]{abntex2cite} % Citações padrão ABNT
\usepackage{hhline}
\usepackage{ucs}
\usepackage{multirow}
\usepackage{bm}
\usepackage{subfiles}
\usepackage{morefloats}
\usepackage[caption=false]{subfig}
\usepackage{pdfpages}
\usepackage{mathtools}
\usepackage{gensymb}
\usepackage{nomencl}
\usepackage[]{algpseudocode}
\usepackage{algorithm}
%\usepackage{algorithm2e}
\usepackage{pdfpages}
\usepackage{etoolbox}
\usepackage{mathrsfs}
\usepackage{chngcntr}
\usepackage{hyperref}
\usepackage{nameref}
\usepackage{cancel}
\usepackage{lastpage}
\usepackage{setspace}
\usepackage{bm} 
\usepackage[export]{adjustbox}
\usepackage{parskip}
\usepackage{lmodern}
\usepackage{epstopdf}

\counterwithin{figure}{chapter}
\counterwithin{table}{chapter} 
%\usepackage[hyphenbreaks]{breakurl}

\usepackage{tikz}
\usetikzlibrary{calc,arrows}
\usetikzlibrary{positioning,shapes}
\usetikzlibrary{backgrounds}
\hypersetup{breaklinks=true}

\newtheorem{theorem}{Teorema}[section]
\newtheorem{corollary}{Corolário}[theorem]
\newtheorem{lemma}[theorem]{Lema}
\newtheorem{definition}[theorem]{Definição}

%
\makeatletter
\setlength{\@fptop}{0pt}
\makeatother

\makenomenclature
\renewcommand{\nomname}{Lista de Símbolos}
\renewcommand{\nompreamble}{Nesta seção são apresentados os principais símbolos matemáticos, operadores e variáveis utilizadados neste trabalho. De um modo geral, símbolos em negrito denotam grandezas vetoriais ou tensoriais, enquanto os escritos em estilo de formatação normal ou itálico representam grandezas escalares. Os casos omissos são descritos ao longo do texto.}

\RequirePackage{ifthen}
\renewcommand{\nomgroup}[1]{%
	\ifthenelse{\equal{#1}{B}}{\item[]\textbf{Capítulo \ref{capitulo:Cap2} - \nameref{capitulo:Cap2}}}
}

\makeatletter
\renewcommand{\ALG@name}{Algoritmo}
\makeatother




% Algoritmos com Declaracoes em Português
\algrenewcommand\algorithmicend{\textbf{fim}}
\algrenewcommand\algorithmicdo{\textbf{faça}}
\algrenewcommand\algorithmicrequire{\textbf{Requer}}
\algrenewcommand\algorithmicwhile{\textbf{enquanto}}
\algrenewcommand\algorithmicfor{\textbf{para}}
\algrenewcommand\algorithmicif{\textbf{se}}
\algrenewcommand\algorithmicthen{\textbf{então}}
\algrenewcommand\algorithmicelse{\textbf{senão}}
\algrenewcommand\algorithmicreturn{\textbf{devolve}}
\algrenewcommand\algorithmicfunction{\textbf{função}}
%
%% Rearranja os finais de cada estrutura
\algrenewtext{EndWhile}{\algorithmicend\ \algorithmicwhile}
\algrenewtext{EndFor}{\algorithmicend\ \algorithmicfor}
\algrenewtext{EndIf}{\algorithmicend\ \algorithmicif}
\algrenewtext{EndFunction}{\algorithmicend\ \algorithmicfunction}

% O comando For, a seguir, retorna 'para #1 -- #2 até #3 faça'
%\algnewcommand\algorithmicto{\textbf{até}}
%\algrenewtext{For}[3]%
%{\algorithmicfor\ #1 $\gets$ #2 \algorithmicto\ #3 \algorithmicdo}



\renewcommand{\ABNTEXchapterfont}{\rmfamily\bfseries}
%insere numero do capitulo junto ao numero da figura e tabela


\AtBeginDocument{%
	\setlength\abovedisplayskip{-0.2cm}
	\setlength\belowdisplayskip{0.3cm}} 
%\setlength{\belowdisplayshortskip}{0.0cm}
%\setlength{\abovedisplayshortskip}{-0.4cm}

% altera o tamanho da fonte dos capítulos e seções
\renewcommand{\ABNTEXchapterfontsize}{\HUGE}
\renewcommand{\ABNTEXsectionfontsize}{\LARGE}
\renewcommand{\ABNTEXsubsectionfontsize}{\large} 
% ---
%mantem o alfabeto caligrafico
\DeclareMathAlphabet{\mathcal}{OMS}{cmsy}{m}{n}
\DeclareMathAlphabet{\mathbfcal}{OMS}{cmsy}{b}{n}
% ----------------------------------
\newcolumntype{L}[1]{>{\raggedright\let\newline\\\arraybackslash\hspace{0pt}}m{#1}}
\newcolumntype{C}[1]{>{\centering\let\newline\\\arraybackslash\hspace{0pt}}m{#1}}
\newcolumntype{R}[1]{>{\raggedleft\let\newline\\\arraybackslash\hspace{0pt}}m{#1}}
% ----------------------------------



% configura o uso do hífen na separação silábica
\hyphenpenalty=2000
\tolerance=200


%================================================================================
% Configuração dos Estilo dos Capítulo
%================================================================================

\setboolean{ABNTEXupperchapter}{true}
\makechapterstyle{icmc}{%
	\renewcommand{\chapterheadstart}{} 
	
	% Secao secundaria (Section) Caixa baixa, Negrito
	\renewcommand*{\cftsectionfont}{\bfseries}
	% Secao terciaria (Subsection) Caixa baixa, Negrito, italico
	\renewcommand*{\cftsubsectionfont}{\itshape}
	% Secao quaternaria (Subsubsection) Caixa baixa, italico
	\renewcommand*{\cftsubsubsectionfont}{\itshape}
	% Secao quinquenária (Subsubsubsection) Caixa baixa
	\renewcommand*{\cftparagraphfont}{\normalsize}
	
	%\renewcommand*{\cftpartfont}{\cftchapterfont}
	%\renewcommand*{\cftpartpagefont}{\cftchapterpagefont}
	
	% tamanhos de fontes de chapter e part
	\ifthenelse{\equal{\ABNTEXisarticle}{true}}{%
		\setlength\beforechapskip{\baselineskip}
		\renewcommand{\chaptitlefont}{\ABNTEXsectionfont\ABNTEXsectionfontsize}
	}{%else
		\setlength{\beforechapskip}{0pt}
		\renewcommand{\ABNTEXchapterfontsize}{\HUGE}
		
		%\renewcommand{\ABNTEXchapterfont}{\sffamily\bfseries}  %alteração da fonte dos capítulos, seções e subseções
		\renewcommand{\chaptitlefont}{\ABNTEXchapterfont\bfseries\ABNTEXchapterfontsize}
	}
	
	\renewcommand{\chapnumfont}{\chaptitlefonts}
	\renewcommand{\parttitlefont}{\ABNTEXpartfont\bfseries\Huge\ABNTEXchapterupperifneeded}
	\renewcommand{\partnumfont}{\ABNTEXpartfont\ABNTEXpartfontsize}
	\renewcommand{\partnamefont}{\ABNTEXpartfont\ABNTEXpartfontsize}
	
	% tamanhos de fontes de section, subsection, subsubsection e subsubsubsection
	\setsecheadstyle{\ABNTEXsectionfont\ABNTEXsectionfontsize\bfseries\ABNTEXsectionupperifneeded}
	\setsubsecheadstyle{\ABNTEXsubsectionfont\ABNTEXsubsectionfontsize\bfseries\itshape\ABNTEXsubsectionupperifneeded}
	\setsubsubsecheadstyle{\ABNTEXsubsubsectionfont\ABNTEXsubsubsectionfontsize\itshape\ABNTEXsubsubsectionupperifneeded}
	\setsubsubsubsecheadstyle{\ABNTEXsubsubsubsectionfont\ABNTEXsubsubsubsectionfontsize\ABNTEXsubsubsubsectionupperifneeded}
	
	% impressao do numero do capitulo
	\renewcommand{\chapternamenum}{}
	
	% impressao do nome do capitulo
	\renewcommand{\printchaptername}{%
		%\chaptitlefont
		%\ifthenelse{\boolean{abntex@apendiceousecao}}{\appendixname}{\chaptername}%
	}
	
	% impressao do titulo do capitulo
	\def\printchaptertitle##1{%
		
		\setboolean{ABNTEXupperchapter}{true}
		
		\ifthenelse{\boolean{abntex@innonumchapter}}{
			\vskip 0ex \hrulefill\chaptitlefont\bfseries\ABNTEXchapterupperifneeded{##1}
			\vskip -0.6ex\hfill\rule{\textwidth}{0.5pt} 
			\vskip -2.8ex\hfill\rule{\textwidth}{2pt}
			\vskip 1.5ex
			
		}{%
			% else
			{\hrulefill
				{\renewcommand{\arraystretch}{1.5} %  1 is the default, change whatever you need
					\begin{tabular}{|c|}
						\rowcolor{black}\color{white}\normalsize\ABNTEXchapterfont
						\ifthenelse{\boolean{abntex@apendiceousecao}}{\MakeTextUppercase{\appendixname}}{\MakeTextUppercase{\chaptername}}  \\ 
						\vspace{-1.5ex}\\ %coloquei para aumentar o espaço entre o título e o número
						\resizebox{!}{1.5cm}{\ABNTEXchapterfont\thechapter}
						\\[2.5ex]
						\hline
			\end{tabular}}} \\
			\vskip 4.3ex \flushright\chaptitlefont\bfseries\ABNTEXchapterupperifneeded{##1} \\
			\vskip -0.6ex\hfill\rule{\textwidth}{0.5pt} \\
			\vskip -2.8ex\hfill\rule{\textwidth}{2pt}\\
			\vskip 1.5ex
		}    
		
	}
	
	% impressao do numero do capitulo     	
	\renewcommand{\printchapternum}{%
		\setboolean{abntex@innonumchapter}{false}
	}
	\renewcommand{\afterchapternum}{}
	
	% impressao do capitulo nao numerado
	\renewcommand\printchapternonum{%
		\setboolean{abntex@innonumchapter}{true}%
	}
}
\chapterstyle{icmc}





% ---
% Informações de dados para CAPA e FOLHA DE ROSTO
% ---

\titulo{\textit{Combinação de discretizações isogeométrica e por elementos finitos na análise de interação fluido-estrutura \textcolor{red}{(Colocar a capa oficial)}}}

\autor{PATRICIA TONON}
\local{SÃO CARLOS/SP}
\data{2021}

\orientador{Prof. Dr. Rodolfo André Kuche Sanches}
\instituicao{%
  Universidade de São Paulo - USP
  \par
  Escola de Engenharia de São Carlos - EESC
  \par
  Programa de Pós-Graduação em Engenharia de Estruturas}
\preambulo{Texto apresentado para o exame de qualifica\c{c}\~ao ao doutorado no Programa de Pós-Graduação em Engenharia de Estruturas da Escola de Engenharia de São Carlos da Universidade de São Paulo, como parte dos requisitos para obtenção do título de Doutor em Ciências, Programa: Engenharia Civil (Estruturas). \newline \newline Área de Concentração: Estruturas \newline \newline Orientador: Prof. Dr. Rodolfo André Kuche Sanches}
% ---

\definecolor{black}{RGB}{0,0,0}

% informações do PDF
\makeatletter
\hypersetup{
		pdftitle={\@title}, 
		pdfauthor={\@author},
    	pdfsubject={\imprimirpreambulo},
	    pdfcreator={LaTeX with abnTeX2},
		pdfkeywords={abnt}{latex}{abntex}{abntex2}{trabalho acadêmico}, 
		colorlinks=true,       	
    	linkcolor=black,          	
    	citecolor=black,        	
    	filecolor=black,      	
		urlcolor=black,
		bookmarksdepth=4
}
\makeatother
% --- 
% Espaçamentos entre linhas e parágrafos 
% --- 

% O tamanho do parágrafo é dado por:
\setlength{\parindent}{1.3cm}

% O espaçamento entre linhas é 1 espaço e meio (default)
%

%outras opções incluem:
%\SingleSpacing
%\DoubleSpacing


% Controle do espaçamento entre um parágrafo e outro:
\setlength{\parskip}{0.0cm}  % tente também \onelineskip


% ---
% compila o indice
% ---
\makeindex
% ---





\renewcommand{\imprimircapa}{
	\begin{capa}
		
	\center \large \textbf{UNIVERSIDADE DE SÃO PAULO \\ ESCOLA DE ENGENHARIA DE SÃO CARLOS \\ DEPARTAMENTO DE ENGENHARIA DE ESTRUTURAS}
	\vspace*{3cm}
	
	{\large \imprimirautor}
	\vfill
	\begin{center}
	\ABNTEXchapterfont\bfseries\LARGE\imprimirtitulo
	\end{center}
	\vfill
	\large\imprimirlocal \\
	\large\imprimirdata
	\end{capa}
}



\makeatletter

\renewcommand{\imprimirfolhaderosto}{
	\begin{center}

		{\large\imprimirautor}
		
		\vspace*{\fill}\vspace*{\fill}
		
		\begin{center}
			\ABNTEXchapterfont\bfseries\LARGE\imprimirtitulo
		\end{center}
		
%		\vspace*{\fill}\vspace*{\fill}
		
%		\normalsize{VERSÃO CORRIGIDA}\\
%		\normalsize{A versão original encontra-se na Escola de Engenharia de São Carlos}	
		
		\vspace*{\fill}\vspace*{\fill}
		
		\abntex@ifnotempty{\imprimirpreambulo}{%
			\hspace{.45\textwidth}
			\begin{minipage}{.5\textwidth}
				\SingleSpacing
				\imprimirpreambulo
			\end{minipage}%
			\vspace*{\fill}
		}%
		\vspace*{\fill}		
		
		{\large\imprimirlocal}
		
		\par
		
		{\large\imprimirdata}
	\end{center}
}

\makeatother




% ----
% Início do documento
% ----
\begin{document}
	



% Muda o espaçamento antes e após fórmulas

% Retira espaço extra obsoleto entre as frases.
\frenchspacing 

\allowdisplaybreaks

% ----------------------------------------------------------
% ELEMENTOS PRÉ-TEXTUAIS
% ----------------------------------------------------------

% ---
% Capa
% ---
\imprimircapa
% ---
\imprimirfolhaderosto
%\newpage
% ---
% Inserir a ficha bibliografica e folha de aprovação
% ---
%\includepdf[pages=1]{ficha_catalografica.pdf}
%\newpage
%\includepdf[pages=1]{imagens/folhadeaprovacao.pdf}
%\cleardoublepage



% ---
% Dedicatória
% ---
%\begin{dedicatoria}
%	\vspace*{\fill
%	%	\centering
%	\noindent
%	\begin{flushright}
%% 		\textit{Dedicatória.}
%		\textit{Dedico este trabalho à minha mãe (in memorian).}
%	\end{flushright}
%	\vspace*{\fill}
%\end{dedicatoria}
%% ---
%\cleardoublepage
%% ---
%% Epígrafe
%% ---
%\begin{epigrafe}
%	\vspace*{\fill}
%	\begin{flushright}
%		\textit{``Há pessoas que desejam saber só por saber, e isso é curiosidade; outras, para alcançarem fama, e isso é vaidade; outras, para enriquecerem com a sua ciência, e isso é um negócio torpe; outras, para serem edificadas, e isso é prudência; outras, para edificarem os outros, e isso é caridade.''\\ (Santo Agostinho)}
%	\end{flushright}
%\end{epigrafe}
% ---

% ---
% Agradecimentos
% ---
%\begin{agradecimentos}
%\chapter*{Agradecimentos}
%
%Primeiramente agradeço à Deus, por conduzir meus passos até este momento e ter sido a fonte de esperança nos momentos mais difíceis. 
%
%Aos meus familiares. Em especial agradeço à minha mãe, que infelizmente partiu cedo demais, mas que dedicou seu amor incondicional à mim e me ensinou sobre as coisas mais importantes da vida. Você continua sendo o meu norte e a minha inspiração. 
%
%Ao meu orientador, prof. Dr. Rodolfo André Kuche Sanches, pela orientação exemplar, amizade e confiança em mim depositados durante tantos anos de trabalho em conjunto.
%
%Aos professores do Departamento de Engenharia de Estruturas, com quem tive o prazer de conviver nos últimos anos, por compartilhar seus conhecimentos e experiências.
%
%Aos muitos outros professores que cruzaram o meu caminho durante essa jornada e, que apesar de todas as dificuldades enfrentadas no sistema público de ensino, continuam firmes no objetivo de oferecer educação de qualidade e acreditando no potencial de seus alunos.
%
%Às instituições de ensino pelas quais passei, por oferecer a infraestrutura necessária para que eu chegasse até aqui. Sobretudo, agradeço ao Departamento de Engenharia de Estruturas que foi um divisor de águas na minha carreira.
%
%Agradeço também os incontáveis amigos que fiz durante o período de mestrado e doutorado. Com certeza tudo seria mais difícil sem vocês. Obrigado por todo o tempo, suporte e companheirismo compartilhados.
%
%Ao prof. Dr. Andrea Barbarulo, que supervisionou a minha pesquisa no período de estágio de doutorado sanduíche na França. Muito obrigado por me acolher no MSSMat, pela atenção depositada e ajuda no desenvolvimento deste trabalho. \textit{(Au prof. Andrea Barbarulo, qui a supervisé mes recherches pendant la période de doctorat sandwich en France. Merci beaucoup de m’accueillir au MSSMat, pour l’attention et l’aide dans le développement de ce travail.)}
%
%Aos amigos que fiz durante o período de intercâmbio, com os quais compartilhei tantos aprendizados.
%
%Aos funcionários do SET, que sempre atenderam prontamente à todas as minhas demandas.
%
%Por fim, agradeço ao CNPq pela bolsa de doutorado e à CAPES pela oportunidade e bolsa concedida no período de doutorado sanduíche no exterior.
%
%\cleardoublepage
%\end{agradecimentos}
% ---


\begin{SingleSpace}
\begin{resumo}[Resumo]
	\begin{flushleft}
		TONON, P. \textbf{Combinação de discretizações isogeométrica e por elementos finitos na análise de interação fluido-estrutura.} 2021. \pageref{LastPage} p. Qualifica\c{c}\~ao da Tese (Doutorado em Engenharia de Estruturas) – Departamento de Engenharia de Estruturas, Escola de Engenharia de São Carlos, Universidade de São Paulo, São Carlos, 2021.\newline 
	\end{flushleft}
	\noindent
Este trabalho apresenta o desenvolvimento de uma ferramenta computacional para a análise numérica de problemas de interação fluido-estrutura, em que o domínio do fluido é discretizado por meio da combinação de aproximações baseadas na Análise Isogeométrica e no Método dos Elementos Finitos tradicional. Consideram-se escoamentos incompressíveis, sendo o domínio fluido representado por uma discretização global fixa e não conforme à estrutura, sobre a qual se sobrepõe uma discretização local, mais refinada e adaptada à interface fluido-estrutura.
O escoamento incompressível, tanto no caso da discretização isogeométrica quanto no dos elementos finitos, é solucionado por meio de uma formulação estabilizada, permitindo aproximações de mesma ordem para as variáveis de velocidade e pressão, com integração temporal implícita realizada pelo método de marcha no tempo $\alpha$-generalizado. O acoplamento entre os modelos local e global de fluido é tratado por uma formulação estabilizada do método Arlequin, que consiste em superpor dois modelos — um discretizado por elementos finitos e outro por análise isogeométrica — e compatibilizá-los por meio de um campo de multiplicadores de Lagrange definido sobre uma região denominada zona de colagem. Para garantir a estabilidade do campo de multiplicadores e ampliar a flexibilidade da formulação, adiciona-se uma parcela estabilizadora baseada no resíduo da equação governante. A movimentação do modelo local de fluido, bem como o acoplamento com a estrutura, são viabilizados pela adoção de uma descrição Lagrangiana-Euleriana Arbitrária das equações governantes.
A estrutura é modelada por meio de elementos de casca com cinemática de Reissner-Mindlin, utilizando uma formulação posicional do método dos elementos finitos em descrição Lagrangiana total, adequada à análise dinâmica com grandes deslocamentos. O acoplamento fluido-estrutura é realizado por um esquema particionado forte do tipo bloco-iterativo, que assegura a interação consistente entre os dois meios.
Além de garantir uma discretização com resolução suficiente para capturar efeitos localizados próximos à estrutura, como os associados à camada limite, a metodologia proposta combina as vantagens dos métodos de malhas móveis e de malhas fixas (contornos imersos), permitindo a simulação eficiente de problemas que, no contexto de métodos de malhas móveis tradicionais, exigiriam remalhamento global. A abordagem desenvolvida também proporciona maior flexibilidade na escolha das discretizações e melhora o desempenho computacional em análises tridimensionais complexas de interação fluido-estrutura.
\newline \newline
  	\textbf{Palavras-chave}:\textit{Interação Fluido-Estrutura.  Análise Isogeométrica. Método dos Elementos Finitos. Método Arlequin. Descrição Lagrangiana-Euleriana Arbitrária}  \cleardoublepage
\end{resumo}
\end{SingleSpace}

% resumo em inglês
\begin{SingleSpace}
\begin{resumo}[Abstract]
	\begin{otherlanguage*}{english}		
		\begin{flushleft}
			TONON, P. \textbf{Combination of isogeometric and finite element discretizations for fluid-structure interaction analysis.} 2021. \pageref{LastPage} p. Thesis qualification (Doctorate in Structural Engineering) – Department of Structural Engineering, São Carlos School of Engineering, University of São Paulo, São Carlos, 2021. \newline 
		\end{flushleft}
			\textcolor{red}{review at the end}
		This work presents the development of a computational tool for numerical analysis of fluid-structure interaction problems, in which the fluid domain is discretized by combining approximations based on Isogeometric Analysis (IGA) and the traditional Finite Element Method (FEM). Incompressible flows are considered, with the fluid domain represented by a fixed global discretization, nonconforming with the structure, upon which a refined local discretization, adapted to the fluid–structure interface, is superposed. The incompressible flow, both in the isogeometric and finite element discretizations, is solved through a stabilized formulation that allows equal-order interpolation for velocity and pressure fields, with implicit time integration performed using the $\alpha$-generalized time-marching method. The coupling between the local and global fluid models is handled by a stabilized formulation of the Arlequin method, which consists in superposing two models—one discretized by finite elements and the other by isogeometric analysis—and enforcing compatibility through a field of Lagrange multipliers defined over a region called the gluing zone. To ensure the stability of the multiplier field and enhance the flexibility of the formulation, a stabilization term based on the residual of the governing equations is added. The motion of the local fluid model, as well as the coupling with the structure, is achieved through an Arbitrary Lagrangian–Eulerian (ALE) description of the governing equations. The structure is modeled using Reissner–Mindlin shell elements within a positional finite element formulation under a total Lagrangian description, suitable for dynamic analyses involving large displacements. The fluid–structure coupling is performed using a strong partitioned block-iterative scheme that ensures consistent interaction between the two media. In addition to providing a discretization capable of capturing localized effects near the structure—such as boundary-layer phenomena—the proposed methodology combines the advantages of moving-mesh and fixed-mesh (immersed boundary) methods. This approach enables efficient simulation of problems that, in the context of traditional moving-mesh methods, would require global remeshing, while also offering greater flexibility in the choice of discretizations and improved computational performance in complex three-dimensional FSI analyses.
    	\newline \newline 
    	{\textbf{Keywords}: \textit{Fluid-structure interaction. Isogeometric analysis. Finite Element Method. Arlequin Method. Arbitrary Lagrangian-Eulerian Description}}
	\end{otherlanguage*}
\end{resumo}
\end{SingleSpace}



% ---
% inserir lista de ilustrações
% ---
\renewcommand{\listfigurename}{Lista de Figuras}
\pdfbookmark[0]{\listfigurename}{lof}

\listoffigures*
\cleardoublepage
% ---

% ---
% inserir lista de tabelas
% ---
\pdfbookmark[0]{\listtablename}{lot}
\listoftables*
\cleardoublepage
% ---

% ---
% inserir lista de símbolos
% ---
\pdfbookmark[0]{\nomname}{las}
\printnomenclature

\cleardoublepage
% ---

% ---
% inserir o sumario
% ---
\renewcommand{\baselinestretch}{1.0}%\normalsize
\pdfbookmark[0]{\contentsname}{toc}
\tableofcontents*
\cleardoublepage
\renewcommand{\baselinestretch}{1.3}%\normalsize

% ---
%\DoubleSpacing

% ----------------------------------------------------------
% ELEMENTOS TEXTUAIS
% ----------------------------------------------------------
\textual

\subfile{introducao}
\subfile{Cap2}
\subfile{Cap3}
\subfile{Cap4}
\subfile{Cap5}
\subfile{Cap6}
\subfile{Cap7}
%\subfile{ConclusoesParciais}

% ----------------------------------------------------------
% ELEMENTOS PÓS-TEXTUAIS
% ----------------------------------------------------------
\postextual 

%\bibliographystyle{abntex2-num}
\bibliography{referencias.bib}
%
%\clearpage
%\textcolor{white}{ }

\printindex


% ----------------------------------------------------------
% Apêndices
% ----------------------------------------------------------

% ---
% Inicia os apêndices
% ---
%\begin{apendicesenv}
%	\subfile{apendices}
%\end{apendicesenv}


% ----------------------------------------------------------
% Anexos
% ----------------------------------------------------------

% ---
% Inicia os anexos
% ---
%\begin{anexosenv}
%
%\subfile{anexos}
%
%\end{anexosenv}


\end{document}
