\documentclass[tese_patricia.tex]{subfiles}
\begin{document}

% ----------------------------------------------------------
% Introducao
% ----------------------------------------------------------
\chapter[Introdução]{Introdução}\label{capitulo:introducao}
% ----------------------------------------------------------

A interação fluido-estrutura caracteriza-se por ser uma classe de problemas em que existe uma interdependência nos comportamentos do fluido e da estrutura. O comportamento do fluido depende do formato da estrutura e sua movimentação, assim como, o movimento e a deformação da estrutura dependem das forças que provém do fluido.

A modelagem numérica dos problemas da engenharia estrutural é um ramo vastamente desenvolvido, sendo a análise de estruturas por elementos finitos em softwares comerciais uma prática corrente entre os engenheiros. Entretanto, quando fala-se de interação fluido-estrutura (IFE), esses softwares encontram-se muito longe de atender à demanda dos engenheiros.
Problemas que envolvem a interação entre fluido e estrutura estão presentes em diversas áreas da engenharia, pode-se citar, por exemplo, a ação do vento sobre edifícios, aerodinâmica de modelos automotivos, problemas de \textit{flutter} em estruturas aeronáuticas e de pontes, ou ainda problemas de escoamento de sangue sobre vasos sanguíneos e órgãos, entre muitos outros. A análise experimental de tais problemas, em geral, é muito custosa e envolve muito tempo, desta forma, é de interesse o desenvolvimento de métodos numéricos que representem adequadamente tais análises e que possibilitem que sejam realizadas dentro de um tempo razoável. O crescimento da informática tem auxiliado nesse processo, embora, ainda muitas análises somente sejam possíveis de serem realizadas em grandes \textit{clusters}, e algumas, devido à complexidade dos problemas, não possam ser simuladas sem grandes simplificações.

A análise computacional dos problemas de IFE envolvem basicamente três partes: dinâmica dos fluidos computacional, mecânica dos sólidos computacional e acoplamento entre os meios fluido e sólido. Uma das maiores dificuldades encontrada nessa área diz respeito ao acoplamento entre fluido e sólido visto que para fluidos aplica-se, em geral, uma descrição matemática Euleriana, e para sólidos, Lagrangiana. O processo de acoplamento é realizado basicamente utilizando-se duas possíveis técnicas: métodos de rastreamento de interface (\textit{interface tracking}) ou método de malhas adaptadas e os métodos  de captura de interface (\textit{interface capturing}) ou método de malhas não-adaptadas. 

Nos métodos de rastreamento de interface, a malha do fluido é adaptada à forma da interface sólido-fluido e acompanha seu movimento, requerendo, dessa forma, procedimentos de atualização da malha do fluido ao longo da análise. Nesse tipo de metodologia uma descrição Lagrangiana-Euleriana arbitrária pode ser aplicada ao fluido, permitindo a movimentação do domínio computacional de maneira independente do movimento do fluido. Essa técnica é adequada para problemas em que a estrutura sofre deslocamentos em pequenas escalas comparados à configuração inicial da estrutura, sem que haja mudança topológica do domínio do fluido, visto que grandes distorções do domínio fluido, em geral, acarretam na necessidade de técnicas de remalhamento, que apresentam um custo computacional muito elevado.

No método de captura de interface se utiliza uma malha fixa para o fluido, na qual o sólido se encontra imerso. Um dos aspectos importantes desse método diz respeito à localização do contorno da estrutura dentro da malha do fluido, podendo ser resolvido por exemplo com o uso de uma função \textit{level-set} baseada na distância assinalada ao contorno. Essa técnica, embora possa ser aplicada para grandes deslocamentos, em geral não é adequada para levar em consideração efeitos localizados que precisem de uma maior precisão da malha, como por exemplo, em regiões de camada limite na vizinhança da estrutura. 

Neste trabalho de doutorado, para análise de problemas de IFE tridimensionais, utilizou-se uma técnica para a partição do domínio do fluido, com malhas superpostas, no intuito de unir as vantagens das metodologias de rastreamento de interface e de captura de interface e ao mesmo tempo proporcionar a combinação de diferentes técnicas de discretizações para as malhas. Nesse contexto, duas discretizações espaciais para o fluido são utilizadas, uma malha global maior, menos refinada e fixa no espaço, e uma malha local menor, mais refinada, em contato com a estrutura e que se move para acomodar as deformações da estrutura. Uma das discretizações pode ser isogeométrica, enquanto a outra em elementos finitos tradicional. Como consequência, caso seja necessária a realização de remalhamento, o mesmo pode ser realizado apenas na malha local, diminuindo o custo computacional. 

A proposta inicial da tese de doutorado era a realização do acoplamento entre as malhas através de uma técnica de modificação do espaço das funções base em uma zona de sobreposição, de maneira a preservar à independência linear das funções e a partição da unidade. Tal formulação, se mostrou eficiente para alguns problemas estudados, entretanto, em simulações mais complexas, a metodologia não apresentou o comportamento esperado. Dessa forma, em alinhamento com os objetivos desse trabalho, utilizou-se para o acoplamento o método Arlequin em sua forma estabilizada, conforme será retratado ao longo do texto. 

Neste capítulo são apresentados o estado da arte dos principais assuntos envolvidos no desenvolvimento deste projeto, os objetivos, a metodologia aplicada e a justificativa para esta pesquisa.


\section{Apresentação do texto}

Este texto está dividido em 8 capítulos os quais serão descritos sucintamente na continuação.

No \textit{Capítulo 1} introduz-se e contextualiza-se o tema de pesquisa. Na sequência, no estado da arte, faz-se uma breve apresentação de algumas das formulações mais utilizadas para a solução dos problemas que envolvem a interação fluido-estrutura e métodos de partição de domínios. Por fim, apresentam-se os objetivos, a metodologia e justificava desta pesquisa.

O \textit{Capítulo 2} compreende a descrição da técnica numérica utilizada para a resolução de problemas da dinâmica dos fluidos computacional. 
Apresentam-se inicialmente as equações governantes em sua forma forte em descrição Euleriana, expandindo-as na continuação para uma descrição Euleriana-Lagrangiana arbitrária. Na sequência, a formulação fraca é obtida através da aplicação do método dos resíduos ponderados utilizando a técnica clássica de Galerkin e apresenta-se a discretização espacial da equações. Para contornar as instabilidades típicas que ocorrem quando aplicado o método de Galerkin, e afim de contornar a condição LBB, apresenta-se uma metodologia estabilizada. Para a integração temporal das equações, o método $\alpha$-generalizado aplicado é exposto. Ao final, o algoritmo da implementação computacional é apresentado e alguns exemplos são avaliados para a verificação do programa.

No \textit{Capítulo 3}, apresenta-se a análise isogeométrica aplicada à Dinâmica dos Fluidos Computacional (DFC) por meio da utilização de funções NURBS. O capítulo se inicia com uma breve contextualização do tema, seguida da descrição das funções-base \textit{B-Splines} e de suas principais características, culminando na geração de geometrias a partir dessas funções. Em seguida, introduzem-se as funções NURBS, construídas a partir das \textit{B-Splines}, destacando-se a obtenção de curvas, superfícies e sólidos NURBS. A abordagem isogeométrica é então detalhada, evidenciando a substituição das tradicionais funções polinomiais de Lagrange, utilizadas no Método dos Elementos Finitos clássico, por funções NURBS na discretização das geometrias e variáveis. Além disso, são explicados os parâmetros de estabilização empregados nas equações governantes discretizadas via IGA. Por fim, verifica-se a implementação computacional da DFC com análise isogeométrica por meio de exemplos numéricos.

\textcolor{red}{ALTERAR}
O \textit{Capítulo 4} apresenta uma breve revisão sobre a mecânica dos sólidos voltada ao equilíbrio de corpos deformáveis em descrição Lagrangiana.  Na sequência, apresentam-se os conceitos do método dos elementos finitos posicional e o elemento finito de casca a ser utilizado nesse projeto. Por fim, um exemplo de problema dinâmico é apresentado.

\textcolor{red}{ALTERAR}
No \textit{Capítulo 5} a técnica de decomposição de domínios é apresentada. Descreve-se a obtenção do novo espaço de funções na zona de sobreposição entre malhas global e local, e apresenta-se por fim um exemplo de verificação voltado à dinâmica dos fluidos computacional.

No \textit{Capítulo 6} apresenta-se a técnica de decomposição de domínios através do método Arlequin estabilizado (RBSAM). A primeira parte do capítulo foi dedicada a descrever o método clássico de Arlequin, para na sequência, introduzir a metodologia estabilizada para a solução de escoamentos incompressíveis. Apresenta-se na sucessão do capítulo a extensão da metodologia para problemas de contorno móveis. Ao final, o algoritmo de implementação é apresentado, bem como, exemplos de validação são avaliados.

No \textit{Capítulo 7} discorre-se sobre a formulação utilizada para análise de problemas de Interação Fluido-Estrutura. No texto, apresentam-se as condições de acoplamento necessárias a solução de um problema de IFE, a técnica de movimentação de malha utilizada, e a metodologia de transferência de condições de contorno em uma interface entre fluido e sólido com malhas não coincidentes. Descreve-se na continuação do texto a teoria envolvida no esquema de acoplamento particionado forte adotado. Por fim, o algoritmo de implementação computacional e exemplos de validação são apresentados.

\textcolor{red}{ALTERAR} No \textit{Capítulo 8} são apresentadas conclusões parciais acerca do que foi desenvolvido até momento, bem como considerações sobre o plano de trabalho e resultados esperados ao final do Doutorado.


\section[Estado da Arte]{Estado da Arte}\label{section:estado_da_arte}

Nesta seção apresenta-se uma breve contextualização das formulações mais importantes relacionadas à metodologia aplicada neste projeto para a resolução dos problemas de interação fluido-estrutura. Assim, aborda-se brevemente o estado da arte da mecânica dos fluidos computacional aplicada a problemas de contornos móveis, mecânica dos sólidos computacional aplicada a problemas dinâmicos com grandes deslocamentos, técnicas de acoplamento numérico fluido-estrutura e métodos de decomposição de domínios e multiescala.

\subsection{Dinâmica dos fluidos computacional}
\label{cfdsection}

Na dinâmica dos fluidos computacional (DFC) técnicas numéricas são aplicadas para obtenção de uma solução aproximada para o conjunto de equações que descrevem o comportamento dos fluidos no espaço e no tempo, visto que a solução analítica para esses problemas é conhecida para poucos e simples casos. Os principais tópicos abordados aqui são referentes às diferentes metodologias aplicadas no que  diz respeito a: discretização espacial, métodos de estabilização e modelagem de escoamentos turbulentos. 

No que diz respeito à discretização espacial a DFC desenvolveu-se inicialmente no âmbito do método das diferenças finitas e do método dos volumes finitos (ver, por exemplo, \citeonline{Chung:2002} e \citeonline{Anderson:1995}). O método dos elementos finitos (MEF), por sua vez, popularizou-se inicialmente em análises de estruturas na década de 50, com problemas baseados em princípios variacionais. Alguns anos depois, passou a ser usado também em problemas da DFC, visto que o mesmo apresenta algumas propriedades vantajosas, como por exemplo, a capacidade de discretizar geometrias complexas com o uso de malhas não estruturadas arbitrárias e a facilidade de aplicação de condições de contorno em geometrias complexas e de alta ordem \cite{ZienkiewiczTN:2005,ReddyG:2000}.

Umas das dificuldades encontradas na aplicação do MEF à dinâmica dos fluidos computacional é o fato de que, ao adotar-se o método clássico de Galerkin na discretização espacial das equações que descrevem o comportamento dos fluidos em descrição Euleriana, obtém-se matrizes assimétricas, e, em escoamentos com convecção dominante, surgem variações espúrias nas variáveis transportadas. Esse problema pode ser amenizado à medida que a malha de elementos finitos é refinada, entretanto, é desejável que o método escolhido apresente resultados estáveis mesmo em malhas mais grosseiras.

Para resolver tal dificuldade, algumas técnicas de estabilização foram propostas, a exemplo da metodologia \textit{Stream-Upwind/Petrov-Galerkin} - SUPG \cite{BrooksH:1982}, \textit{Galerkin Least-Squares}-GLS \cite{HughesFH:1989}  e \textit{Sub-Grid Scale}-SGS \cite{Hughes:1995}. Todas essas formulações baseiam-se na introdução de termos estabilizantes ao problema, contendo as variações espúrias que ocorrem em problemas com convecção dominante. Outra possibilidade, diz respeito ao uso do método Taylor-Galerkin (T-G), introduzido por \citeonline{Donea:1984} onde a estabilização é obtida pela introdução de termos de mais alta ordem para a expansão em série de Taylor no processo de discretização temporal.

Uma das metodologias mais difundidas para estabilização dos termos convectivos, é a técnica SUPG, aplicada nesse estudo, que consiste em adicionar à forma fraca da equação da quantidade de movimento, o resíduo da equação da quantidade de movimento ponderado por uma função especialmente escolhida para adicionar difusão na direção das linhas de corrente. Diversos autores contribuíram para consolidação dessa técnica, dentre os quais pode-se citar, \citeonline{HughesT:1984}, \citeonline{Tezduyar:1992d}, \citeonline{CatabrigaC:2002}. O parâmetro adimensional estabilizador cuja função é aplicar uma escala na parcela adicionada, possui sua obtenção discutida em diversos trabalhos, tais como os \citeonline{TakizawaTO:2018} e \citeonline{OtoguroTT:2020}.

Outra dificuldade da DFC diz respeito aos escoamentos incompressíveis. Ao levar-se em conta a incompressibilidade do escoamento, obtém-se a chamada equação da continuidade, onde tem-se apenas o termo do divergente do vetor velocidade. Do ponto de vista computacional, esse aspecto traz problemas na obtenção do campo de pressão. Nesses casos, a utilização da pressão e da velocidade como variáveis primárias aproximadas por funções de forma de mesmo grau pode conduzir instabilidades na resolução do sistema. Essas instabilidades podem ser contornadas utilizando elementos que respeitem a restrição de \textit{Ladyzhenskaya-Babuška-Brezzi} ou LBB, onde a pressão é interpolada por funções de forma de ordem menor, sendo tais elementos conhecidos como Taylor-Hood \cite{BrezziF:1991,ZienkiewiczTN:2005,StrangF:2008}.

Uma metodologia de estabilização semelhante à técnica SUPG foi também desenvolvida para contornar esse problema \cite{HughesFB:1986,TezduyarMRS:1992a}. Essa metodologia é conhecida como PSPG (\textit{Pressure stabilized Petrov-Galerkin}), adotada nesse estudo, e consiste em adicionar à forma fraca da equação da continuidade o resíduo da equação da quantidade de movimento ponderado pelo gradiente da equação da continuidade multiplicado por uma constante estabilizadora. 

Outra consideração importante nas simulações numéricas diz respeito à reprodução de escoamentos turbulentos. As equações de Navier-Stokes descrevem tanto escoamentos laminares como turbulentos, entretanto, a utilização da chamada Simulação Direta de Turbulência leva a custos computacionais elevados, visto que requer uma malha refinada de maneira a representar adequadamente todas as escalas de turbulência. Para contornar esse problema, diferentes técnicas podem ser empregadas, destacando-se os métodos \textit{Reynolds-Averaged Navier-Stokes} (RANS) \cite{Speziale1991,Alfonsi2009} e Simulações de grandes Vórtices (LES) \cite{LaunderS:1972,Germano1991,Wilcox:1993,PIOMELLI1999}.

Os métodos RANS baseiam-se na decomposição das variáveis de fluxo em uma média temporal e uma componente de flutuação. Essa abordagem permite que as equações governantes sejam manipuladas de forma a representar as médias de longo prazo do fluxo, enquanto as flutuações turbulentas são tratadas como termos adicionais, muitas vezes modelados por equações de fechamento. A definição da média pode variar conforme as características do problema. Nas simulações de grandes vórtices o objetivo principal é capturar as estruturas turbulentas de grande escala, que são responsáveis pela maior parte da transferência de momento e energia, e aplicar um modelo para os vórtices de pequena escala. 

O método Variacional Multiescala (VMS) \cite{Hughes:1995,Hughesetal:1998,Hughesetal:2001,BazilevsTT:2013} tem intuito de garantir concomitantemente a estabilização para os efeitos de convecção, para o campo de pressão e para problemas de vorticidade. O método, a partir de princípios variacionais, propõem a representação do problema físico por meio de sua decomposição em grandes e pequenas escalas, resolvendo-as separadamente.  A modelagem do espaço de pequenas escalas é realizado em termos de resíduos das equações de conservação de massa e de conservação da quantidade de movimento. Essa metodologia tem se mostrado adequada para tratamento de problemas de camada limite ou turbulência, os quais apresentam um intervalo de escalas muito amplos.

A Análise Isogeométrica (IGA - \textit{Isogeometric Analysis}) é uma metodologia para análise numérica de problemas descritos por equações diferenciais e foi introduzida primeiramente por \citeonline{HughesCB:2005}. Pode-se dizer que se trata de uma generalização do método dos elementos finitos clássico, a partir do uso de funções base especiais. Na análise isogeométrica, as funções base utilizadas são aquelas aplicadas nos sistemas CAD (\textit{Computed Aided Desing}), ou seja, nas tecnologias aplicadas na engenharia de \textit{design}, animação, artes gráficas e visualização.  Dentro das possibilidades de funções, as mais conhecidas são as funções NURBS(\textit{Non-Uniform Rational B-Splines}) \cite{PiegT:1996}, fazendo que esse seja um ponto de partida para os estudos sobre IGA. Um dos principais objetivos do desenvolvimento dessa ferramenta é a integração entre os sistemas CAD e as técnicas numéricas baseadas em elementos finitos, as quais requerem a geração de malhas baseadas nos dados obtidos em programas CAD. 

Uma das principais vantagens do uso dessa metodologia é representação exata de geometrias mesmo em malhas pouco refinadas, visto que essas funções são capazes de representar exatamente seções cônicas, círculos, cilindros, esferas e elipsoides. Além disso, outra característica matemática que a torna uma boa opção a ser utilizada, é a suavidade das funções NURBS, que são continuas $p-1$ vezes entre os elementos, sendo $p$ o grau da função base. A descrição exata das geometrias é uma característica desejável em problemas que envolvem fenômenos de camada limite, os quais dependem fortemente da precisão geométrica da superfície do corpo imerso no escoamento. Alguns problemas envolvendo escoamentos turbulentos e interação fluido-estrutura, podem ser consultados em: \citeonline{Bazilevsetal:2007,ZhangBGBH:2007,BazilevsCH:2008,BazilevsMCH:2010,BazilevsA:2010}.

Outras metodologias aplicando diretamente funções \textit{B-Splines} também tem se mostrado eficiente para a análise de problemas da dinâmica dos fluidos computacional, como pode ser visto nos trabalhos de \citeonline{HolligRW:2001,BazilevsTT:2013,BazilevsTT:2014}.


\subsection{Dinâmica de estruturas computacional considerando grandes deslocamentos}
\label{csdsection}

A análise de interação fluido estrutura recai muitas vezes em problemas onde é necessário a consideração da não linearidade geométrica da estrutura devido aos grandes deslocamentos ou a efeitos acoplados de membrana e flexão. Dentro desse grupo de problemas pode-se citar \textit{flutter} de grande amplitude, sistemas de desaceleração (paraquedas), aplicações biomédicas, entre outros.

A solução numérica de problemas estruturais é realizada tradicionalmente aplicando-se o método dos elementos finitos. Dentro do contexto da análise não-linear de estruturas utilizando MEF, a formulação corrotacional proposta por \citeonline{Truesdell:1955} é muito popular e descreve a mudança de configuração da estrutura decompondo os movimentos do sólido em rígido e de deformação, e descrevendo-os em termos dos deslocamentos e rotações nodais. Essa formulação, aplicada para pórticos, treliças e cascas, pode ser vista nos trabalhos de \citeonline{HughesL:1981a,HughesL:1981b,Argyris:1982,SimoF:1989,Ibrahimbegovic:2002,BattiniP:2006}.

A formulação corrotacional, ao descrever rotações como parâmetros nodais, apresenta uma limitação para grandes deslocamentos, visto que não se pode aplicar a propriedade comutativa a essa grandeza. Para resolver este problema, utilizam-se formulações linearizadas de Euler-Rodrigues para aproximação das rotações, conforme pode ser visto, por exemplo, em \citeonline{GruttmanSW:2000,CodaP:2010}.

A conservação da energia nessa formulação é um assunto muito controverso em problemas de dinâmica não-linear de cascas e barras. Isso porque no uso da formulação corrotacional, as rotações finitas, que são parâmetros nodais, apresentam objetividade apenas para pequenos incrementos, além disso, a aplicação da formulação resulta em matriz de massa variável, proibindo o uso de algumas processos de integração temporal bem estabelecidos \cite{SanchesC:2013}.

Motivado por \citeonline{Bonet:2000}, \citeonline{Coda:2003} introduz uma formulação baseada em posições, sem rotações como parâmetros nodais. Essa formulação tem sido aplicada com sucesso para problemas de pórticos e cascas \cite{CodaG:2004,CodaP:2010,CarrazedoC:2010,CodaP:2011,SanchesC:2016}, incluindo problemas de interação fluido-estrutura \cite{SanchesC:2013,SanchesC:2014,FernandesCS:2019,AvanciniS:2020}.

Em \citeonline{SanchesC:2013}, os autores utilizam o integrador temporal de Newmark para análise de problemas dinâmicos não-lineares de estruturas de cascas no contexto da IFE com grandes deslocamentos e rotações de corpo rígido. Nesse trabalho, os autores apresentam a prova da conservação da quantidade de movimento linear e angular no uso dessa metodologia, e testam a estabilidade e conservação de energia para problemas com pequenas deformações. 

Baseado no último trabalho citado, neste projeto, aplica-se a formulação Lagrangiana total para elementos de cascas baseada em posições e vetores generalizados, o que evita o uso de aproximações para grandes rotações e permite o uso do integrador de Newmark nos problemas dinâmicos da IFE que apresentam grandes deslocamentos e rotações.

\subsection{Acoplamento fluido-estrutura}
\label{couplingsection}


O problema de interação fluido-estrutura pode ser descrito como um conjunto de equações diferenciais e condições de contornos associadas ao fluido e a estrutura que precisam ser satisfeitas ao mesmo tempo. Como sólidos e fluidos geralmente apresentam descrições matemáticas diferentes, sendo os sólidos tradicionalmente analisados por descrições Lagrangianas e os fluidos por descrições Eulerianas, um dos desafios da análise computacional de IFE é o acoplamento entre os dois meios. A solução de acoplamento a ser aplicada pode ser classificada em dois tipos de metodologias: métodos de rastreamento de interface (\textit{interface tracking}) ou método de malhas adaptadas e métodos de captura de interface (\textit{interface capturing}) ou método de malhas não-adaptadas \cite{Houetal:2012,BazilevsTT:2013b}.

Nos métodos de rastreamento de interface, à medida em que a interface fluido-estrutura move, o domínio espacial do fluido muda seu formato, e a malha do fluido é movimentada para acomodar a mudança da interface. Nesse tipo de metodologia duas possíveis técnicas podem ser aplicadas na modelagem do domínio fluido: a descrição Lagrangiana-Euleriana arbitrária \cite{HughesLZ:1981,DoneaGH:1982,KanchiM:2007} ou a formulação Espaço-Tempo (\textit{Space-Time - ST}) \cite{TezduyarBL:1992b,TezduyarBML:1992c,TakizawaT:2012}, sendo que ambas permitem a movimentação arbitrária da discretização espacial. A principal vantagem do método de malhas adaptadas é a capacidade de controlar a dimensão da malha próxima a interface, bem como a conformidade dos domínios, e como consequência, garantir a precisão dos resultados nessa região.

A técnica empregada para movimentação de malhas é muito importante nesse tipo de problemas, pois deve ser eficiente de maneira a resultar em elementos que possuam uma mínima distorção e alteração de volume, e de forma a evitar que a malha necessite ser reconstruída. Diversas técnicas vêm sido desenvolvidas para essa finalidade e podem ser divididas em três categorias. Na primeira os deslocamentos são impostos na interface entre estrutura e fluido e o campo de deslocamentos é obtido através da resolução de um problema de valor de contorno, formulando-se o problema através de analogia de molas \cite{BottassoDS:2005}, sólido \cite{JohnsonT:1994,SteinTB:2004}, suavização Laplaciana \cite{KanchiM:2007}, entre outras.

O segundo grupo são esquemas ponto-a-ponto, nos quais os deslocamentos da malha são diretamente interpolados a partir dos deslocamentos impostos na interface \cite{DoneaGH:1982,TezduyarABJ:1993,SanchesC:2014}. Existem ainda métodos híbridos, que combinam vantagens de diferentes técnicas de movimentação de malhas \cite{Lefrancois:2008,FernandesCS:2019}. 

Nos métodos rastreamento de interface, no entanto, em alguns casos o remalhamento torna-se inevitável, como em problemas com grandes distorções do domínio ou em problemas com mudanças topológicas, fazendo com que o custo computacional se torne muito elevado.

Por sua vez, os métodos de captura de interface são capazes de lidar com mudanças topológicas e grandes deslocamentos. Para isso, utilizam-se os chamados métodos de contornos imersos, introduzido por \citeonline{Peskin:1972}, nos quais mantém-se a malha do fluido fixa e permite-se que a estrutura mova-se dentro dessa malha. Nesses métodos é necessário que as posições da estrutura sejam identificadas dentro da malha do fluido a cada passo de tempo \cite{MittalI:2005,WangRGF:2011}. Uma das formas de identificação é realizada através de uma função distância assinalada do contorno da estrutura (método \textit{level-set}). Nesse contexto, pode-se citar os trabalhos de  \citeonline{CirakR:2005} aplicados no âmbito dos volumes finitos e de \citeonline{SanchesC:2014} e \citeonline{AkkermanBBFK:2012} em elementos finitos. A principal desvantagem desse tipo de metodologia é que a resolução da discretização na camada limite fica limitada a discretização da malha de elementos finitos onde a interface estiver posicionada no instante de análise.

A resolução dos problemas da IFE pode ser realizada através de duas variações principais: Métodos particionados \cite{RouxG:2009,BazilevsHKWB:2011, SanchesC:2013,SanchesC:2014,FernandesCS:2019} e métodos monolíticos \cite{Blom:1998,Hubneretal:2004,HronM:2007,Avancini:2023}. No primeiro grupo, as equações para fluido e estrutura são resolvidas separadamente, sendo as condições de acoplamento transmitidas de um meio para o outro na interface, em geral, em termos de condições de Dirichlet-Neumann. No segundo grupo, de métodos monolíticos, fluido e estrutura são tratados como entidade única, com um único sistema de equações gerado para fluido e estrutura, sendo as condições de contorno de interface atendidas de maneira implícita durante o processo.

As técnicas de acoplamento particionado do tipo Dirichlet-Neumann caracterizam-se pela aplicação na interface de condições de contorno de Dirichlet no fluido (velocidades provenienentes da movimentação da estrutura) e de Neumann no sólido (forças proveninentes da variação dos campos de pressão e das tensões viscosas no fluido). Essas formulações podem ainda ser classificadas em fracas (explícitas), ou fortes (implícitas). No acoplamento particionado fraco, as equações são resolvidas de uma maneira desacoplada e só no passo de tempo seguinte são aplicadas as condições de contorno na interface. Para o acoplamento particionado forte usa-se de processos iterativos de acoplamento dentro de um passo de tempo. Esse tipo de resolução, aplicada nesse trabalho, também é conhecida como bloco-iterativa \cite{BazilevsTT:2013}, na qual ocorre uma modificação da matriz tangente com relação ao método monolítico, sendo os sistemas do fluido, da estrutura e da malha tratados em blocos separados. Esse tipo de metodologia particionada facilita a solução dos problemas de IFE devido ao total desacoplamento entre os \textit{solvers} de estrutura e de fluido.

Os esquemas particionados podem apresentar, entretanto, algumas desvantagens, como a defasagem que pode ocorrer entre as integrações temporais do fluido e da estrutura quando as condições de contorno na interface entre fluido e estrutura são aplicadas de maneira explícita, e, ainda, instabilidades numéricas como o efeito de massa adicionada \cite{FelippaPF:2001}. Em escoamentos governados pelo campo de pressão, a ação do fluido sobre a estrutura funciona como uma massa adicional, alterando sua inércia \cite{TallecM:2001}. Em escoamentos incompressíveis, nos quais a densidade do sólido e do fluido são muito próximas ou quando a estrutura é muito esbelta esse fenômeno pode ocasionar instabilidades numéricas em técnicas de acoplamento particionado fraco ou dificuldades de convergência no caso da esquema particionado forte. 

Uma das formas de contornar esse problema é a alteração do esquema de acoplamento do tipo Dirichlet-Neumann para condições de contorno de Robin, que consiste em uma combinação linear das condições de Dirichlet e Neumann, ver por exemplo, \citeonline{BadiaNV:2008}. Outra possibilidade é a metodologia introduzida por \citeonline{TezduyarBL:1992b}, chamada de \textit{augmented mass} que consiste em multiplicar a massa da matriz tangente respectiva à estrutura por um fator que dependerá do tipo de problema em análise. Cabe ressaltar ainda, para os casos de acoplamento do tipo bloco-iterativo, o uso da relaxação de Aitken, proposto por \citeonline{IronsT:1969}, e que demonstra-se muito eficiente em trabalhos sequentes \cite{KuttlerW:2008,FernandesCS:2019}.

 
\subsection{Métodos Multiescala e Técnicas de partição de domínios}
\label{arlequinsection}

Em diversas áreas da engenharia se faz necessário levar em consideração efeitos localizados, geralmente de menor escala, em um modelo global. Dentro da análise estrutural pode-se citar problemas de fissuras, orifícios, imperfeições; na mecânica dos fluidos, problemas de camada limite, a interface entre dois fluidos; e na interação fluido-estrutura a interface entre estrutura-fluido, entre outros.

Para uma solução precisa desse tipo de problemas, faz-se necessário a aplicação de técnicas que levem em consideração os efeitos locais, mas ao mesmo tempo não tornem a simulação inviável devido ao seu custo computacional.

O método dos elementos finitos, tradicionalmente aplicado para as análises numéricas de equações diferenciais, foi desenvolvido a partir de um modelo mecânico de meio contínuo, apresentando pouca flexibilidade para a consideração desses efeitos. Os refinamentos \textit{p} e \textit{h} são metodologias eficientes, entretanto, para alguns problemas dinâmicos, demandam técnicas de remalhamento, e podem ser muito caros computacionalmente.

Em busca de aprimorar o Método dos Elementos Finitos (MEF), diversas propostas têm sido apresentadas para aumentar a flexibilidade na resolução de problemas multiescala, como pode-se citar, por exemplo, o caso dos elementos finitos difusos \cite{NayrolesTV:1992} onde o conceito de partículas foi introduzido, resultando em uma generalização do método dos elementos finitos sem a necessidade de malha. Ou ainda,  o método de Galerkin livre de elementos que é uma combinação entre métodos sem malha e o MEF (ver \citeonline{Belytschko:1995}). Com esse mesmo intuito pode-se citar o método de partição da unidade \cite{MelenkB:1996}, o método dos elementos finitos generalizado (G-FEM) \cite{StrouboulisCB:2001} e o método dos elementos finitos estendido (X-FEM) \cite{Moes:2003}, os quais introduzem o enriquecimento à base aproximadora por meio de funções capazes de capturar efeitos localizados. Os métodos G-FEM e X-FEM são, entretanto, fortemente dependentes do conhecimento local da solução, ou de pelo menos, seu aspecto espacial.

Pesquisas como as de \citeonline{FarhatHF:2001} propõem enriquecimentos descontínuos nos espaços funcionais, incorporando modos regulares por meio de formulações discretas de Galerkin e multiplicadores de Lagrange. Além disso, métodos de discretização que não dependem diretamente da interface, fundamentados na técnica de Nitsche, foram desenvolvidos para lidar com problemas envolvendo descontinuidades materiais, como demonstrado no estudo de \citeonline{HansboH:2002}.

Dentro do contexto da mecânica dos fluidos, \citeonline{TezduyarAB:1998,TezduyarA:2000} introduziram a técnica \textit{EDICT} (\textit{Enhanced-Discretization Interface-Capturing Technique}) para captura de interface com aprimoramento da discretização para problemas bifásicos ou com superfície livre. Para isso, nessa região de interface definem-se um subconjunto de elementos (sub-malhas), que posteriormente são refinados sucessivamente, de modo a melhorar a precisão da solução. Como resultado obtém-se uma discretização melhorada para capturar a interface, entretanto, as sub-malhas provenientes, não representam com exatidão descontinuidades na interface. Uma versão mais eficiente dessa técnica foi proposta em \citeonline{TezduyarS:2005}, na qual um método iterativo multinível é projetado para a captura de efeitos do escoamento em pequenas escalas, permitindo a simulação de problemas mais complexos.

No âmbito da DFC pode-se citar ainda o método Variacional Multiescala (VMS) \cite{Hughesetal:1998} que utiliza o conceito de micromodelos e macromodelos, sendo que os micromodelos capturam efeitos em pequenas escalas de maneira a corrigir os macromodelos.

Outro grupo de métodos proposto para flexibilizar o MEF em problemas com efeitos locais são os baseados em superposição de um domínio local a um domínio global. A técnica Chimera definida por \citeonline{BenekSDB:1986} traz a introdução de orifícios na região de superposição dos modelos, definindo um contorno artificial para o modelo global, e a transmissão de dados ocorre através desses contornos artificiais gerados. O método S \cite{Fish:1992} trata o modelo local como um enriquecimento ao global, e a solução é obtida através da soma dos campos de interesse de cada domínio.

O método Arlequin \cite{Dhia:1998,DhiaR:2001}, por sua vez também baseia-se na superposição de modelos de modo a combinar um modelo local mais refinado a um global, no entanto, esse processo é realizado através do cruzamento e colagem entre os modelos em uma zona de superposição e fazendo-se o uso para tal de multiplicadores de Lagrange.  O método Arlequin vem sendo utilizado amplamente em diversas áreas da mecânica dos sólidos (ver, por exemplo, \citeonline{DhiaJ:2010,CaleyronCFP:2013,DhiaT:2011,BaumanDEOP:2008,BiscaniGBHC:2016,JamondD:2013}), na DFC e IFE, entretanto, ainda é pouco explorado. \citeonline{fernier:hal-03991421} aplica a metodolgia para análise de escoamentos compressíveis, e \citeonline{FernandesEtAll:2020} para análise de escoamentos incompressíveis e de IFE para problemas bidimensionais. Nesse trabalho será feita uma extensão do trabalho de  \citeonline{Fernandes:2020} para problemas tridimensionais de IFE e levando em consideração diferentes discretizações matemáticas para as malhas global e local.


\section[Objetivos]{Objetivos}

O principal objetivo deste trabalho é o desenvolvimento e implementação computacional de uma formulação para análise de problemas tridimensionais de interação fluido-estrutura. A formulação deve permitir a consideração de efeitos localizados por meio de uma técnica de partição de domínios, além de viabilizar o uso combinado de aproximações por elementos finitos clássicos e análise isogeométrica.

Para tal finalidade, enumeram-se os seguintes objetivos específicos:

\begin{itemize}
	\item Expansão de um código computacional para análise de escoamentos incompressíveis baseado em método dos elementos finitos tradicional para um código que contemple análise de problemas da DFC tridimensionais e que inclua a possibilidade de discretização através de análise isogeométrica;
	
	\item Estudo de técnicas de partição de domínios para levar em conta efeitos localizados no âmbito da DFC;
	
	\item Implementação da técnica de partição de domínios no código de dinâmica dos fluidos computacional contemplando problemas da DFC com contornos móveis;
	
	\item Estudo aprofundado de um código pré-desenvolvido de análise não-linear geométrica de estruturas de cascas utilizando o MEF posicional;
	
	\item Acoplamento entre os códigos computacionais da DFC e de sólido através do emprego de uma técnica particionada do tipo bloco-iterativa;
	
	\item  Validação dos códigos computacionais através da simulação de problemas da dinâmica dos fluidos, dinâmica das estruturas e problemas IFE.
	
\end{itemize}

% ----------------------------------------------------------
% Metodologia
% ----------------------------------------------------------
\section[Metodologia]{Metodologia} 
% ----------------------------------------------------------

Em função da complexidade envolvida na implementação computacional dos códigos desenvolvidos optou-se pelo uso da linguagem de programação C++ orientada a objetos, visto que esta linguagem já vem sendo utilizada com sucesso no grupo de trabalho da presente estudante de doutorado. Além disso, a programação orientada a objetos proporciona uma maior modularidade dos códigos desenvolvidos e uma maior facilidade para o acoplamento entre módulos distintos.  Todas as implementações são realizadas utilizando bibliotecas, compiladores e softwares livres ou de código aberto, em ambiente Linux.

O projeto de pesquisa iniciou-se pela dinâmica dos fluidos computacional tendo como base os desenvolvimentos realizados em \citeonline{Tonon:2016} e 
um código computacional de dinâmica dos fluidos para análises de escoamentos incompressíveis bidimensionais desenvolvido por \citeonline{Fernandes:2016} e \citeonline{Fernandes:2020} em seus trabalhos de mestrado e doutorado respectivamente. Primeiramente, ampliou-se o código pré-existente de maneira que o mesmo contemplasse análises de problemas tridimensionais. Na sequência, incluiu-se a este código baseado em método dos elementos finitos clássico a possibilidade do uso de análise isogeométrica.

A partir desse ponto, iniciou-se o processo de estudo da metodologia de decomposição de domínios e sua implementação para problemas bidimensionais da DFC foi realizada, conforme a formulação apresentada no Cap. \ref{capitulo:Cap5}. Devido a dificuldades encontradas para simulação de problemas mais complexos, o método Arlequin \cite{Dhia:1998}, em sua versão estabilizada conforme o trabalho de \citeonline{Fernandes:2020}, foi estudado e implementado computacionalmente para problemas bidimensionais e tridimensionais da DFC.

Para a análise dos problemas não-lineares geométricos de estruturas de cascas baseado no MEF posicional, estudaram-se os textos apresentados em \citeonline{SanchesC:2010a,SanchesC:2010b} e \citeonline{Coda:2018}, e empregou-se um código computacional cedido pelo pesquisador Rosicley Júnior Rodrigues Rosa desenvolvido em seu trabalho de mestrado  \cite{Rosa:2021} com linguagem de programação em C++ orientada a objeto e Phyton.

Na sequência deste projeto, realizou-se o acoplamento entre os códigos de fluidos e de estrutura, utilizando-se a metodologia de acoplamento particionado forte através da técnica bloco-iterativa.

Para maior eficiência na resolução dos problemas, os códigos da DFC, de estruturas e de IFE apresentam paralelização em protocolo MPI (\textit{Message passing interface}). O processamento paralelo acontece a partir da divisão do domínio de elementos finitos entre os processos, o qual é realizado através da biblioteca METIS\footnote{Disponível em: \url{http://glaros.dtc.umn.edu/gkhome/metis/metis/overview}}. O METIS proporciona divisão do domínio de elementos finitos em número semelhantes de elementos entre os processos e agrupando-os por proximidade geométrica.

É importante ressaltar que os códigos contam com a interface e implementações do pacote PETSc\footnote{Disponível em: \url{http://https://www.mcs.anl.gov/petsc/}}. Essa biblioteca é desenvolvida em código aberto e possui uma grande quantidade de método iterativos e diretos para solução de sistemas algébricos e também pré-condicionadores. Além do mais, o PETSc possui uma interface bem desenvolvida com outras bibliotecas, como por exemplo, com o METIS citado anteriormente. 

As malhas de elementos finitos utilizadas nas análises são obtidas através do software GMSH\footnote{Disponível em:\url{ https://gmsh.info/}} e a etapa de pós-processamento e visualização é realizada no Kitware Paraview\footnote{Disponível em:\url{ http://https://www.paraview.org/}} e  Gnuplot\footnote{Disponível em:\url{ https://gnuplot.info/}}. Para problemas aplicando a análise isogeométrica, a etapa de pré-processamento é realizada com um código desenvolvido pela autora e seu orientador durante seu trabalho de mestrado \cite{Tonon:2016}.

No que diz respeito à infraestrutura, utiliza-se o \textit{cluster} disponível no Laboratório de Informática e de Mecânica Computacional (LIMC) do SET para a simulação de problemas mais complexos, e um computador pessoal para a simulação de problemas mais simples.

%\vspace{-0.8cm}

 \section[Justificativa]{Justificativa}

Os problemas de interação fluido-estrutura estão presentes em todas as partes, na engenharia, nas ciências, na medicina e também no dia-a-dia das pessoas.
O projeto de estruturas cada vez mais esbeltas, a necessidade de obtenção de energia elétrica a partir de fontes de energia limpa, como as usinas eólicas, o estudo de \textit{airbags}, o bombeamento do sangue pelos ventrículos do coração humano e o abrir e fechar das válvulas do coração, são apenas alguns dos exemplos que demonstram a necessidade de se aprofundar nos estudos da interação fluido-estrutura computacional.

Enquanto que no campo engenharia estrutural os pacotes comerciais baseados em MEF estão em constante evolução, e podem resolver uma grande gama de problemas, os softwares que tratam de problemas da dinâmica dos fluidos computacional e de problemas multifísicos, como os problemas da IFE, ainda precisam evoluir muito para suprirem a demanda dos pesquisadores. A simulação numérica de problemas reais de IFE é ainda muito difícil de ser realizada em função do elevado custo computacional, e muitas vezes, devido a grande complexidade dos problemas, ainda é impossível simulá-los sem que sejam realizadas grandes simplificações. Dessa forma, os ensaios experimentais, ainda são em grande parte das vezes, a melhor forma de se estudar o comportamento de IFE, embora, sejam muito custosos e demorados.

Dentro desse contexto, muitos pesquisadores tem se esforçado para que a análise de problemas da IFE computacionalmente seja possível e eficiente. Com essa mesma proposta, nesse projeto pretende-se desenvolver uma ferramenta computacional eficiente para análise tridimensional de problemas de interação fluido-estrutura utilizando uma combinação entre método dos elementos finitos e análise isogeométrica.  Nesse trabalho, será aplicado o método Arlequin para a superposição de malhas no modelo do fluido, com uma malha local mais refinada e deformável em contato com a superfície da estrutura sobreposta a uma malha global fixa e com discretização mais grosseira. Dessa forma, ainda que a estrutura mude drasticamente, não se faz necessário o remalhamento de toda a malha que compõe o fluido, diminuindo assim o custo computacional. Esta proposta compartilha as vantagens dos métodos de malhas adaptadas e de malhas não adaptadas, possuindo a possibilidade de alcançar uma ótima convergência.

%
%\clearpage
%
%\textcolor{white}{ }

\end{document}