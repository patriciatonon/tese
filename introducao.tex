\chapter[Introdução]{Introdução}\label{capitulo:introducao}

A interação fluido-estrutura caracteriza-se por ser uma classe de problemas em que existe uma interdependência entre comportamentos do fluido e da estrutura. O comportamento do fluido depende da forma e da movimentação da estrutura, assim como, o movimento e a deformação da estrutura dependem das forças provenientes do fluido.

A modelagem numérica dos problemas da engenharia estrutural é um ramo vastamente desenvolvido, sendo a análise de estruturas por elementos finitos em softwares comerciais uma prática corrente entre os engenheiros. Entretanto, quando se trata de interação fluido-estrutura (IFE), esses softwares estão longe de atender à demanda dos engenheiros.

Problemas que envolvem a interação entre fluido e estrutura estão presentes em diversas áreas, podendo-se citar como exemplos a ação do vento sobre edifícios, aerodinâmica de modelos automotivos, problemas de \textit{flutter} em estruturas aeronáuticas e de pontes, ou ainda problemas de escoamento de sangue em vasos sanguíneos e órgãos, entre muitos outros. A análise experimental de tais problemas, em geral, é muito custosa e demanda bastante tempo e equipamentos complexos. Dessa forma, é de interesse o desenvolvimento de métodos numéricos que permitam simular adequadamente tais problemas dentro de um tempo razoável. O crescimento da informática tem auxiliado nesse processo, contudo, muitas análises ainda só podem ser realizadas em grandes \textit{clusters} e, em alguns casos, devido à complexidade dos problemas, não podem ser simuladas sem grandes simplificações.

A análise computacional dos problemas de IFE possui basicamente três componentes: a dinâmica dos fluidos computacional, a mecânica dos sólidos computacional e o acoplamento entre os meios fluido e sólido. Uma das maiores dificuldades encontradas nessa área, diz respeito à compatibilização das formulações da mecânica dos fluidos e dos sólidos, visto que, em geral, para fluidos aplica-se uma descrição matemática Euleriana, e para sólidos, Lagrangiana. Dessa forma, existem duas formas comuns de se realizar o acoplamento fluido-estrutura, que são os métodos de malhas conformes, ou de malhas móveis, e os métodos de malhas não-conformes, ou de malhas fixas. 

Nos métodos de malhas conformes, a malha do fluido é conforme ao domínio computacional do sólido e acompanha seu movimento, requerendo, assim, procedimentos de atualização (deformação ou deformação associada à reconstrução) dessa malha ao longo da análise. Nesse tipo de metodologia, uma descrição Lagrangiana-Euleriana arbitrária (ALE) pode ser aplicada ao fluido, permitindo a movimentação do domínio computacional de maneira independente do movimento das partículas de fluido. Essa técnica é adequada para problemas em que a estrutura sofre deslocamentos em pequenas escalas em relação à configuração inicial da estrutura, sem que haja mudança topológica do domínio do fluido, visto que grandes distorções do domínio fluido, em geral, acarretam na necessidade de técnicas especiais de remalhamento, que apresentam um custo computacional elevado.

Nos métodos de malhas não-conformes, utiliza-se uma malha fixa para o fluido, na qual o sólido se encontra imerso, sendo adotadas técnicas de contorno imerso para a imposição das condições de acoplamento. Um dos aspectos importantes desse método diz respeito à localização do contorno da estrutura dentro da malha do fluido, o que pode ser resolvido, por exemplo, com o uso de uma função \textit{level-set} baseada na distância assinalada ao contorno do sólido. Essa técnica pode ser aplicada a qualquer escala de deslocamentos, inclusive em problemas com mudanças topológicas no domínio do fluido, entretanto, não é eficiente para levar em consideração efeitos localizados que exigem maior resolução da malha, como, por exemplo, em regiões de camada limite na vizinhança da estrutura.

Neste trabalho, busca-se, no contexto da análise tridimensional de interação fluido-estrutura, empregar técnicas de partição de domínio com malhas superpostas, de modo a unir as vantagens das metodologias de malhas móveis e de malhas fixas, e, ao mesmo tempo, possibilitar a combinação de diferentes técnicas de discretização (por elementos finitos e isogeométrica). Para isso, duas discretizações espaciais para o domínio do fluido são superpostas: uma malha global, maior, menos refinada, fixa no espaço e não conforme à estrutura; e uma malha local, menor, mais refinada e conforme à estrutura, que se move para acomodar suas deformações. Uma das discretizações pode ser isogeométrica, enquanto a outra pode ser baseada em elementos finitos. Como consequência, caso seja necessário realizar o remalhamento, ele pode ser feito apenas na malha local, reduzindo o custo computacional. 

Foram consideradas duas técnicas para partição de domínio. A primeira, denominada método da combinação dos espaços de função, conforme proposto por \citeonline{Rosa:2022} consiste em ponderar as funções de forma da discretização local e da discretização global por funções de combinação e unir os espaços local e global, gerando uma nova base. A segunda consiste no método Arlequin proposto por \citeonline{Dhia:1998}, em sua versão estabilizada com base no trabalho de \citeonline{FernandesEtAll:2020}, que consiste em acoplar os modelos local e global por meio de multiplicadores de Lagrange definidos sobre a zona de superposição. A primeira técnica, embora tenha se demonstrado bastante robusta em outros problemas, mostrou-se inadequada para a formulação estabilizada para escoamentos incompressíveis a número de Reynolds elevados, enquanto o método Arlequin, embora mais custoso, mostra-se adequado para qualquer número de Reynolds.

\section{Apresentação do texto}

Este texto está dividido em 8 capítulos os quais serão descritos sucintamente na continuação.

No \textit{Capítulo 1} introduz-se e contextualiza-se o tema de pesquisa. Na sequência, no estado da arte, faz-se uma breve apresentação de algumas das formulações mais utilizadas para a solução dos problemas que envolvem a interação fluido-estrutura e métodos de partição de domínios. Por fim, apresentam-se os objetivos, a metodologia e justificava desta pesquisa.

O \textit{Capítulo 2} compreende a descrição da técnica numérica utilizada para a resolução de problemas da dinâmica dos fluidos computacional. 
Apresentam-se inicialmente as equações governantes em sua forma forte em descrição Euleriana, expandindo-as na continuação para uma descrição Euleriana-Lagrangiana arbitrária. Na sequência, a formulação fraca é obtida através da aplicação do método dos resíduos ponderados utilizando a técnica clássica de Galerkin e apresenta-se a discretização espacial da equações. Para contornar as instabilidades típicas que ocorrem quando aplicado o método de Galerkin, e afim de contornar a condição LBB, apresenta-se uma metodologia estabilizada. Para a integração temporal das equações, o método $\alpha$-generalizado aplicado é exposto. Ao final, o algoritmo da implementação computacional é apresentado e alguns exemplos são avaliados para a verificação do programa.

No \textit{Capítulo 3}, apresenta-se a análise isogeométrica aplicada à Dinâmica dos Fluidos Computacional (DFC) por meio da utilização de funções NURBS. O capítulo se inicia com uma breve contextualização do tema, seguida da descrição das funções-base \textit{B-Splines} e de suas principais características, culminando na geração de geometrias a partir dessas funções. Em seguida, introduzem-se as funções NURBS, construídas a partir das \textit{B-Splines}, destacando-se a obtenção de curvas, superfícies e sólidos NURBS. A abordagem isogeométrica é então detalhada, evidenciando a substituição das tradicionais funções polinomiais de Lagrange, utilizadas no Método dos Elementos Finitos clássico, por funções NURBS na discretização das geometrias e variáveis. Além disso, são explicados os parâmetros de estabilização empregados nas equações governantes discretizadas via AIG. Por fim, verifica-se a implementação computacional da DFC com análise isogeométrica por meio de exemplos numéricos.

O \textit{Capítulo 4} apresenta uma breve revisão sobre a mecânica dos sólidos voltada a cinemática e ao equilíbrio de corpos deformáveis em descrição Lagrangiana, assim como elenca o princípio da estacionariedade de energia e a apresenta o modelo constitutivo de Saint-Venant-Kirchhoff adotado nesse trabalho. Na sequência, apresentam-se os conceitos do método dos elementos finitos posicional e o elemento finito de casca a ser utilizado nesse projeto para análise não-linear dinâmica de sólidos. Por fim, o algoritmo da implementação computacional é exibido e um problema de casca cilíndrica com \textit{snap through} dinâmico é simulado.

No \textit{Capítulo 5} a técnica de partição de domínios é apresentada.  Descreve-se inicialmente a combinação proposta para os espaços de funções respectivos as malhas local e global com intuito de obter-se um novo espaço de funções independentes na zona de sobreposição. Na sequência, descreve-se a metodologia para o cálculo da função ponderadora de combinação dentro do domínio.
O roteiro de implementação computacional é então exibido, e apresenta-se, um exemplo de verificação voltado à dinâmica dos fluidos computacional.

No \textit{Capítulo 6} apresenta-se a técnica de decomposição de domínios através do método Arlequin estabilizado (RBSAM). A primeira parte do capítulo foi dedicada a descrever o método clássico de Arlequin, para na sequência, introduzir a metodologia estabilizada para a solução de escoamentos incompressíveis. Apresenta-se na sucessão do capítulo a extensão da metodologia para problemas de contorno móveis. Ao final, o algoritmo de implementação é apresentado, bem como, exemplos de validação são avaliados.

No \textit{Capítulo 7} discorre-se sobre a formulação utilizada para análise de problemas de Interação Fluido-Estrutura. No texto, apresentam-se as condições de acoplamento necessárias a solução de um problema de IFE, a técnica de movimentação de malha utilizada, e a metodologia de transferência de condições de contorno em uma interface entre fluido e sólido com malhas não coincidentes. Descreve-se na continuação do texto a teoria envolvida no esquema de acoplamento particionado forte adotado. Por fim, o algoritmo de implementação computacional e exemplos de validação são apresentados.

No \textit{Capítulo 8} são apresentadas as considerações finais sobre o trabalho desenvolvido. 

\section[Estado da Arte]{Estado da arte}\label{section:estado_da_arte}

Nesta seção apresenta-se uma breve contextualização dos principais assuntos relacionados a este trabalho. Assim, aborda-se brevemente o estado da arte da mecânica dos fluidos computacional aplicada a escoamentos com contornos móveis, a análise isogeométrica aplicada a problemas da dinâmica dos fluidos computacional, a mecânica dos sólidos computacional aplicada a problemas dinâmicos com grandes deslocamentos com o foco em elementos de cascas, às técnicas numéricas para acoplamento fluido-estrutura e os métodos de decomposição de domínios e multiescala. 

\subsection{Dinâmica dos fluidos computacional}
\label{cfd}

A dinâmica dos fluidos computacional (DFC) trata da obtenção de soluções numéricas para as  equações diferenciais que descrevem o comportamento dos fluidos no espaço e no tempo, tendo em vista que a solução analítica para esses problemas é conhecida apenas em raros casos e sob hipóteses simplificadoras. Os principais tópicos abordados aqui referem-se às diferentes metodologias aplicadas à discretização espacial, às fontes de instabilidade numérica e aos métodos de estabilização.

No que diz respeito à discretização espacial, a DFC desenvolveu-se inicialmente no âmbito do método das diferenças finitas e do método dos volumes finitos (ver, por exemplo, \citeonline{Anderson:1995} e \citeonline{Chung:2002}). O método dos elementos finitos (MEF), por sua vez, popularizou-se inicialmente em análises de estruturas na década de 1950, com formulações baseadas em princípios variacionais. Alguns anos depois, passou a ser utilizado também em problemas da DFC, visto que apresenta propriedades vantajosas, como, por exemplo, a facilidade de discretizar geometrias complexas com o uso de malhas não estruturadas arbitrárias e a facilidade de aplicar condições de contorno em geometrias complexas e de alta ordem \cite{ReddyG:2000,ZienkiewiczTN:2005a}.

Uma das dificuldades encontradas na aplicação do MEF à dinâmica dos fluidos computacional é o fato de que, ao se adotar o método clássico de Galerkin na discretização espacial das equações governantes em descrição Euleriana, obtêm-se matrizes assimétricas e, em escoamentos com convecção dominante, surgem variações espúrias nas variáveis transportadas \cite{BrooksH:1982,ZienkiewiczTN:2005a}. Esse problema pode ser amenizado à medida que a malha de elementos finitos é refinada; entretanto, é desejável que o método escolhido apresente resultados estáveis mesmo em malhas mais grosseiras.

Para resolver tal dificuldade, algumas técnicas de estabilização foram propostas, a exemplo dos métodos \textit{Stream-Upwind/Petrov-Galerkin} (SUPG) \cite{BrooksH:1982}, \textit{Galerkin Least-Squares} (GLS) \cite{HughesFH:1989} e \textit{Sub-Grid Scale} (SGS) \cite{Hughes:1995}. Todas essas formulações baseiam-se na introdução de termos estabilizantes ao problema, de modo a conter as variações espúrias em casos de convecção dominante. Outra possibilidade diz respeito ao uso do método Taylor-Galerkin (T-G), introduzido por \citeonline{Donea:1984}, no qual a estabilização é obtida pela inclusão de termos de ordem superior que exercem efeito estabilizante, ao se empregar a expansão em série de Taylor no processo de discretização temporal.

Uma das metodologias mais difundidas para a estabilização dos termos convectivos é a técnica SUPG, a qual é aplicada neste estudo. Essa técnica consiste em adicionar, à forma fraca da equação da quantidade de movimento, o resíduo dessa equação ponderado por uma função especialmente escolhida para introduzir estabilização na direção das linhas de corrente, resultando em uma formulação consistente. Diversos autores contribuíram para a consolidação dessa técnica, dentre os quais podem ser citados \citeonline{CatabrigaC:2002}, \citeonline{HughesT:1984} e \citeonline{Tezduyar:1992}. O parâmetro adimensional estabilizador, cuja função é aplicar uma escala na parcela adicionada, tem sua obtenção discutida em diversos trabalhos, tais como \citeonline{OtoguroTT:2020} e \citeonline{TakizawaTO:2018}.

Outra dificuldade da aplicação do MEF à mecânica dos fluidos diz respeito aos escoamentos incompressíveis. Ao levar-se em conta a incompressibilidade do escoamento, obtém-se a chamada equação da continuidade, na qual aparece apenas o termo do divergente do vetor velocidade. Do ponto de vista numérico, a pressão atua como um multiplicador de Lagrange, impondo a condição de divergente da velocidade nulo. Nesse caso, para que o sistema tenha solução única e resulte em uma formulação estável, é necessário observar as restrições de \textit{Ladyzhenskaya-Babuška-Brezzi} (LBB) na escolha dos espaços de funções para a aproximação da pressão e da velocidade, não sendo possível interpolar essas variáveis por polinômios de mesma ordem \cite{BrezziF:1991,StrangF:2008,ZienkiewiczTN:2005b}. Dessa forma, foram desenvolvidos diversos elementos, denominados Taylor-Hood, que atendem a essas restrições \cite{DoneaH:2003}.


De modo a permitir o uso do mesmo espaço de funções para pressão e velocidade, aumentando assim a flexibilidade do método, surgiram técnicas de estabilização do campo de pressão. Uma metodologia de estabilização para problemas incompressíveis, semelhante à técnica SUPG, foi apresentada por \citeonline{HughesFB:1986} para escoamentos de Stokes, posteriormente aplicada ao problema de Navier-Stokes e denominada PSPG (\textit{Pressure Stabilized Petrov-Galerkin}) por \citeonline{Tezduyar:1992}. Essa metodologia é adotada neste estudo e consiste em adicionar, à forma fraca da equação da continuidade, o resíduo da equação da quantidade de movimento ponderado pelo gradiente da função teste da equação da continuidade, multiplicado por um parâmetro de estabilização.

Outra consideração importante nas simulações numéricas diz respeito à reprodução de escoamentos turbulentos. As equações de Navier-Stokes descrevem tanto escoamentos laminares como turbulentos, entretanto, a utilização da chamada Simulação Direta de Turbulência leva a custos computacionais elevados, visto que requer uma malha refinada de maneira a representar adequadamente todas as escalas de turbulência. Para contornar esse problema, diferentes técnicas podem ser empregadas, destacando-se os métodos \textit{Reynolds-Averaged Navier-Stokes} (RANS) \cite{Alfonsi2009,Speziale1991} e Simulações de grandes Vórtices (\textit{Large Eddy Simulation} - LES) \cite{Germano1991,LaunderS:1972,PIOMELLI1999,Wilcox:1993}.

Os métodos RANS baseiam-se na decomposição das variáveis de fluxo em uma média temporal e em uma componente de flutuação. Essa abordagem permite que as equações governantes sejam manipuladas de forma a representar as médias de longo prazo do escoamento, enquanto as flutuações turbulentas são tratadas como termos adicionais, muitas vezes modelados por equações de fechamento. A definição da média pode variar conforme as características do problema. Já nas simulações LES, o objetivo principal é capturar as estruturas turbulentas de grande escala, responsáveis pela maior parte da transferência de quantidade de movimento e energia, e aplicar um modelo para os vórtices de pequena escala.

O método Variacional Multiescala (VMS) \cite{BazilevsTT:2013a,Hughes:1995,Hughesetal:1998,Hughesetal:2001} permite lidar simultaneamente com os efeitos da convecção dominante, a instabilidade associada ao campo de pressão em problemas incompressíveis e a representação adequada de estruturas relacionadas à vorticidade. O método, a partir de princípios variacionais, propõem a representação do problema físico por meio de sua decomposição em escalas grandes (resolvidas) e pequenas (não resolvidas), tratando-as separadamente.  A modelagem do espaço de pequenas escalas é realizado em termos de resíduos das equações de conservação de massa e de conservação da quantidade de movimento. 


\subsection{Análise isogeométrica}
\label{AIGsection}

A Análise Isogeométrica (AIG) é uma metodologia para análise numérica de problemas descritos por equações diferenciais e foi introduzida primeiramente por \citeonline{HughesCB:2005}. Pode-se dizer que se trata de uma generalização do método dos elementos finitos clássico, a partir do uso de funções base especiais. Na análise isogeométrica, as funções base utilizadas são aquelas aplicadas nos sistemas CAD (\textit{Computed Aided Desing}), ou seja, nas tecnologias aplicadas na engenharia de \textit{design}, animação, artes gráficas e visualização.  Dentro das possibilidades de funções, as mais conhecidas são as funções NURBS (\textit{Non-Uniform Rational B-Splines}) \cite{PiegT:1996}, fazendo que esse seja um ponto de partida para os estudos sobre AIG. Um dos principais objetivos do desenvolvimento dessa ferramenta é a integração entre os sistemas CAD e as técnicas numéricas baseadas em elementos finitos, as quais requerem a geração de malhas baseadas nos dados obtidos em programas CAD. 
		
Uma das principais vantagens do uso dessa metodologia é representação exata de geometrias mesmo em malhas pouco refinadas, visto que essas funções são capazes de representar exatamente seções cônicas, círculos, cilindros, esferas e elipsoides. Além disso, outra característica matemática que a torna uma boa opção a ser utilizada, é a suavidade das funções NURBS, que são continuas $p-1$ vezes entre os elementos, sendo $p$ o grau da função base. A descrição exata das geometrias é uma característica desejável em problemas que envolvem fenômenos de camada limite, os quais dependem fortemente da precisão geométrica da superfície do corpo imerso no escoamento. Alguns problemas envolvendo escoamentos turbulentos e interação fluido-estrutura, podem ser consultados em: \citeonline{BazilevsA:2010, Bazilevsetal:2007,BazilevsCH:2008,BazilevsMCH:2010,ZhangBGBH:2007}.
	
Outras metodologias aplicando diretamente funções \textit{B-Splines} também tem se mostrado eficiente para a análise de problemas da dinâmica dos fluidos computacional, como pode ser visto nos trabalhos de \citeonline{BazilevsTT:2014,BazilevsTT:2013a,HolligRW:2001}.


\subsection{Dinâmica de estruturas computacional considerando grandes deslocamentos}
\label{csdsection}

A análise de problemas de interação fluido-estrutura, muitas vezes requer a consideração da não linearidade geométrica da estrutura, devido a grandes deslocamentos ou a efeitos acoplados de membrana e flexão. Dentro desse grupo de problemas, podem-se citar o \textit{flutter} de grande amplitude, sistemas de desaceleração (como paraquedas), aplicações biomédicas, entre outros.

Atualmente a solução numérica de problemas estruturais é majoritariamente realizada por meio do método dos elementos finitos. No contexto da análise não linear geométrica de estruturas, a formulação corrotacional proposta por \citeonline{Truesdell:1955} é muito popular e descreve a mudança de configuração da estrutura, decompondo seus movimentos em rígido e de deformação, e representando-os em termos dos deslocamentos e rotações nodais. Essa formulação, aplicada a pórticos, treliças e cascas, pode ser encontrada nos trabalhos de \citeonline{Argyris:1982,BattiniP:2006,HughesL:1981a,HughesL:1981b,Ibrahimbegovic:2002,SimoF:1989}.

A formulação corrotacional, ao empregar rotações como parâmetros nodais, apresenta uma limitação para grandes deslocamentos, visto que não se pode aplicar a propriedade comutativa a essa grandeza. Para contornar esse problema, utilizam-se as formulações linearizadas de Euler-Rodrigues para a aproximação das rotações finitas, conforme pode ser observado, por exemplo, em \citeonline{CodaP:2010,GruttmanSW:2000}. A conservação de energia em problemas dinâmicos de estruturas reticuladas é um tema que ainda desperta discussões na literatura. Parte dessa controvérsia decorre do fato de que as rotações finitas mantêm sua objetividade apenas quando consideradas em pequenos incrementos. Além disso, na formulação corrotacional, a matriz de massa deixa de ser constante, o que inviabiliza o uso de métodos clássicos de integração temporal empregados na análise dinâmica linear, como o método de Newmark  \cite{SanchesC:2013}.

Motivado por \citeonline{Bonet:2000}, \citeonline{Coda:2003} introduz uma formulação baseada em posições, denominada de MEF posicional, sem o emprego de rotações como parâmetros nodais. Essa formulação tem sido aplicada com sucesso para análise de sólidos, pórticos e cascas (Coda, 2018; Carrazedo;
Coda, 2010; Coda; Paccola, 2010, 2011; Greco; Coda, 2004; Sanches;
Coda, 2016), incluindo problemas de interação fluido-estrutura (Avancini; Sanches, 2020;
Fernandes; Coda; Sanches, 2019; Sanches; Coda, 2013, 2014). Entre as vantagens da formulação posicional do MEF destaca-se ainda o fato de ela gerar uma matriz de massa constante, facilitando a realização de análises dinâmicas das estruturas.

A formulação não linear geométrica do elemento finito de casca posicional, aplicado nesse trabalho,foi proposta por \cite{CodaP:2007}, apresentando inicialmente seis graus de liberdade por nó — três associados às posições e três às componentes do vetor generalizado. Posteriormente, diante do problema de travamento volumétrico, os autores ampliaram o modelo com a introdução de um sétimo parâmetro, responsável por representar a variação linear da espessura da casca \cite{CodaP:2008}.

Em \citeonline{SanchesC:2013}, os autores utilizam o integrador temporal de Newmark para a análise de problemas dinâmicos não lineares de estruturas de cascas, no contexto da IFE, com grandes deslocamentos e rotações de corpo rígido. Nesse trabalho, os autores apresentam a demonstração da conservação da quantidade de movimento linear e angular no uso dessa metodologia, e testam a estabilidade e a conservação de energia em problemas com pequenas deformações e grandes deslocamentos, demonstrando que a formulação é adequada para os problemas de interação fluido-estrutura.

Em virtude da eficiência dessa formulação na resolução de problemas dinâmicos não lineares de estruturas, sobretudo nos casos que envolvem interação fluido-estrutura, o presente trabalho adota o MEF posicional aplicado a cascas como modelo matemático para representar as estruturas.

\subsection{Acoplamento fluido-estrutura}
\label{couplingsection}

O problema de interação fluido-estrutura pode ser descrito como um conjunto de equações diferenciais e condições de contornos associadas ao fluido e à estrutura que precisam ser satisfeitas ao mesmo tempo. Como sólidos e fluidos geralmente apresentam descrições matemáticas diferentes, sendo a mecânica dos sólidos tradicionalmente formulada por descrições Lagrangianas e a mecânica dos fluidos por descrições Eulerianas, um dos desafios da análise computacional de IFE é a compatibilização dessas diferentes descrições. Os métodos de acoplamento encontrados na literatura, em geral, podem ser classificado em dois tipos: métodos de malhas móveis, ou método de malhas conformes; e métodos de malhas fixas ou método de malhas não-conformes \cite{BazilevsTT:2013b,Houetal:2012}.

Nos métodos de malhas móveis, à medida em que a interface fluido-estrutura se movimenta, o domínio computacional do fluido é deformado, e a malha do fluido é movimentada para acomodar a mudança da interface. Nesse tipo de metodologia duas possíveis técnicas podem ser aplicadas na modelagem do domínio fluido: a descrição Lagrangiana-Euleriana arbitrária \cite{DoneaGH:1982,HughesLZ:1981,KanchiM:2007} ou a formulação Espaço-Tempo para domínios deformáveis \cite{TakizawaT:2012,TezduyarBL:1992,TezduyarBML:1992b}, sendo que ambas permitem a movimentação arbitrária (independente das partículas) da discretização espacial. A principal vantagem do método de malhas adaptadas é a capacidade de controlar o refinamento da malha do fluido próxima a interface fluido-estrutura, bem como a conformidade dos domínios, e como consequência, garantir a captura de efeitos de camada limite nessa região, garantindo precisão dos resultados.

A técnica empregada para movimentação de malhas é muito importante nos métodos de malhas móveis, pois essa deve ser eficiente de maneira a resultar em elementos que possuam mínima distorção e alteração de volume, e de forma a evitar que a malha necessite ser reconstruída. Diversas técnicas têm sido desenvolvidas para essa finalidade e podem ser divididas em três categorias. Na primeira, os deslocamentos são impostos na interface entre estrutura e fluido e o campo de deslocamentos é obtido através da resolução de um problema de valor de contorno, formulando-se o problema através de analogia com estrutura de molas \cite{BottassoDS:2005} ou com sólido elástico \cite{JohnsonT:1994,SteinTB:2004} ou emprego da equação de Laplace para distribuição dos deslocamentos \cite{KanchiM:2007}, entre outras. O segundo grupo são esquemas ponto-a-ponto, nos quais os deslocamentos da malha são diretamente interpolados a partir dos deslocamentos impostos na interface \cite{DoneaGH:1982,SanchesC:2014,TezduyarABJ:1993}. Existem ainda métodos híbridos, que combinam vantagens de diferentes técnicas de movimentação de malhas \cite{FernandesCS:2019,Lefrancois:2008}. 

Nos métodos de malhas móveis, entretanto, em alguns casos o remalhamento torna-se inevitável, como em problemas com grandes distorções do domínio, e em especial, em problemas com mudanças topológicas do domínio do fluido, fazendo com que o custo computacional se torne muito elevado.

Por sua vez, os métodos de malhas fixas são capazes de lidar com mudanças topológicas e grandes deslocamentos. Para isso, utilizam-se os chamados métodos de contornos imersos, tal como o introduzido por \citeonline{Peskin:1972}, onde mantém-se a malha do fluido fixa e permite-se que a estrutura mova-se dentro dessa malha. Nesses métodos, é necessário que as posições da estrutura sejam identificadas dentro da malha do fluido a cada passo de tempo \cite{MittalI:2005,WangRGF:2011}. Uma das formas de identificação é através de uma função distância assinalada do contorno da estrutura (método \textit{level-set}). Nesse contexto, pode-se citar os trabalhos de  \citeonline{CirakR:2005} aplicados no âmbito dos volumes finitos e de \citeonline{AkkermanBBFK:2012} e \citeonline{SanchesC:2014} em elementos finitos. A principal desvantagem desse tipo de metodologia é que a resolução da discretização na camada limite fica limitada à discretização da malha de elementos finitos onde a interface estiver posicionada no instante de análise.

Em termos da forma de solução do sistema acoplado, as técnicas de acoplamento disponíveis dividem-se em duas classes principais: métodos particionados (Bazilevs et al.,
2011; Fernandes; Coda; Sanches, 2019; Roux; Garaud, 2009; Sanches; Coda, 2013, 2014) e métodos monolíticos \cite{Avancini:2023,Blom:1998,Hubneretal:2004,HronM:2007}. No primeiro grupo, as equações do fluido e da estrutura são resolvidas separadamente, sendo as condições de acoplamento transmitidas de um meio para o outro na interface, em geral, em termos de condições de Dirichlet-Neumann ao longo do processo de solução. No segundo grupo, o dos métodos monolíticos, fluido e estrutura são tratados como entidade única, com um único sistema de equações gerado para fluido e estrutura, sendo as condições de acoplamento atendidas de maneira implícita durante o equacionamento.

As técnicas de acoplamento particionado do tipo Dirichlet-Neumann em geral consistem na aplicação de condições de contorno de Dirichlet no contorno do fluido que está em contato com a estrutura (velocidades provenienentes da movimentação da estrutura) e de Neumann no contorno do sólido que está em contato com o fluido (forças proveninentes da pressão e das tensões viscosas no fluido). Os métodos particionados podem ainda ser subdividos em acoplamentos fracos (explícitos), ou fortes (implícitos). No acoplamento particionado fraco, as equações são resolvidas de uma maneira desacoplada e só no passo de tempo seguinte as condições de acoplamento são transmitidas para de um meio para o outro. Já no o acoplamento particionado forte, as condições de acoplamento são atualizadas a cada iteração do processo de solução do sistema não linear dentro de cada passo de tempo. Esse tipo de resolução, aplicada nesse trabalho, também é conhecida como bloco-iterativa \cite{BazilevsTT:2013a}, e pode ser representada por uma  modificação da matriz tangente monolítica do método de Newton-Raphson, permitindo que os sistemas do fluido, da estrutura e da malha sejam tratados em blocos separados. Esse tipo de metodologia particionada facilita a solução dos problemas de IFE devido ao total desacoplamento entre os \textit{solvers} de estrutura e de fluido.

Os esquemas particionados podem apresentar, entretanto, algumas desvantagens, como a defasagem que pode ocorrer entre as integrações temporais do fluido e da estrutura quando as condições de contorno na interface entre fluido e estrutura são aplicadas de maneira explícita, e, ainda, instabilidades numéricas como o efeito de massa adicionada \cite{FelippaPF:2001}. Em escoamentos governados pelo campo de pressão, a ação do fluido sobre a estrutura funciona como uma massa adicional, alterando sua inércia \cite{TallecM:2001}. Em escoamentos incompressíveis, nos quais a densidade do sólido e do fluido podem ser muito próximas, ou quando a estrutura é muito esbelta, esse fenômeno pode ocasionar erros elevados e instabilidades nos acoplamentos fracos, ou  perda de convergência e instabilidades numéricas em técnicas de acoplamento particionado forte. 

Uma das formas de se contornar esse problema é a alteração do esquema de acoplamento do tipo Dirichlet-Neumann para condições de contorno de Robin, que consiste em uma combinação linear das condições de Dirichlet e Neumann, ver por exemplo, \citeonline{BadiaNV:2008}. A metodologia introduzida por \citeonline{TezduyarBL:1992}, chamada de \textit{augmented mass}, aplicada nesse trabalho, consiste em multiplicar a massa da matriz tangente respectiva à estrutura por um fator que dependerá do tipo de problema em análise, também pode ser empregada para essa finalidade. Outra metodologia, que demonstra-se muito eficiente para os casos de acoplamento do tipo bloco-iterativo, como mostram os trabalhos de \cite{FernandesCS:2019,KuttlerW:2008}, é o uso da relaxação de Aitken, proposto por \citeonline{IronsT:1969}.

\subsection{Métodos multiescala e técnicas de partição de domínios}
\label{arlequinsection}

Em diversas áreas da engenharia, faz-se necessário considerar efeitos localizados, geralmente de menor escala, dentro de um modelo global. Na análise estrutural, podem ser citados problemas envolvendo fissuras, orifícios e imperfeições; na mecânica dos fluidos, fenômenos de camada limite e a interface entre dois fluidos em escoamentos multifásicos; e, na interação fluido-estrutura, a própria interface entre estrutura e fluido.

Para obter soluções precisas nesse tipo de problema, é necessária a aplicação de técnicas que considerem os efeitos locais sem, contudo, tornar a simulação inviável pelo elevado custo computacional.

O método dos elementos finitos, tradicionalmente utilizado para a análise numérica de equações diferenciais, foi desenvolvido a partir de modelos mecânicos de meios contínuos, o que lhe confere pouca flexibilidade para a consideração desses efeitos. Os refinamentos \textit{p} e \textit{h} constituem metodologias eficientes; contudo, em determinados problemas dinâmicos, exigem técnicas de remalhamento e podem acarretar custos computacionais elevados.

Em busca de aprimorar o Método dos Elementos Finitos (MEF), diversas propostas têm sido apresentadas com o objetivo de aumentar sua flexibilidade na resolução de problemas com efeitos localizados. Entre elas, pode-se citar os elementos finitos difusos \cite{NayrolesTV:1992}, nos quais o conceito de partículas foi introduzido, resultando em uma generalização do MEF sem a necessidade de malha. Outra proposta é o método de Galerkin livre de elementos, que combina características de métodos sem malha com o MEF (ver \citeonline{Belytschko:1995}). Na mesma direção, destacam-se o método de partição da unidade \cite{MelenkB:1996}, o método dos elementos finitos generalizado (G-FEM) \cite{StrouboulisCB:2001} e o método dos elementos finitos estendido (X-FEM) \cite{Moes:2003}, os quais introduzem o enriquecimento da base aproximadora por meio de funções capazes de capturar efeitos localizados. Contudo, tanto o G-FEM quanto o X-FEM apresentam forte dependência do conhecimento prévio da solução local ou, ao menos, de sua distribuição espacial.

Pesquisas como as de \citeonline{FarhatHF:2001} propõem enriquecimentos descontínuos nos espaços funcionais, incorporando modos regulares por meio de formulações discretas de Galerkin e multiplicadores de Lagrange. Além disso, métodos de discretização que não dependem diretamente da interface, fundamentados na técnica de Nitsche, foram desenvolvidos para lidar com problemas envolvendo descontinuidades materiais, como demonstrado no estudo de \citeonline{HansboH:2002}.

No contexto da mecânica dos fluidos, \citeonline{TezduyarA:2000,TezduyarAB:1998} introduziram a técnica \textit{EDICT} (\textit{Enhanced-Discretization Interface-Capturing Technique}) para a captura de interface, com aprimoramento da discretização em problemas bifásicos ou com superfície livre. Para isso, nessa região de interface definem-se subconjuntos de elementos (sub-malhas), que posteriormente são refinados sucessivamente, de modo a melhorar a precisão da solução. Como resultado, obtém-se uma discretização mais adequada para capturar a interface; entretanto, as sub-malhas geradas não representam com exatidão as descontinuidades na interface. Uma versão mais eficiente dessa técnica foi proposta em \citeonline{TezduyarS:2005}, na qual um método iterativo multinível é projetado para a captura dos efeitos do escoamento em pequenas escalas, permitindo a simulação de problemas mais complexos.

Pode-se citar ainda o método Variacional Multiescala (VMS) \cite{Hughesetal:1998} que utiliza o conceito de micromodelos e macromodelos, sendo que os micromodelos capturam efeitos em pequenas escalas de maneira a corrigir os macromodelos, sendo muito utilizado para a obtenção de métodos estabilizados para a mecânica dos fluidos.

Outro grupo de métodos proposto para flexibilizar o MEF em problemas com efeitos locais, é o dos métodos baseados em superposição de um domínio computacional local a um domínio global. A técnica Chimera definida por \citeonline{BenekSDB:1986} traz a introdução de orifícios na região de superposição dos modelos, definindo um contorno artificial para o modelo global, e a transmissão de dados ocorre através desses contornos artificiais gerados. O método S \cite{Fish:1992} trata o modelo local como um enriquecimento ao global, e a solução é obtida através da soma dos campos de interesse de cada domínio.

O método Arlequin \cite{Dhia:1998,DhiaR:2001}, por sua vez, também baseia-se na superposição de modelos de modo a combinar um modelo local mais refinado a um global, no entanto, esse processo é realizado através do cruzamento e colagem entre os modelos em uma zona de superposição, fazendo-se isso através do uso de multiplicadores de Lagrange.  O método Arlequin vem sendo utilizado amplamente em diversas áreas da mecânica dos sólidos (ver, por exemplo, \citeonline{BaumanDEOP:2008,BiscaniGBHC:2016,CaleyronCFP:2013,DhiaJ:2010,DhiaT:2011,JamondD:2013}), na DFC e IFE, entretanto, ainda é pouco explorado. \citeonline{fernier:hal-03991421} aplica a metodologia para análise de escoamentos compressíveis, e \citeonline{FernandesEtAll:2020} utiliza uma versão estabilizada do método para análise de escoamentos incompressíveis e de IFE para problemas bidimensionais.

Ainda no contexto de superposição de domínios, pode-se citar o método de combinação dos espaços de função, proposto por \citeonline{Rosa:2022}, que consiste em ponderar as funções de forma da discretização local e da discretização global por funções de combinação e unir os espaços local e global, gerando uma nova base. No trabalho citado, a técnica foi aplicada para problemas de fratura da dinâmica de estruturas computacional para grandes deslocamentos.

No presente estudo, foram aplicadas duas metodologias de partição de domínios para análise dos fluidos dentro do contexto dos problemas de IFE tridimensionais, o método de combinação de espaços de funções e o método Arlequin estabilizado. Essas técnicas permitiram a consideração de efeitos localizados nos problemas analisados, além de possibilitarem a união de discretizações por método dos elementos finitos tradicional e análise isogeométrica.

\section[Objetivos]{Objetivos}

O principal objetivo deste trabalho é o desenvolvimento e implementação computacional de uma formulação tridimensional para análise de problemas de interação fluido-estrutura que permita a consideração de efeitos localizados no domínio do fluido por meio de técnica de partição de domínios, além de viabilizar o uso combinado de aproximações por elementos finitos clássicos e análise isogeométrica.

Para tal finalidade, enumeram-se os seguintes objetivos específicos:

\begin{itemize}
	\item Desenvolvimento de um programa para a análise bi e tridimensional de escoamentos Newtonianos incompressíveis, que permita a utilização tanto da discretização por elementos finitos quanto da discretização isogeométrica;
	
	\item Estudo de técnicas de partição de domínios para levar em conta efeitos localizados no âmbito da DFC;
	
	\item Implementação de técnicas de partição de domínios no código de dinâmica dos fluidos computacional contemplando problemas da DFC com contornos móveis;
	
	\item Estudo aprofundado da formulação Lagrangiana total baseada em posições para estruturas de cascas, bem como do código computacional para análise dinâmica de cascas desenvolvido no grupo de pesquisa em que este trabalho está inserido;
	
	\item Acoplamento entre os códigos computacionais para fluido e para estruturas através do emprego de uma técnica particionada do tipo bloco-iterativa;
	
	\item  Verificação dos códigos computacionais através da simulação de problemas da dinâmica dos fluidos, dinâmica das estruturas e de IFE e comparação com resultados da literatura.
	
\end{itemize}

\section[Metodologia]{Metodologia} 
% ----------------------------------------------------------
A base da metodologia adotada para cada sub-problema envolvido nesta tese, consiste em estudo da literatura, desenvolvimento de formulação numérica, implementação computacional e verificação do código implementado.

Em função da complexidade envolvida na implementação das ferramentas computacionais propostas, optou-se pelo uso da linguagem de programação C++ orientada a objetos, visto que esta já vem sendo utilizada com sucesso no grupo de pesquisas em que este trabalho se insere, facilitando o aproveitamento de códigos pré-existentes. Além disso, a programação orientada a objetos proporciona maior modularidade aos códigos e maior facilidade para o acoplamento entre módulos distintos. Todas as implementações são realizadas utilizando bibliotecas, compiladores e softwares livres ou de código aberto, em ambiente Linux.

Toma-se como base como base os desenvolvimentos na área de análise isogeométrica da mecânica dos fluidos de  \citeonline{Tonon:2016} e um código computacional de dinâmica dos fluidos para análises de escoamentos incompressíveis bidimensionais desenvolvido no trabalho de doutorado de \citeonline{Fernandes:2020}. Esse código é inicialmente ampliado para contemplar elementos tridimensionais. Na sequência, implementa-se nesse código discretização isogeométrica por meio de funções NURBs.

A partir desse ponto, inicia-se o estudo das metodologias de decomposição de domínios e sua implementação para problemas bidimensionais da DFC, onde são consideradas duas técnicas, o método da combinação dos espaços de funções \cite{Rosa:2021,Rosa:2022} e o Método Arlequin em sua versão estabilizada conforme o trabalho de \citeonline{Fernandes:2020}, de modo a adotar o mais eficiente para as análises de IFE.

Os estudos e desenvolvimentos em relação à mecânica das estruturas fforam focados nas estruturas de casca, com base nos trabalhos de \citeonline{Coda:2018} e Sanches e Coda (2013, 2014), sendo empregado um código computacional desenvolvido no grupo de pesquisa em linguagem de programação em C++ orientada a objeto, que engloba tanto MEF quanto AIG.

O acoplamento entre os códigos para fluido e para estrutura é desenvolvido de forma particionada forte. Para maior eficiência na resolução dos problemas, adota-se o protocolo MPI (\textit{Message passing interface}) para processamento paralelo com memória distribuída. A partição do domínio discretizado entre os processos,  é realizado através da biblioteca METIS\footnote{Disponível em: \url{http://glaros.dtc.umn.edu/gkhome/metis/metis/overview}}, e o pacote  PETSc\footnote{Disponível em: \url{http://https://www.mcs.anl.gov/petsc/}} (Portable, Extensible Toolkit for Scientific Computation) é adotada para a solução de sistemas lineares  em processamento paralelo.

Para a geração de malhas de elementos finitos emprega-se o programa GMSH\footnote{Disponível em:\url{ https://gmsh.info/}}, enquanto para a  geração dos \textit{grids} para análise isogeométrica, emprega-se o programa desenvolvido pela autora e seu orientador durante seu trabalho de mestrado \cite{Tonon:2016}. Para pós-processamento e visualização dos resultados, utilizam-se os programas Kitware Paraview\footnote{Disponível em:\url{ http://https://www.paraview.org/}} e  Gnuplot\footnote{Disponível em:\url{ https://gnuplot.info/}}. 

No que diz respeito à infraestrutura, utiliza-se o \textit{cluster} disponível no Laboratório de Informática e de Mecânica Computacional (LIMC) do SET para a simulação de problemas mais complexos, e um computador pessoal para a simulação de problemas mais simples.

\section[Justificativa]{Justificativa}

A motivação central desta pesquisa decorre da relevância científica e tecnológica dos problemas de interação fluido-estrutura (IFE) e das limitações das metodologias atualmente disponíveis para sua análise numérica. Embora os avanços recentes em dinâmica dos fluidos computacional, mecânica dos sólidos computacional e técnicas de acoplamento tenham possibilitado progressos significativos na modelagem desses problemas, ainda persistem desafios importantes, especialmente quando se trata de situações envolvendo grandes deslocamentos estruturais, escoamentos tridimensionais incompressíveis a altos números de Reynolds e mudanças topológicas no domínio do fluido.
	
Do ponto de vista computacional, os métodos tradicionais de acoplamento com malhas móveis apresentam limitações relacionadas ao elevado custo de remalhamentos sucessivos, inevitáveis em problemas com grandes distorções do domínio fluido. Por outro lado, os métodos de malhas fixas, embora adequados para lidar com mudanças topológicas, apresentam deficiências na representação de fenômenos localizados, como os efeitos de camada limite em torno de estruturas imersas. Assim, existe uma lacuna metodológica que justifica a busca por técnicas capazes de combinar as vantagens das abordagens existentes e, ao mesmo tempo, mitigar suas limitações.
	
Nesse contexto, a proposta de utilizar técnicas de decomposição de domínio com malhas superpostas apresenta-se como uma alternativa promissora. A sobreposição de malhas globais (fixas e menos refinadas) à malhas locais (móveis e conformes à estrutura) possibilita tanto o tratamento adequado de grandes deslocamentos e mudanças topológicas quanto a captura precisa de efeitos de fronteira, reduzindo a necessidade de remalhamentos extensivos. Além disso, a adoção de diferentes discretizações — combinando elementos finitos e análise isogeométrica — permite explorar as vantagens de cada metodologia: a flexibilidade geométrica dos elementos finitos e a descrição exata de superfícies e continuidade elevada das funções NURBS.

Adicionalmente, a literatura ainda é incipiente no que se refere à aplicação conjunta da análise isogeométrica e do método dos elementos finitos em problemas de IFE tridimensionais. A investigação do método de combinação dos espaços de funções e do método Arlequin estabilizado, aplicados ao acoplamento entre malhas globais e locais, representa uma contribuição original deste trabalho, tanto no âmbito teórico quanto no computacional. Em particular, a avaliação da robustez do Arlequin estabilizado para escoamentos incompressíveis em altos números de Reynolds e sua extensão para a análise de problemas de IFE tridimensionais configuram avanços relevantes frente às metodologias atualmente disponíveis.
	
Portanto, esta tese justifica-se pela necessidade de desenvolver e consolidar técnicas numéricas mais eficientes e robustas para a análise de problemas complexos de interação fluido-estrutura. O desenvolvimento proposto contribuirá não apenas para o avanço do conhecimento científico na área de mecânica computacional, mas também para aplicações práticas em engenharia, tais como análise aeroelástica, projeto de estruturas submetidas à ação do vento, dinâmica de sistemas biomecânicos e estudo de fenômenos hidrodinâmicos em engenharia naval e oceânica.
