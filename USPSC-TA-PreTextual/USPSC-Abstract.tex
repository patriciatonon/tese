%% USPSC-Abstract.tex
%\autor{Silva, M. J.}
\begin{resumo}[Abstract]
 \begin{otherlanguage*}{english}
	\begin{flushleft} 
		\setlength{\absparsep}{0pt} % ajusta o espaçamento dos parágrafos do resumo		
 		\SingleSpacing  		\imprimirautorabr~~\textbf{\imprimirtitleabstract}.	\imprimirdata.  \pageref{LastPage} p. 
		%Substitua p. por f. quando utilizar oneside em \documentclass
		%\pageref{LastPage} f.
		\imprimirtipotrabalhoabs~-~\imprimirinstituicao, \imprimirlocal, 	\imprimirdata. 
 	\end{flushleft}
	\OnehalfSpacing 
This work presents the development of a computational tool for numerical analysis of fluid-structure interaction problems, in which the fluid domain is discretized by combining approximations based on Isogeometric Analysis (IGA) and the traditional Finite Element Method (FEM). Incompressible flows are considered, with the fluid domain represented by a fixed global discretization, nonconforming with the structure, upon which a refined local discretization, adapted to the fluid–structure interface, is superposed. The incompressible flow is solved through a stabilized formulation that allows equal-order interpolation for velocity and pressure fields, with implicit time integration performed using the $\alpha$-generalized method. The coupling between the local and global fluid models is handled by a stabilized formulation of the Arlequin method, which consists in superposing two models—one discretized by finite elements and the other by isogeometric analysis—and enforcing compatibility through a field of Lagrange multipliers defined over a region called the gluing zone. To ensure the stability of the multiplier field and enhance the flexibility of the formulation, a stabilization term based on the residual of the governing equations is added. The motion of the local fluid model, as well as the coupling with the structure, is achieved through an Arbitrary Lagrangian–Eulerian (ALE) description of the governing equations. The structure is modeled using Reissner–Mindlin shell elements within a positional finite element formulation under a total Lagrangian description, suitable for dynamic analyses involving large displacements. The fluid–structure coupling is performed using a strong partitioned block-iterative scheme that ensures consistent interaction between the two media. In addition to providing a discretization capable of capturing localized effects near the structure—such as boundary-layer phenomena—the proposed methodology combines the advantages of moving-mesh and fixed-mesh (immersed boundary) methods. This approach enables efficient simulation of problems that, in the context of traditional moving-mesh methods, would require global remeshing, while also offering greater flexibility in the choice of discretizations and improved computational performance in complex three-dimensional FSI analyses.

\vspace{\onelineskip}

\noindent 
   \textbf{Keywords}: Fluid-structure interaction. Isogeometric analysis. Finite Element Method. Arlequin Method. Arbitrary Lagrangian-Eulerian Description.
 \end{otherlanguage*}
\end{resumo}
