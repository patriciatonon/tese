%% USPSC-Resumo.tex
\setlength{\absparsep}{18pt} % ajusta o espaçamento dos parágrafos do resumo		
\begin{resumo}
	\begin{flushleft} 
			\setlength{\absparsep}{0pt} % ajusta o espaçamento da referência	
			\SingleSpacing 
			\imprimirautorabr~~\textbf{\imprimirtituloresumo}.	\imprimirdata. \pageref{LastPage} p. 
			%Substitua p. por f. quando utilizar oneside em \documentclass
			%\pageref{LastPage} f.
			\imprimirtipotrabalho~-~\imprimirinstituicao, \imprimirlocal, \imprimirdata. 
 	\end{flushleft}
\OnehalfSpacing 			
Este trabalho apresenta o desenvolvimento de uma ferramenta computacional para a análise numérica de problemas de interação fluido-estrutura, em que o domínio do fluido é discretizado por meio da combinação de aproximações baseadas na Análise Isogeométrica e no Método dos Elementos Finitos tradicional. Consideram-se escoamentos incompressíveis, sendo o domínio fluido representado por uma discretização global fixa e não conforme à estrutura, sobre a qual se sobrepõe uma discretização local, mais refinada e adaptada à interface fluido-estrutura.
O escoamento incompressível, é solucionado por meio de uma formulação estabilizada, permitindo aproximações de mesma ordem para as variáveis de velocidade e pressão, com integração temporal implícita realizada pelo método $\alpha$-generalizado. O acoplamento entre os modelos local e global de fluido é tratado por uma formulação estabilizada do método Arlequin, que consiste em superpor dois modelos — um discretizado por elementos finitos e outro por análise isogeométrica — e compatibilizá-los por meio de um campo de multiplicadores de Lagrange definido sobre uma região denominada zona de colagem. Para garantir a estabilidade do campo de multiplicadores e ampliar a flexibilidade da formulação, adiciona-se uma parcela estabilizadora baseada no resíduo da equação governante. A movimentação do modelo local de fluido, bem como o acoplamento com a estrutura, são viabilizados pela adoção de uma descrição Lagrangiana-Euleriana Arbitrária das equações governantes.
A estrutura é modelada por meio de elementos de casca com cinemática de Reissner-Mindlin, utilizando uma formulação posicional do método dos elementos finitos em descrição Lagrangiana total, adequada à análise dinâmica com grandes deslocamentos. O acoplamento fluido-estrutura é realizado por um esquema particionado forte do tipo bloco-iterativo, que assegura a interação consistente entre os dois meios.Além de garantir uma discretização com resolução suficiente para capturar efeitos localizados próximos à estrutura, como os associados à camada limite, a metodologia proposta combina as vantagens dos métodos de malhas móveis e de malhas fixas (contornos imersos), permitindo a simulação eficiente de problemas que, no contexto de métodos de malhas móveis tradicionais, exigiriam remalhamento global. A abordagem desenvolvida também proporciona maior flexibilidade na escolha das discretizações e melhora o desempenho computacional em análises tridimensionais complexas de interação fluido-estrutura.

\textbf{Palavras-chave}: Interação Fluido-Estrutura.  Análise Isogeométrica. Método dos Elementos Finitos. Método Arlequin. Descrição Lagrangiana-Euleriana Arbitrária.
\end{resumo}

