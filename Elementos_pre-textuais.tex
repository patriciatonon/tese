%Elementos Pr�-Textuais

\instituicao{Escola de Engenharia de S\~ao Carlos, Universidade de S\~ao Paulo}
\unidade{ESCOLA DE ENGENHARIA DE S\~AO CARLOS}
\unidademin{Escola de Engenharia de S\~ao Carlos}
\universidademin{Universidade de S\~ao Paulo}
% A EESC n�o inclui a nota "Vers�o original", portanto o comando abaixo n�o tem a mensagem entre {}
\notafolharosto{Universidade de S\~ao Paulo}
%Para a vers�o corrigida tire a % do comando/declara��o abaixo e inclua uma % antes do comando acima
%\notafolharosto{VERS\~AO CORRIGIDA}

% dados complementares para CAPA e FOLHA DE ROSTO
\universidade{UNIVERSIDADE DE S\~AO PAULO}
%\titulo{An\'alise num\'erica de efeitos termo-mec\^anicos em s\'olidos incluindo mudan\c{c}a de fase e situa\c{c}\~oes de contato}
\titulo{An\'alise num\'erica de fen\^omenos termo-mec\^anicos com mudan\c{c}a de fase s\'olido-l\'iquido em grandes deforma\c{c}\~oes}
% An�lise num�rica de fen�menos termo-mec�nicos com mudan�a de fase s�lido-l�quido em grandes deforma��es
% P�ricles Rafael Pav�o Carvalho
% Rodolfo Andr� Kuche Sanches
\titleabstract{Numerical analysis of thermo-mechanical phenomena with phase change and large strain constitutive models}
\autor{P\'ericles Rafael Pav\~ao Carvalho}
\autorficha{Carvalho, P\'ericles Rafael Pav\~ao}
\autorabr{CARVALHO, P. R. P.}
% \notafolharosto{VERS\~AO CORRIGIDA \\ (A vers\~ao original encontra-se na Escola de Engenharia de S\~ao Carlos)}

\cutter{S856m}
% Para gerar a ficha catalogr�fica sem o C�digo Cutter, basta 
% incluir uma % na linha acima e tirar a % da linha abaixo
%\cutter{ }

\local{S\~ao Carlos}
\data{\the\year}
% Quando for Orientador, basta incluir uma % antes do comando abaixo
%\renewcommand{\orientadorname}{Orientadora:}
%\newcommand{\coorientadorname}{Coorientador:}
\orientador{Prof. Dr. Rodolfo Andr\'e Kuche Sanches}
\orientadorcorpoficha{orientador Rodolfo Andr\'e Kuche Sanches}
\orientadorficha{Sanches, Rodolfo Andr\'e Kuche, orient}
%Se houver co-orientador, inclua % antes das duas linhas (antes dos comandos \orientadorcorpoficha e \orientadorficha) 
%          e tire a % antes dos 3 comandos abaixo
%\coorientador{Prof. Dr. Jo\~ao Alves Serqueira}
%\orientadorcorpoficha{orientadora Elisa Gon\c{c}alves Rodrigues ;  co-orientador Jo\~ao Alves Serqueira}
%\orientadorficha{Rodrigues, Elisa Gon\c{c}alves, orient. II. Serqueira, Jo\~ao Alves, co-orient}

\notaautorizacao{AUTORIZO A REPRODU\c{C}\~AO E DIVULGA\c{C}\~AO TOTAL OU PARCIAL DESTE TRABALHO, POR QUALQUER MEIO CONVENCIONAL OU ELETR\^ONICO PARA FINS DE ESTUDO E PESQUISA, DESDE QUE CITADA A FONTE.}
\notabib{~  ~}

\tipotrabalho{Tese}
\area{Estruturas}
%\opcao{Nome da Op��o}
% O preambulo deve conter o tipo do trabalho, o objetivo, 
% o nome da institui��o, a �rea de concentra��o e op��o quando houver
\preambulo{Relat\'orio cient\'ifico apresentado \`a FAPESP abrangendo o per\'iodo entre 01/05/2019 e 10/04/2020, relativo ao processo de n\'umero 2018/23957-2 para doutorado fluxo cont\'inuo (per\'iodo de vig\^encia de 01/05/2019 a 30/04/2022).}
% \notaficha{Disserta\c{c}\~ao (Mestrado) - Programa de P\'os-Gradua\c{c}\~ao em Engenharia Civil (Engenharia de Estruturas) e \'Area de Concentra\c{c}\~ao em Estruturas}



