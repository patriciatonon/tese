%% USPSC-pre-textual-EESC.tex
%% Camandos para definição do tipo de documento (tese ou dissertação), área de concentração, opção, preâmbulo, titulação 
%% referentes aos Programas de Pós-Graduação
\instituicao{Escola de Engenharia de S\~ao Carlos, Universidade de S\~ao Paulo}
\unidade{ESCOLA DE ENGENHARIA DE S\~AO CARLOS}
\unidademin{Escola de Engenharia de S\~ao Carlos}
\universidademin{Universidade de S\~ao Paulo}

% A EESC não inclui a nota "Versão original", portanto o comando abaixo não tem a mensagem entre {}
\notafolharosto{ }
%Para a versão corrigida tire a % do comando/declaração abaixo e inclua uma % antes do comando acima
%\notafolharosto{VERS\~AO CORRIGIDA}
% ---
% dados complementares para CAPA e FOLHA DE ROSTO
% ---
\universidade{UNIVERSIDADE DE S\~AO PAULO}
\titulo{Combinação de discretizações isogeométrica e por elementos finitos na análise de interação fluido-estrutura}
\titleabstract{Combination of isogeometric and finite element discretizations for fluid-structure interaction analysis}
\tituloresumo{Combinação de discretizações isogeométrica e por elementos finitos na análise de interação fluido-estrutura}
\autor{Patr\'icia Tonon}
\autorficha{Tonon, Patr\'icia}
\autorabr{TONON, P.}

\cutter{S856m}
% Para gerar a ficha catalográfica sem o Código Cutter, basta 
% incluir uma % na linha acima e tirar a % da linha abaixo
%\cutter{ }

\local{S\~ao Carlos}
\data{2025}
% Quando for Orientador, basta incluir uma % antes do comando abaixo
\renewcommand{\orientadorname}{Orientador:}
% Quando for Coorientadora, basta tirar a % utilizar o comando abaixo
%\newcommand{\coorientadorname}{Coorientador:}
\orientador{Prof. Dr. Rodolfo André Kuche Sanches}
\orientadorcorpoficha{orientador Rodolfo André Kuche Sanches}
\orientadorficha{Sanches, Rodolfo André Kuche, orient}

%Se houver co-orientador, inclua % antes das duas linhas (antes dos comandos \orientadorcorpoficha e \orientadorficha) 
%          e tire a % antes dos 3 comandos abaixo
%\coorientador{Prof. Dr. Jo\~ao Alves Serqueira}
%\orientadorcorpoficha{orientadora Elisa Gon\c{c}alves Rodrigues ;  co-orientador Jo\~ao Alves Serqueira}
%\orientadorficha{Rodrigues, Elisa Gon\c{c}alves, orient. II. Serqueira, Jo\~ao Alves, co-orient}

\notaautorizacao{AUTORIZO A REPRODU\c{C}\~AO E DIVULGA\c{C}\~AO TOTAL OU PARCIAL DESTE TRABALHO, POR QUALQUER MEIO CONVENCIONAL OU ELETR\^ONICO PARA FINS DE ESTUDO E PESQUISA, DESDE QUE CITADA A FONTE.}
\notabib{~  ~}

\newcommand{\programa}[1]{

% DCEA ==========================================================================
\ifthenelse{\equal{#1}{DCEA}}{
    \tipotrabalho{Tese (Doutorado)}
    \tipotrabalhoabs{Thesis (Doctor)}
    \area{Ci\^encias da Engenharia Ambiental}
	%\opcao{Nome da Opção}
    % O preambulo deve conter o tipo do trabalho, o objetivo, 
	% o nome da instituição, a área de concentração e opção quando houver
	\preambulo{Tese apresentada \`a Escola de Engenharia de S\~ao Carlos da Universidade de S\~ao Paulo, para obten\c{c}\~ao do t\'itulo de Doutor em Ci\^encias - Programa de P\'os-Gradua\c{c}\~ao em Ci\^encias da Engenharia Ambiental.}
	\notaficha{Tese (Doutorado) - Programa de P\'os-Gradua\c{c}\~ao e \'Area de Concentra\c{c}\~ao em Ci\^encias da Engenharia Ambiental}
    }{
% MCEA ===========================================================================
\ifthenelse{\equal{#1}{MCEA}}{
	\tipotrabalho{Disserta\c{c}\~ao (Mestrado)}
	\tipotrabalhoabs{Dissertation (Master)}
	\area{Ci\^encias da Engenharia Ambiental}
	%\opcao{Nome da Opção}
	% O preambulo deve conter o tipo do trabalho, o objetivo, 
	% o nome da instituição, a área de concentração e opção quando houver
	\preambulo{Disserta\c{c}\~ao apresentada \`a Escola de Engenharia de S\~ao Carlos da Universidade de S\~ao Paulo, para obten\c{c}\~ao do t\'itulo de Mestre em Ci\^encias - Programa de P\'os-Gradua\c{c}\~ao em Ci\^encias da Engenharia Ambiental.}
	\notaficha{Disserta\c{c}\~ao (Mestrado) - Programa de P\'os-Gradua\c{c}\~ao e \'Area de Concentra\c{c}\~ao em Ci\^encias da Engenharia Ambiental}
        }{
% DEE =======================================================================
\ifthenelse{\equal{#1}{DEE}}{
    \tipotrabalho{Tese (Doutorado)}
    \tipotrabalhoabs{Thesis (Doctor)}
    \area{Estruturas}
	%\opcao{Nome da Opção}
    % O preambulo deve conter o tipo do trabalho, o objetivo, 
	% o nome da instituição, a área de concentração e opção quando houver
	\preambulo{Tese apresentada \`a Escola de Engenharia de S\~ao Carlos da Universidade de S\~ao Paulo, para obten\c{c}\~ao do t\'itulo de Doutor em Ci\^encias - Programa de P\'os-Gradua\c{c}\~ao em Engenharia Civil (Engenharia de Estruturas).}
	\notaficha{Tese (Doutorado) - Programa de P\'os-Gradua\c{c}\~ao em Engenharia Civil (Engenharia de Estruturas) e \'Area de Concentra\c{c}\~ao em Estruturas}
    }{
% MEE ===========================================================================
\ifthenelse{\equal{#1}{MEE}}{
	\tipotrabalho{Disserta\c{c}\~ao (Mestrado)}
	\tipotrabalhoabs{Dissertation (Master)}
	\area{Estruturas}
	%\opcao{Nome da Opção}
	% O preambulo deve conter o tipo do trabalho, o objetivo, 
	% o nome da instituição, a área de concentração e opção quando houver
	\preambulo{Disserta\c{c}\~ao apresentada \`a Escola de Engenharia de S\~ao Carlos da Universidade de S\~ao Paulo, para obten\c{c}\~ao do t\'itulo de Mestre em Ci\^encias - Programa de P\'os-Gradua\c{c}\~ao em Engenharia Civil (Engenharia de Estruturas).}
	\notaficha{Disserta\c{c}\~ao (Mestrado) - Programa de P\'os-Gradua\c{c}\~ao em Engenharia Civil (Engenharia de Estruturas) e \'Area de Concentra\c{c}\~ao em Estruturas}
    }{
% DEPP ==========================================================================
\ifthenelse{\equal{#1}{DEPP}}{
    \tipotrabalho{Tese (Doutorado)}
    \tipotrabalhoabs{Thesis (Doctor)}
    \area{Processos e Gest\~ao de Opera\c{c}\~oes}
	%\opcao{Nome da Opção}
    % O preambulo deve conter o tipo do trabalho, o objetivo, 
	% o nome da instituição, a área de concentração e opção quando houver
	\preambulo{Tese apresentada \`a Escola de Engenharia de S\~ao Carlos da Universidade de S\~ao Paulo, para obten\c{c}\~ao do t\'itulo de Doutor em Ci\^encias - Programa de P\'os-Gradua\c{c}\~ao em Engenharia de Produ\c{c}\~ao.}
	\notaficha{Tese (Doutorado) - Programa de P\'os-Gradua\c{c}\~ao em Engenharia de Produ\c{c}\~ao e \'Area de Concentra\c{c}\~ao em Processos e Gest\~ao de Opera\c{c}\~oes}
    }{
% MEPP ===========================================================================
\ifthenelse{\equal{#1}{MEPP}}{
	\tipotrabalho{Disserta\c{c}\~ao (Mestrado)}
	\tipotrabalhoabs{Dissertation (Master)}
	\area{Processos e Gest\~ao de Opera\c{c}\~oes}
	%\opcao{Nome da Opção}
	% O preambulo deve conter o tipo do trabalho, o objetivo, 
	% o nome da instituição, a área de concentração e opção quando houver
	\preambulo{Disserta\c{c}\~ao apresentada \`a Escola de Engenharia de S\~ao Carlos da Universidade de S\~ao Paulo, para obten\c{c}\~ao do t\'itulo de Mestre em Ci\^encias - Programa de P\'os-Gradua\c{c}\~ao em Engenharia de Produ\c{c}\~ao.}
	\notaficha{Disserta\c{c}\~ao (Mestrado) - Programa de P\'os-Gradua\c{c}\~ao em Engenharia de Produ\c{c}\~ao e \'Area de Concentra\c{c}\~ao em Processos e Gest\~ao de Opera\c{c}\~oes}
    }{			
% DEPE ==========================================================================
\ifthenelse{\equal{#1}{DEPE}}{
    \tipotrabalho{Tese (Doutorado)}
    \tipotrabalhoabs{Thesis (Doctor)}
    \area{Economia, Organiza\c{c}\~oes e Gest\~ao do Conhecimento}
	%\opcao{Nome da Opção}
    % O preambulo deve conter o tipo do trabalho, o objetivo, 
	% o nome da instituição, a área de concentração e opção quando houver
	\preambulo{Tese apresentada \`a Escola de Engenharia de S\~ao Carlos da Universidade de S\~ao Paulo, para obten\c{c}\~ao do t\'itulo de Doutor em Ci\^encias - Programa de P\'os-Gradua\c{c}\~ao em Engenharia de Produ\c{c}\~ao.}
	\notaficha{Tese (Doutorado) - Programa de P\'os-Gradua\c{c}\~ao em Engenharia de Produ\c{c}\~ao e \'Area de Concentra\c{c}\~ao em Economia, Organiza\c{c}\~oes e Gest\~ao do Conhecimento}
    }{
% MEPE ===========================================================================
\ifthenelse{\equal{#1}{MEPE}}{
	\tipotrabalho{Disserta\c{c}\~ao (Mestrado)}
	\tipotrabalhoabs{Dissertation (Master)}
	\area{Economia, Organiza\c{c}\~oes e Gest\~ao do Conhecimento}
	%\opcao{Nome da Opção}
	% O preambulo deve conter o tipo do trabalho, o objetivo, 
	% o nome da instituição, a área de concentração e opção quando houver
	\preambulo{Disserta\c{c}\~ao apresentada \`a Escola de Engenharia de S\~ao Carlos da Universidade de S\~ao Paulo, para obten\c{c}\~ao do t\'itulo de Mestre em Ci\^encias - Programa de P\'os-Gradua\c{c}\~ao em Engenharia de Produ\c{c}\~ao.}
	\notaficha{Disserta\c{c}\~ao (Mestrado) - Programa de P\'os-Gradua\c{c}\~ao em Engenharia de Produ\c{c}\~ao e \'Area de Concentra\c{c}\~ao em Economia, Organiza\c{c}\~oes e Gest\~ao do Conhecimento}
    }{			
% DETI ==========================================================================
\ifthenelse{\equal{#1}{DETI}}{
	\tipotrabalho{Tese (Doutorado)}
    \tipotrabalhoabs{PhD Dissertation}
    \area{Infraestrutura de Transportes}
	%\opcao{Nome da Opção}
    % O preambulo deve conter o tipo do trabalho, o objetivo, 
	% o nome da instituição, a área de concentração e opção quando houver
	\preambulo{Tese apresentada \`a Escola de Engenharia de S\~ao Carlos da Universidade de S\~ao Paulo, para obten\c{c}\~ao do t\'itulo de Doutor em Ci\^encias - Programa de P\'os-Gradua\c{c}\~ao em Engenharia de Transportes.}
	\notaficha{Tese (Doutorado) - Programa de P\'os-Gradua\c{c}\~ao em Engenharia de Transportes e \'Area de Concentra\c{c}\~ao em Infraestrutura de Transportes}
    }{
% METI ===========================================================================
\ifthenelse{\equal{#1}{METI}}{
	\tipotrabalho{Disserta\c{c}\~ao (Mestrado)}
	\tipotrabalhoabs{M.Sc. Thesis}
	\area{Infraestrutura de Transportes}
	%\opcao{Nome da Opção}
	% O preambulo deve conter o tipo do trabalho, o objetivo, 
	% o nome da instituição, a área de concentração e opção quando houver
	\preambulo{Disserta\c{c}\~ao apresentada \`a Escola de Engenharia de S\~ao Carlos da Universidade de S\~ao Paulo, para obten\c{c}\~ao do t\'itulo de Mestre em Ci\^encias - Programa de P\'os-Gradua\c{c}\~ao em Engenharia de Transportes.}
	\notaficha{Disserta\c{c}\~ao (Mestrado) - Programa de P\'os-Gradua\c{c}\~ao em Engenharia de Transportes e \'Area de Concentra\c{c}\~ao em Infraestrutura de Transportes}
    }{	   			
% DETP ==========================================================================
\ifthenelse{\equal{#1}{DETP}}{
    \tipotrabalho{Tese (Doutorado)}
    \tipotrabalhoabs{PhD Dissertation}
    \area{Planejamento e Opera\c{c}\~ao de Sistemas de Transporte}
	%\opcao{Nome da Opção}
    % O preambulo deve conter o tipo do trabalho, o objetivo, 
	% o nome da instituição, a área de concentração e opção quando houver
	\preambulo{Tese apresentada \`a Escola de Engenharia de S\~ao Carlos da Universidade de S\~ao Paulo, para obten\c{c}\~ao do t\'itulo de Doutor em Ci\^encias - Programa de P\'os-Gradua\c{c}\~ao em Engenharia de Transportes.}
	\notaficha{Tese (Doutorado) - Programa de P\'os-Gradua\c{c}\~ao em Engenharia de Transportes e \'Area de Concentra\c{c}\~ao em Planejamento e Opera\c{c}\~ao de Sistemas de Transporte}
    }{
% METP ===========================================================================
\ifthenelse{\equal{#1}{METP}}{
	\tipotrabalho{Disserta\c{c}\~ao (Mestrado)}
	\tipotrabalhoabs{M.Sc. Thesis}
	\area{Planejamento e Opera\c{c}\~ao de Sistemas de Transporte}
	%\opcao{Nome da Opção}
	% O preambulo deve conter o tipo do trabalho, o objetivo, 
	% o nome da instituição, a área de concentração e opção quando houver
	\preambulo{Disserta\c{c}\~ao apresentada \`a Escola de Engenharia de S\~ao Carlos da Universidade de S\~ao Paulo, para obten\c{c}\~ao do t\'itulo de Mestre em Ci\^encias - Programa de P\'os-Gradua\c{c}\~ao em Engenharia de Transportes.}
	\notaficha{Disserta\c{c}\~ao (Mestrado) - Programa de P\'os-Gradua\c{c}\~ao em Engenharia de Transportes e \'Area de Concentra\c{c}\~ao em Planejamento e Opera\c{c}\~ao de Sistemas de Transporte}
    }{	    
% DEEP ==========================================================================
\ifthenelse{\equal{#1}{DEEP}}{
    \tipotrabalho{Tese (Doutorado)}
    \tipotrabalhoabs{Thesis (Doctor)}
    \area{Processamento de Sinais e Instrumenta\c{c}\~ao}
	%\opcao{Nome da Opção}
	% O preambulo deve conter o tipo do trabalho, o objetivo, 
	% o nome da instituição, a área de concentração e opção quando houver
	\preambulo{Tese apresentada \`a Escola de Engenharia de S\~ao Carlos da Universidade de S\~ao Paulo, para obten\c{c}\~ao do t\'itulo de Doutor em Ci\^encias - Programa de P\'os-Gradua\c{c}\~ao em Engenharia El\'etrica.}
	\notaficha{Tese (Doutorado) - Programa de P\'os-Gradua\c{c}\~ao em Engenharia El\'etrica e \'Area de Concentra\c{c}\~ao em Processamento de Sinais e Instrumenta\c{c}\~ao}
    }{
% MEEP ===========================================================================
\ifthenelse{\equal{#1}{MEEP}}{
	\tipotrabalho{Disserta\c{c}\~ao (Mestrado)}
	\tipotrabalhoabs{Dissertation (Master)}
	\area{Processamento de Sinais e Instrumenta\c{c}\~ao}
	%\opcao{Nome da Opção}
	% O preambulo deve conter o tipo do trabalho, o objetivo, 
	% o nome da instituição, a área de concentração e opção quando houver
	\preambulo{Disserta\c{c}\~ao apresentada \`a Escola de Engenharia de S\~ao Carlos da Universidade de S\~ao Paulo, para obten\c{c}\~ao do t\'itulo de Mestre em Ci\^encias - Programa de P\'os-Gradua\c{c}\~ao em Engenharia El\'etrica.}
	\notaficha{Disserta\c{c}\~ao (Mestrado) - Programa de P\'os-Gradua\c{c}\~ao em Engenharia El\'etrica e \'Area de Concentra\c{c}\~ao em Processamento de Sinais e Instrumenta\c{c}\~ao}
    }{	  
% DEED ==========================================================================
\ifthenelse{\equal{#1}{DEED}}{
    \tipotrabalho{Tese (Doutorado)}
    \tipotrabalhoabs{Thesis (Doctor)}
    \area{Sistemas Din\^amicos}
	%\opcao{Nome da Opção}
    % O preambulo deve conter o tipo do trabalho, o objetivo, 
	% o nome da instituição, a área de concentração e opção quando houver
	\preambulo{Tese apresentada \`a Escola de Engenharia de S\~ao Carlos da Universidade de S\~ao Paulo, para obten\c{c}\~ao do t\'itulo de Doutor em Ci\^encias - Programa de P\'os-Gradua\c{c}\~ao em Engenharia El\'etrica.}
	\notaficha{Tese (Doutorado) - Programa de P\'os-Gradua\c{c}\~ao em Engenharia El\'etrica e \'Area de Concentra\c{c}\~ao em Sistemas Din\^amicos}
    }{
% MEED ===========================================================================
\ifthenelse{\equal{#1}{MEED}}{
	\tipotrabalho{Disserta\c{c}\~ao (Mestrado)}
	\tipotrabalhoabs{Dissertation (Master)}
	\area{Sistemas Din\^amicos}
	%\opcao{Nome da Opção}
	% O preambulo deve conter o tipo do trabalho, o objetivo, 
	% o nome da instituição, a área de concentração e opção quando houver
	\preambulo{Disserta\c{c}\~ao apresentada \`a Escola de Engenharia de S\~ao Carlos da Universidade de S\~ao Paulo, para obten\c{c}\~ao do t\'itulo de Mestre em Ci\^encias - Programa de P\'os-Gradua\c{c}\~ao em Engenharia El\'etrica.}
	\notaficha{Disserta\c{c}\~ao (Mestrado) - Programa de P\'os-Gradua\c{c}\~ao em Engenharia El\'etrica e \'Area de Concentra\c{c}\~ao em Sistemas Din\^amicos}
    }{	  
% DEEE ==========================================================================
\ifthenelse{\equal{#1}{DEEE}}{
    \tipotrabalho{Tese (Doutorado)}
    \tipotrabalhoabs{Thesis (Doctor)}
    \area{Sistemas El\'etricos de Pot\^encia}
	%\opcao{Nome da Opção}
    % O preambulo deve conter o tipo do trabalho, o objetivo, 
	% o nome da instituição, a área de concentração e opção quando houver
	\preambulo{Tese apresentada \`a Escola de Engenharia de S\~ao Carlos da Universidade de S\~ao Paulo, para obten\c{c}\~ao do t\'itulo de Doutor em Ci\^encias - Programa de P\'os-Gradua\c{c}\~ao em Engenharia El\'etrica.}
	\notaficha{Tese (Doutorado) - Programa de P\'os-Gradua\c{c}\~ao em Engenharia El\'etrica e \'Area de Concentra\c{c}\~ao em Sistemas El\'etricos de Pot\^encia}
    }{
% MEEE ===========================================================================
\ifthenelse{\equal{#1}{MEEE}}{
    \tipotrabalho{Disserta\c{c}\~ao (Mestrado)}
    \tipotrabalhoabs{Dissertation (Master)}
    \area{Sistemas El\'etricos de Pot\^encia}
	%\opcao{Nome da Opção}
    % O preambulo deve conter o tipo do trabalho, o objetivo, 
	% o nome da instituição, a área de concentração e opção quando houver
	\preambulo{Disserta\c{c}\~ao apresentada \`a Escola de Engenharia de S\~ao Carlos da Universidade de S\~ao Paulo, para obten\c{c}\~ao do t\'itulo de Mestre em Ci\^encias - Programa de P\'os-Gradua\c{c}\~ao em Engenharia El\'etrica.}
	\notaficha{Disserta\c{c}\~ao (Mestrado) - Programa de P\'os-Gradua\c{c}\~ao em Engenharia El\'etrica e \'Area de Concentra\c{c}\~ao em Sistemas El\'etricos de Pot\^encia}
    }{	  
% DEET ==========================================================================
\ifthenelse{\equal{#1}{DEET}}{
    \tipotrabalho{Tese (Doutorado)}
    \tipotrabalhoabs{Thesis (Doctor)}
    \area{Telecomunica\c{c}\~oes}
	%\opcao{Nome da Opção}
    % O preambulo deve conter o tipo do trabalho, o objetivo, 
	% o nome da instituição, a área de concentração e opção quando houver
	\preambulo{Tese apresentada \`a Escola de Engenharia de S\~ao Carlos da Universidade de S\~ao Paulo, para obten\c{c}\~ao do t\'itulo de Doutor em Ci\^encias - Programa de P\'os-Gradua\c{c}\~ao em Engenharia El\'etrica.}
	\notaficha{Tese (Doutorado) - Programa de P\'os-Gradua\c{c}\~ao em Engenharia El\'etrica e \'Area de Concentra\c{c}\~ao em Telecomunica\c{c}\~oes}
    }{
% MEET ===========================================================================
\ifthenelse{\equal{#1}{MEET}}{
	\tipotrabalho{Disserta\c{c}\~ao (Mestrado)}
	\area{Telecomunica\c{c}\~oes}
	%\opcao{Nome da Opção}
	% O preambulo deve conter o tipo do trabalho, o objetivo, 
	% o nome da instituição, a área de concentração e opção quando houver
	\preambulo{Disserta\c{c}\~ao apresentada \`a Escola de Engenharia de S\~ao Carlos da Universidade de S\~ao Paulo, para obten\c{c}\~ao do t\'itulo de Mestre em Ci\^encias - Programa de P\'os-Gradua\c{c}\~ao em Engenharia El\'etrica.}
	\notaficha{Disserta\c{c}\~ao (Mestrado) - Programa de P\'os-Gradua\c{c}\~ao em Engenharia El\'etrica e \'Area de Concentra\c{c}\~ao em Telecomunica\c{c}\~oes}
    }{	
% DEHS ==========================================================================
\ifthenelse{\equal{#1}{DEHS}}{
    \tipotrabalho{Tese (Doutorado)}
    \tipotrabalhoabs{Thesis (Doctor)}
    \area{Hidr\'aulica e Saneamento}
	%\opcao{Nome da Opção}
    % O preambulo deve conter o tipo do trabalho, o objetivo, 
	% o nome da instituição, a área de concentração e opção quando houver
	\preambulo{Tese apresentada \`a Escola de Engenharia de S\~ao Carlos da Universidade de S\~ao Paulo, para obten\c{c}\~ao do t\'itulo de Doutor em Ci\^encias - Programa de P\'os-Gradua\c{c}\~ao em Engenharia Hidr\'aulica e Saneamento.}
	\notaficha{Tese (Doutorado) - Programa de P\'os-Gradua\c{c}\~ao e \'Area de Concentra\c{c}\~ao em Engenharia Hidr\'aulica e Saneamento}
    }{
% MEHS ===========================================================================
\ifthenelse{\equal{#1}{MEHS}}{
	\tipotrabalho{Disserta\c{c}\~ao (Mestrado)}
	\tipotrabalhoabs{Dissertation (Master)}
	\area{Hidr\'aulica e Saneamento}
	%\opcao{Nome da Opção}
	% O preambulo deve conter o tipo do trabalho, o objetivo, 
	% o nome da instituição, a área de concentração e opção quando houver
	\preambulo{Disserta\c{c}\~ao apresentada \`a Escola de Engenharia de S\~ao Carlos da Universidade de S\~ao Paulo, para obten\c{c}\~ao do t\'itulo de Mestre em Ci\^encias - Programa de P\'os-Gradua\c{c}\~ao em Hidr\'aulica e Saneamento.}
	\notaficha{Disserta\c{c}\~ao (Mestrado) - Programa de P\'os-Gradua\c{c}\~ao e \'Area de Concentra\c{c}\~ao em Engenharia Hidr\'aulica e Saneamento} 
    }{	
% DEMA ==========================================================================
\ifthenelse{\equal{#1}{DEMA}}{
    \tipotrabalho{Tese (Doutorado)}
    \tipotrabalhoabs{Thesis (Doctor)}
    \area{Aerona\'utica}
	%\opcao{Nome da Opção}
    % O preambulo deve conter o tipo do trabalho, o objetivo, 
	% o nome da instituição, a área de concentração e opção quando houver
	\preambulo{Tese apresentada \`a Escola de Engenharia de S\~ao Carlos da Universidade de S\~ao Paulo, para obten\c{c}\~ao do t\'itulo de Doutor em Ci\^encias - Programa de P\'os-Gradua\c{c}\~ao em Engenharia Mec\^anica.}
	\notaficha{Tese (Doutorado) - Programa de P\'os-Gradua\c{c}\~ao em Engenharia Mec\^anica e \'Area de Concentra\c{c}\~ao em Aerona\'utica}
    }{
% MEMA ===========================================================================
\ifthenelse{\equal{#1}{MEMA}}{
	\tipotrabalho{Disserta\c{c}\~ao (Mestrado)}
	\tipotrabalhoabs{Dissertation (Master)}
	\area{Aerona\'utica}
	%\opcao{Nome da Opção}
	% O preambulo deve conter o tipo do trabalho, o objetivo, 
	% o nome da instituição, a área de concentração e opção quando houver
	\preambulo{Disserta\c{c}\~ao apresentada \`a Escola de Engenharia de S\~ao Carlos da Universidade de S\~ao Paulo, para obten\c{c}\~ao do t\'itulo de Mestre em Ci\^encias - Programa de P\'os-Gradua\c{c}\~ao em Engenharia Mec\^anica.}
	\notaficha{Disserta\c{c}\~ao (Mestrado) - Programa de P\'os-Gradua\c{c}\~ao em Engenharia Mec\^anica e \'Area de Concentra\c{c}\~ao em Aerona\'utica}
    }{	
% DEMD ==========================================================================
\ifthenelse{\equal{#1}{DEMD}}{
    \tipotrabalho{Tese (Doutorado)}
    \tipotrabalhoabs{Thesis (Doctor)}
    \area{Din\^amica e Mecatr\^onica}
	%\opcao{Nome da Opção}
    % O preambulo deve conter o tipo do trabalho, o objetivo, 
	% o nome da instituição, a área de concentração e opção quando houver
	\preambulo{Tese apresentada \`a Escola de Engenharia de S\~ao Carlos da Universidade de S\~ao Paulo, para obten\c{c}\~ao do t\'itulo de Doutor em Ci\^encias - Programa de P\'os-Gradua\c{c}\~ao em Engenharia Mec\^anica.}
	\notaficha{Tese (Doutorado) - Programa de P\'os-Gradua\c{c}\~ao em Engenharia Mec\^anica e \'Area de Concentra\c{c}\~ao em Din\^amica e Mecatr\^onica}
    }{
% MEMD ===========================================================================
\ifthenelse{\equal{#1}{MEMD}}{
	\tipotrabalho{Disserta\c{c}\~ao (Mestrado)}
	\tipotrabalhoabs{Dissertation (Master)}
	\area{Din\^amica e Mecatr\^onica}
	%\opcao{Nome da Opção}
	% O preambulo deve conter o tipo do trabalho, o objetivo, 
	% o nome da instituição, a área de concentração e opção quando houver
	\preambulo{Disserta\c{c}\~ao apresentada \`a Escola de Engenharia de S\~ao Carlos da Universidade de S\~ao Paulo, para obten\c{c}\~ao do t\'itulo de Mestre em Ci\^encias - Programa de P\'os-Gradua\c{c}\~ao em Engenharia Mec\^anica.}
	\notaficha{Disserta\c{c}\~ao (Mestrado) - Programa de P\'os-Gradua\c{c}\~ao em Engenharia Mec\^anica e \'Area de Concentra\c{c}\~ao em Din\^amica e Mecatr\^onica}
    }{	
% DEMF ==========================================================================
\ifthenelse{\equal{#1}{DEMF}}{
    \tipotrabalho{Tese (Doutorado)}
    \tipotrabalhoabs{Thesis (Doctor)}
    \area{Projeto, Materiais e Manufatura}
	%\opcao{Nome da Opção}
    % O preambulo deve conter o tipo do trabalho, o objetivo, 
	% o nome da instituição, a área de concentração e opção quando houver
	\preambulo{Tese apresentada \`a Escola de Engenharia de S\~ao Carlos da Universidade de S\~ao Paulo, para obten\c{c}\~ao do t\'itulo de Doutor em Ci\^encias - Programa de P\'os-Gradua\c{c}\~ao em Engenharia Mec\^anica.}
	\notaficha{Tese (Doutorado) - Programa de P\'os-Gradua\c{c}\~ao em Engenharia Mec\^anica e \'Area de Concentra\c{c}\~ao em Projeto, Materiais e Manufatura}
    }{
% MEMF ===========================================================================
\ifthenelse{\equal{#1}{MEMF}}{
	\tipotrabalho{Disserta\c{c}\~ao (Mestrado)}
	\tipotrabalhoabs{Dissertation (Master)}
	\area{Projeto, Materiais e Manufatura}
	%\opcao{Nome da Opção}
	% O preambulo deve conter o tipo do trabalho, o objetivo, 
	% o nome da instituição, a área de concentração e opção quando houver
	\preambulo{Disserta\c{c}\~ao apresentada \`a Escola de Engenharia de S\~ao Carlos da Universidade de S\~ao Paulo, para obten\c{c}\~ao do t\'itulo de Mestre em Ci\^encias - Programa de P\'os-Gradua\c{c}\~ao em Engenharia Mec\^anica.}
	\notaficha{Disserta\c{c}\~ao (Mestrado) - Programa de P\'os-Gradua\c{c}\~ao em Engenharia Mec\^anica e \'Area de Concentra\c{c}\~ao em Projeto, Materiais e Manufatura}
    }{	
% DEMT ==========================================================================
\ifthenelse{\equal{#1}{DEMT}}{
    \tipotrabalho{Tese (Doutorado)}
    \tipotrabalhoabs{Thesis (Doctor)}
    \area{Termoci\^encias e Mec\^anica de Fluidos}
	%\opcao{Nome da Opção}
    % O preambulo deve conter o tipo do trabalho, o objetivo, 
	% o nome da instituição, a área de concentração e opção quando houver
	\preambulo{Tese apresentada \`a Escola de Engenharia de S\~ao Carlos da Universidade de S\~ao Paulo, para obten\c{c}\~ao do t\'itulo de Doutor em Ci\^encias - Programa de P\'os-Gradua\c{c}\~ao em Engenharia Mec\^anica.}
	\notaficha{Tese (Doutorado) - Programa de P\'os-Gradua\c{c}\~ao em Engenharia Mec\^anica e \'Area de Concentra\c{c}\~ao em Termoci\^encias e Mec\^anica de Fluidos}
    }{
% MEMT ===========================================================================
\ifthenelse{\equal{#1}{MEMT}}{
	\tipotrabalho{Disserta\c{c}\~ao (Mestrado)}
	\tipotrabalhoabs{Dissertation (Master)}
	\area{Termoci\^encias e Mec\^anica de Fluidos}
	%\opcao{Nome da Opção}
	% O preambulo deve conter o tipo do trabalho, o objetivo, 
	% o nome da instituição, a área de concentração e opção quando houver
	\preambulo{Disserta\c{c}\~ao apresentada \`a Escola de Engenharia de S\~ao Carlos da Universidade de S\~ao Paulo, para obten\c{c}\~ao do t\'itulo de Mestre em Ci\^encias - Programa de P\'os-Gradua\c{c}\~ao em Engenharia Mec\^anica.}
	\notaficha{Disserta\c{c}\~ao (Mestrado) - Programa de P\'os-Gradua\c{c}\~ao em Engenharia Mec\^anica e \'Area de Concentra\c{c}\~ao em Termoci\^encias e Mec\^anica de Fluidos}
    }{	
% DCEM ==========================================================================
\ifthenelse{\equal{#1}{DCEM}}{
    \tipotrabalho{Tese (Doutorado)}
    \tipotrabalhoabs{Thesis (Doctor)}
    \area{Caracteriza\c{c}\~ao, Desenvolvimento e Aplica\c{c}\~ao de Materiais}
	%\opcao{Nome da Opção}
    % O preambulo deve conter o tipo do trabalho, o objetivo, 
	% o nome da instituição, a área de concentração e opção quando houver
	\preambulo{Tese apresentada \`a Escola de Engenharia de S\~ao Carlos da Universidade de S\~ao Paulo, para obten\c{c}\~ao do t\'itulo de Doutor em Ci\^encias - Programa de P\'os-Gradua\c{c}\~ao em Ci\^encia e Engenharia de Materiais.}
	\notaficha{Tese (Doutorado) - Programa de P\'os-Gradua\c{c}\~ao em Ci\^encias e Engenharia de Materiais e \'Area de Concentra\c{c}\~ao em Desenvolvimento, Caracteriza\c{c}\~ao e Aplica\c{c}\~ao de Materiais}
    }{
% MCEM ===========================================================================
\ifthenelse{\equal{#1}{MCEM}}{
	\tipotrabalho{Disserta\c{c}\~ao (Mestrado)}
	\tipotrabalhoabs{Dissertation (Master)}
	\area{Caracteriza\c{c}\~ao, Desenvolvimento e Aplica\c{c}\~ao de Materiais}
	%\opcao{Nome da Opção}
	% O preambulo deve conter o tipo do trabalho, o objetivo, 
	% o nome da instituição, a área de concentração e opção quando houver
	\preambulo{Disserta\c{c}\~ao apresentada \`a Escola de Engenharia de S\~ao Carlos da Universidade de S\~ao Paulo, para obten\c{c}\~ao do t\'itulo de Mestre em Ci\^encias - Programa de P\'os-Gradua\c{c}\~ao em Ci\^encia e Engenharia de Materiais.}
	\notaficha{Disserta\c{c}\~ao (Mestrado) - Programa de P\'os-Gradua\c{c}\~ao em Ci\^encias e Engenharia de Materiais e \'Area de Concentra\c{c}\~ao em Desenvolvimento, Caracteriza\c{c}\~ao e Aplica\c{c}\~ao de Materiais}
    }{	
% DGEO ==========================================================================
\ifthenelse{\equal{#1}{DGEO}}{
    \tipotrabalho{Tese (Doutorado)}
    \tipotrabalhoabs{Thesis (Doctor)}
    \area{Geotecnia}
	%\opcao{Nome da Opção}
    % O preambulo deve conter o tipo do trabalho, o objetivo, 
	% o nome da instituição, a área de concentração e opção quando houver
	\preambulo{Tese apresentada \`a Escola de Engenharia de S\~ao Carlos da Universidade de S\~ao Paulo, para obten\c{c}\~ao do t\'itulo de Doutor em Ci\^encias - Programa de P\'os-Gradua\c{c}\~ao em Geotecnia.}
	\notaficha{Tese (Doutorado) - Programa de P\'os-Gradua\c{c}\~ao e \'Area de Concentra\c{c}\~ao em Geotecnia}
    }{
% MGEO ===========================================================================
\ifthenelse{\equal{#1}{MGEO}}{
	\tipotrabalho{Disserta\c{c}\~ao (Mestrado)}
	\tipotrabalhoabs{Dissertation (Master)}
	\area{Geotecnia}
	%\opcao{Nome da Opção}
	% O preambulo deve conter o tipo do trabalho, o objetivo, 
	% o nome da instituição, a área de concentração e opção quando houver
	\preambulo{Disserta\c{c}\~ao apresentada \`a Escola de Engenharia de S\~ao Carlos da Universidade de S\~ao Paulo, para obten\c{c}\~ao do t\'itulo de Mestre em Ci\^encias - Programa de P\'os-Gradua\c{c}\~ao em Geotecnia.}
	\notaficha{Disserta\c{c}\~ao (Mestrado) - Programa de P\'os-Gradua\c{c}\~ao e \'Area de Concentra\c{c}\~ao em Geotecnia}
    }{	
% DIUB ==========================================================================
\ifthenelse{\equal{#1}{DIUB}}{
    \tipotrabalho{Tese (Doutorado)}
    \tipotrabalhoabs{Thesis (Doctor)}
    \area{Bioengenharia}
	%\opcao{Nome da Opção}
    % O preambulo deve conter o tipo do trabalho, o objetivo, 
	% o nome da instituição, a área de concentração e opção quando houver
	\preambulo{Tese apresentada \`a Escola de Engenharia de S\~ao Carlos da Universidade de S\~ao Paulo, para obten\c{c}\~ao do t\'itulo de Doutor em Ci\^encias - Programa de P\'os-Gradua\c{c}\~ao Interunidades em Bioengenharia.}
	\notaficha{Tese (Doutorado) - Programa de P\'os-Gradua\c{c}\~ao Interunidades em Bioengenharia e \'Area de Concentra\c{c}\~ao em Bioengenharia}
    }{
% MIUB ===========================================================================
\ifthenelse{\equal{#1}{MIUB}}{
	\tipotrabalho{Disserta\c{c}\~ao (Mestrado)}
	\tipotrabalhoabs{Dissertation (Master)}
	\area{Bioengenharia}
	%\opcao{Nome da Opção}
	% O preambulo deve conter o tipo do trabalho, o objetivo, 
	% o nome da instituição, a área de concentração e opção quando houver
	\preambulo{Disserta\c{c}\~ao apresentada \`a Escola de Engenharia de S\~ao Carlos da Universidade de S\~ao Paulo, para obten\c{c}\~ao do t\'itulo de Mestre em Ci\^encias - Programa de P\'os-Gradua\c{c}\~ao Interunidades em Bioengenharia.}
	\notaficha{Disserta\c{c}\~ao (Mestrado) - Programa de P\'os-Gradua\c{c}\~ao Interunidades em Bioengenharia e \'Area de Concentra\c{c}\~ao em Bioengenharia}
    }{	
% MRNECA ===========================================================================
\ifthenelse{\equal{#1}{MRNECA}}{
	\tipotrabalho{Disserta\c{c}\~ao (Mestrado)}
	\tipotrabalhoabs{Dissertation (Master)}
	\area{Ensino de Ci\^encias Ambientais}
	%\opcao{Nome da Opção}
	% O preambulo deve conter o tipo do trabalho, o objetivo, 
	% o nome da instituição, a área de concentração e opção quando houver
	\preambulo{Disserta\c{c}\~ao apresentada \`a Escola de Engenharia de S\~ao Carlos da Universidade de S\~ao Paulo, para obten\c{c}\~ao do t\'itulo de Mestre em Ci\^encias - Programa de P\'os-Gradua\c{c}\~ao em Rede Nacional para Ensino das Ci\^encias Ambientais.}
	\notaficha{Disserta\c{c}\~ao (Mestrado) - Programa de P\'os-Gradua\c{c}\~ao em Rede Nacional para Ensino das Ci\^encias Ambientais e \'Area de Concentra\c{c}\~ao em Ensino de Ci\^encias Ambientais}
    }{	         
% Outros
	\tipotrabalho{Disserta\c{c}\~ao/Tese (Mestrado/Doutorado)}
	\tipotrabalhoabs{Dissertation/Thesis (Master/Doctor)}
    \area{Nome da \'Area}
    \opcao{Nome da Op\c{c}\~ao}
    % O preambulo deve conter o tipo do trabalho, o objetivo, 
	% o nome da instituição, a área de concentração e opção quando houver
	\preambulo{Disserta\c{c}\~ao/Tese apresentada \`a Escola de Engenharia de S\~ao Carlos da Universidade de S\~ao Paulo, para obten\c{c}\~ao do t\'itulo de Mestre/Doutor em Ci\^encias - Programa de P\'os-Gradua\c{c}\~ao em Engenharia.}
	\notaficha{Disserta\c{c}\~ao/Tese (Mestrado/Doutorado) - Programa de P\'os-Gradua\c{c}\~ao e \'Area de Concentra\c{c}\~ao em Engenharia}		
    }}}}}}}}}}}}}}}}}}}}}
    }}}}}}}}}}}}}}}}}