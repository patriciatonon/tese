% !TeX spellcheck = pt_BR
\documentclass[tese_patricia]{subfiles}
\begin{document}

% ---------------------------------------------------------- 
% Métodos de malhas sobrepostas
% ----------------------------------------------------------
\chapter{Conclusões Parciais} 
% ----------------------------------------------------------

O principal intuito desse trabalho, conforme foi apresentado nos capítulos anteriores, é a criação de uma ferramenta computacional para análise de interação fluido-estrutura para problemas 2D e 3D. Essa ferramenta conta com uma técnica de partição de domínios para a malha do fluido, fazendo com que se tenha uma malha global menos refinada e fixa e uma malha local deformável e mais refinada e que é gerada de maneira a se levar em consideração efeitos locais na interface entre fluido e estrutura. Dentro do contexto de partição de domínios diferentes aproximações podem ser utilizadas para a local e global.

Até o presente momento deste projeto, conta-se com um versátil código de DFC que possibilita a análise de problemas 2D e 3D, e pode utilizar como aproximação numérica tanto o Método dos Elementos Finitos, quanto a análise Isogeométrica, conforme pode ser visualizado nos exemplos estudados nos Cap. 1 e Cap. 2.

No Cap. 3, demonstrou-se que o código de análise não-linear geométrica de cascas baseado no método dos elementos finitos posicional, cedido pelo professor Humberto Breves Coda, é muito robusto e atende as necessidades deste projeto para as posteriores análises de interação Fluido-Estrutura.

No Cap.4 apresenta-se a técnica de partição de domínios, para levar-se em conta efeitos localizados nas malhas de fluidos. Para validação da metodologia proposta o problema clássico da cavidade 2D, considerando-se o problema estacionário de Navier-Stokes, foi analisado e obtiveram-se resultados ótimos e promissores.

Na sequência desse projeto, a técnica de partição de domínios será ampliada para problemas da DFC com variação temporal, e diferentes possibilidades para os termos estabilizadores serão estudadas. Além disso, o código será ampliado para uma versão tridimensional. 

Ao final do projeto, os programas da DFC com a técnica de partição de domínios e o código de estruturas serão acoplados através de uma técnica de acoplamento particionado forte. Além disso, exemplos de problemas da IFE serão avaliados para a verificação do código proposto neste trabalho de doutorado.





\end{document}
